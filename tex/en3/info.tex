\documentclass{article}
\usepackage[czech]{babel} % Czech language
\usepackage[shortlabels]{enumitem} % Custom enumeration
\usepackage{graphicx} % Import images
\usepackage{float} % Use [H] to force figure position
\usepackage{indentfirst} % Indent first paragraph
\usepackage{emoji} % Emojis
\usepackage{pgffor} % Loops
\usepackage{tikz} % TikZ
\usepackage{pgfplots} % TikZ plots
\usepackage{circuitikz} % Use circuitikz for circuit diagrams
\usepackage{amsmath} % Math
\usetikzlibrary{arrows.meta}

\makeatletter
\providecommand\add@text{}
\newcommand\tagaddtext[1]{%
    \gdef\add@text{#1\gdef\add@text{}}}% 
\renewcommand\tagform@[1]{%
    \maketag@@@{\llap{\add@text\quad}(\ignorespaces#1\unskip\@@italiccorr)}%
}
\makeatother

\newcommand{\cvHead}[1]{\head{Cvičení #1}}

\newcommand{\head}[1]{
    \title{\textbf{#1}\\Elektroenergetika 3}
    \author{Petr Jílek}
    \date{2024}
}

\newcommand{\spicy}{\emoji{hot-pepper}}

% My font space
\newcommand{\myFS}{\;}

% Text in math mode
\newcommand{\te}[1]{\textrm{#1}}

% --------------------
% Units
% --------------------

\newcommand{\uM}{\textrm{m}} % Meter
\newcommand{\uMsq}{\uM^\textrm{2}} % Meter squared
\newcommand{\uMcu}{\uM^\textrm{3}} % Meter cubed
\newcommand{\uS}{\textrm{s}} % Second
\newcommand{\uKG}{\textrm{kg}} % Kilogram
\newcommand{\uJ}{\textrm{J}} % Joule
\newcommand{\uK}{\textrm{K}} % Kelvin
\newcommand{\uDEGREE}{^\circ} % Degree
\newcommand{\uCELS}{\uDEGREE \textrm{C}} % Celsius
\newcommand{\uW}{\textrm{W}} % Watt
\newcommand{\uMW}{\textrm{MW}} % Mega Watt
\newcommand{\uKW}{\textrm{kW}} % Kilo Watt
\newcommand{\uWH}{\textrm{Wh}} % Watt hour
\newcommand{\uKWH}{\textrm{kWh}} % Kilo Watt hour
\newcommand{\uCCY}{\textrm{CCY}} % Currency
\newcommand{\uCZK}{\textrm{CZK}} % Czech Crown
\newcommand{\uPERCENT}{\textrm{\%}} % Percent
\newcommand{\uNOUNIT}{\textrm{--}} % No unit
\newcommand{\uYEAR}{\textrm{year}} % Year
\newcommand{\uMONTH}{\textrm{month}} % Month
\newcommand{\uHOUR}{\textrm{hour}} % Hour
\newcommand{\uLM}{\textrm{lm}} % Lumen
\newcommand{\uHinv}{\textrm{h}^{-1}} % Hour inverse

\newcommand{\uJperK}{\uJ / \uK} % Joule per Kelvin
\newcommand{\uKperM}{\uK / \uM} % Kelvin per Meter
\newcommand{\uKperS}{\uK / \uS} % Kelvin per Second
\newcommand{\uMsqperS}{\uMsq/\uS} % Meter squared per Second
\newcommand{\uWperMsq}{\uW / \uMsq} % Watt per Meter squared
\newcommand{\uWperMperK}{\uW / \left( \uM \cdot \uK \right)} % Watt per Meter per Kelvin
\newcommand{\uJperKGperK}{\uJ / \left( \uKG \cdot \uK \right)} % Joule per Kilogram per Kelvin
\newcommand{\uKperMsq}{\uK / \uMsq} % Kelvin per Meter squared
\newcommand{\uJperSperMperK}{\uJ / \left( \uS \cdot \uM \cdot \uK \right)} % Joule per Second per Meter per Kelvin
\newcommand{\uKGperMcu}{\uKG / \uMcu} % Kilogram per Meter cubed
\newcommand{\uMsqKperW}{\uMsq \cdot \uK / \uW} % Meter squared Kelvin per Watt
\newcommand{\uWperMsqperK}{\uW / \left( \uMsq \cdot \uK \right)} % Watt per Meter squared per Kelvin
\newcommand{\uKWHperMsqperYEAR}{\uKWH / \left( \uMsq \cdot \uYEAR \right)} % Kilo Watt hour per Meter squared per Year
\newcommand{\uKWHperMsq}{\uKWH / \uMsq} % Kilo Watt hour per Meter squared
\newcommand{\uLMperW}{\uLM / \uW} % Lumen per Watt
\newcommand{\uKWperMsq}{\uKW / \uMsq} % Kilo Watt per Meter squared
\newcommand{\uCCYperYEAR}{\uCCY / \uYEAR} % Currency per Year
\newcommand{\uKWHperYEAR}{\uKWH / \uYEAR} % Kilo Watt hour per Year
\newcommand{\uNOUNITperYEAR}{\uNOUNIT / \uYEAR} % No unit per Year
\newcommand{\uCCYperKWH}{\uCCY / \uKWH} % Currency per Kilo Watt hour
\newcommand{\uCZKperYEAR}{\uCZK / \uYEAR} % Czech Crown per Year
\newcommand{\uCZKperKWH}{\uCZK / \uKWH} % Czech Crown per Kilo Watt hour
\newcommand{\uCZKperMONTH}{\uCZK / \uMONTH} % Czech Crown per Month



% --------------------
% Unit equations
% --------------------

\newcommand{\ueqM}{$\uM$}
\newcommand{\ueqMsq}{$\uMsq$}
\newcommand{\ueqMcu}{$\uMcu$}
\newcommand{\ueqS}{$\uS$}
\newcommand{\ueqJ}{$\uJ$}
\newcommand{\ueqK}{$\uK$}
\newcommand{\ueqDEGREE}{$\uDEGREE$}
\newcommand{\ueqCELS}{$\uCELS$}
\newcommand{\ueqW}{$\uW$}
\newcommand{\ueqMW}{$\uMW$}
\newcommand{\ueqKW}{$\uKW$}
\newcommand{\ueqWH}{$\uWH$}
\newcommand{\ueqKWH}{$\uKWH$}
\newcommand{\ueqCCY}{$\uCCY$}
\newcommand{\ueqCZK}{$\uCZK$}
\newcommand{\ueqPERCENT}{$\uPERCENT$}
\newcommand{\ueqNOUNIT}{$\uNOUNIT$}
\newcommand{\ueqYEAR}{$\uYEAR$}
\newcommand{\ueqMONTH}{$\uMONTH$}
\newcommand{\ueqHOUR}{$\uHOUR$}
\newcommand{\ueqLM}{$\uLM$}
\newcommand{\ueqHinv}{$\uHinv$}

\newcommand{\ueqJperK}{$\uJperK$}
\newcommand{\ueqKperM}{$\uKperM$}
\newcommand{\ueqKperS}{$\uKperS$}
\newcommand{\ueqMsqperS}{$\uMsqperS$}
\newcommand{\ueqWperMsq}{$\uWperMsq$}
\newcommand{\ueqWperMperK}{$\uWperMperK$}
\newcommand{\ueqJperKGperK}{$\uJperKGperK$}
\newcommand{\ueqKperMsq}{$\uKperMsq$}
\newcommand{\ueqJperSperMperK}{$\uJperSperMperK$}
\newcommand{\ueqKGperMcu}{$\uKGperMcu$}
\newcommand{\ueqMsqKperW}{$\uMsqKperW$}
\newcommand{\ueqWperMsqperK}{$\uWperMsqperK$}
\newcommand{\ueqKWHperMsqperYEAR}{$\uKWHperMsqperYEAR$}
\newcommand{\ueqKWHperMsq}{$\uKWHperMsq$}
\newcommand{\ueqLMperW}{$\uLMperW$}
\newcommand{\ueqKWperMsq}{$\uKWperMsq$}
\newcommand{\ueqCCYperYEAR}{$\uCCYperYEAR$}
\newcommand{\ueqKWHperYEAR}{$\uKWHperYEAR$}
\newcommand{\ueqNOUNITperYEAR}{$\uNOUNITperYEAR$}
\newcommand{\ueqCCYperKWH}{$\uCCYperKWH$}
\newcommand{\ueqCZKperYEAR}{$\uCZKperYEAR$}
\newcommand{\ueqCZKperKWH}{$\uCZKperKWH$}
\newcommand{\ueqCZKperMONTH}{$\uCZKperMONTH$}


\head{Info}

\begin{document}

\maketitle
\tableofcontents
\newpage




\section{Kontakt}
Přednášející: Jan Kyncl
\begin{itemize}
    \item email: \href{mailto:kyncl@fel.cvut.cz}{kyncl@fel.cvut.cz}
\end{itemize}

Cvičící: Petr Jílek
\begin{itemize}
    \item email: \href{mailto:jilekpet@fel.cvut.cz}{jilekpet@fel.cvut.cz}
\end{itemize}




\section{Harmonogram}
Celkem bude 7 cvičení. Struktura cvičení bude následující:
\begin{itemize}
    \item 5 cvičení,
    \item 1 příprava na test,
    \item 1 test.
\end{itemize}




\section{Materiály}
Materiály jsou dostupné na Moodle. Struktura materiálů je následující:
\begin{itemize}
    \item \texttt{teorie.pdf} - základní teorie,
    \item ukázkový test:
          \begin{enumerate}
              \item \texttt{test\_1\_1\_zadani.pdf} - zadání ukázkového testu,
              \item \texttt{test\_1\_1\_reseni.pdf} - řešení ukázkového testu,
          \end{enumerate}
    \item cvičení:
          \begin{enumerate}
              \item \texttt{cv1.pdf} - Energie,
              \item \texttt{cv2.pdf} - Sdílení tepla,
              \item \texttt{cv3.pdf} - Sdílení tepla - Válec a koule,
              \item \texttt{cv4.pdf} - Symetrizace,
              \item \texttt{cv5.pdf} - Aplikace.
          \end{enumerate}
\end{itemize}

Ve cvičeních jsou sekce označené úrovní obtížnosti pomocí ikonek \spicy. Počet ikonek značí obtížnost úlohy. Čím více ikonek, tím obtížnější úloha.

Pokud obtížnost subsekce není označena, tak se bere obtížnost nadřazené sekce, nebo nadřazené subsekce (obtížnost se propisuje dolů). Pokud je příslušná subsekce označena obtížností, tak tato obtížnost má přednost a je brána jako platná. Dále se tato obtížnost propisuje do subsekcí této subsekce.

\begin{notebox}{Poznámka}
    \begin{itemize}
        \item Pokud subsekce \textbf{není označena} obtížností, tak se kouknětě \textbf{výš}.
        \item Pokud subsekce \textbf{je označena} obtížností, tak tato obtížnost \textbf{má přednost} před obtížnostmi výše.
    \end{itemize}
\end{notebox}




\section{Test}
V testu budou příklady podobné příkladům ze cvičení s úrovní \spicy, nebo \spicy \spicy. Vyšší úrovně v testu nebudou. Z testu bude možno získat maximálně 50 bodů. Struktura testu a bodování bude následující:
\begin{itemize}
    \item vyplnění jména a příjmení (2 body),
    \item číselný příklad (16 bodů),
    \item číselný příklad na symetrizaci (16 bodů),
    \item teoretická otázka z přednášky (16 bodů).
\end{itemize}




\section{GitHub}
Nejaktuálnější verze materiálů je dostupná na GitHubu. Odkaz na repozitář je: \href{https://github.com/petr-jilek/ctu}{https://github.com/petr-jilek/ctu}.\\

Pro naclonování repozitáře použijte následující příkaz:
\begin{verbatim}
    git clone git@github.com:petr-jilek/ctu.git
\end{verbatim}

Pokud jste v materiálech našli chybu, nebo máte návrh na vylepšení, nebo byste chtěli přidat nový obsah, tak můžete vytvořit \textbf{issue} na GitHubu, případně vytvořit separátní branch a následně vytvořit \textbf{pull request}.




\section{Poznámka}
Změny jsou vyhrazeny.

\end{document}
