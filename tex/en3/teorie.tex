\documentclass{article}
\usepackage[czech]{babel} % Czech language
\usepackage[shortlabels]{enumitem} % Custom enumeration
\usepackage{graphicx} % Import images
\usepackage{float} % Use [H] to force figure position
\usepackage{indentfirst} % Indent first paragraph
\usepackage{emoji} % Emojis
\usepackage{pgffor} % Loops
\usepackage{tikz} % TikZ
\usepackage{pgfplots} % TikZ plots
\usepackage{circuitikz} % Use circuitikz for circuit diagrams
\usepackage{amsmath} % Math
\usetikzlibrary{arrows.meta}

\makeatletter
\providecommand\add@text{}
\newcommand\tagaddtext[1]{%
    \gdef\add@text{#1\gdef\add@text{}}}% 
\renewcommand\tagform@[1]{%
    \maketag@@@{\llap{\add@text\quad}(\ignorespaces#1\unskip\@@italiccorr)}%
}
\makeatother

\newcommand{\cvHead}[1]{\head{Cvičení #1}}

\newcommand{\head}[1]{
    \title{\textbf{#1}\\Elektroenergetika 3}
    \author{Petr Jílek}
    \date{2024}
}

\newcommand{\spicy}{\emoji{hot-pepper}}

% My font space
\newcommand{\myFS}{\;}

% Text in math mode
\newcommand{\te}[1]{\textrm{#1}}

% --------------------
% Units
% --------------------

\newcommand{\uM}{\textrm{m}} % Meter
\newcommand{\uMsq}{\uM^\textrm{2}} % Meter squared
\newcommand{\uMcu}{\uM^\textrm{3}} % Meter cubed
\newcommand{\uS}{\textrm{s}} % Second
\newcommand{\uKG}{\textrm{kg}} % Kilogram
\newcommand{\uJ}{\textrm{J}} % Joule
\newcommand{\uK}{\textrm{K}} % Kelvin
\newcommand{\uDEGREE}{^\circ} % Degree
\newcommand{\uCELS}{\uDEGREE \textrm{C}} % Celsius
\newcommand{\uW}{\textrm{W}} % Watt
\newcommand{\uMW}{\textrm{MW}} % Mega Watt
\newcommand{\uKW}{\textrm{kW}} % Kilo Watt
\newcommand{\uWH}{\textrm{Wh}} % Watt hour
\newcommand{\uKWH}{\textrm{kWh}} % Kilo Watt hour
\newcommand{\uCCY}{\textrm{CCY}} % Currency
\newcommand{\uCZK}{\textrm{CZK}} % Czech Crown
\newcommand{\uPERCENT}{\textrm{\%}} % Percent
\newcommand{\uNOUNIT}{\textrm{--}} % No unit
\newcommand{\uYEAR}{\textrm{year}} % Year
\newcommand{\uMONTH}{\textrm{month}} % Month
\newcommand{\uHOUR}{\textrm{hour}} % Hour
\newcommand{\uLM}{\textrm{lm}} % Lumen
\newcommand{\uHinv}{\textrm{h}^{-1}} % Hour inverse

\newcommand{\uJperK}{\uJ / \uK} % Joule per Kelvin
\newcommand{\uKperM}{\uK / \uM} % Kelvin per Meter
\newcommand{\uKperS}{\uK / \uS} % Kelvin per Second
\newcommand{\uMsqperS}{\uMsq/\uS} % Meter squared per Second
\newcommand{\uWperMsq}{\uW / \uMsq} % Watt per Meter squared
\newcommand{\uWperMperK}{\uW / \left( \uM \cdot \uK \right)} % Watt per Meter per Kelvin
\newcommand{\uJperKGperK}{\uJ / \left( \uKG \cdot \uK \right)} % Joule per Kilogram per Kelvin
\newcommand{\uKperMsq}{\uK / \uMsq} % Kelvin per Meter squared
\newcommand{\uJperSperMperK}{\uJ / \left( \uS \cdot \uM \cdot \uK \right)} % Joule per Second per Meter per Kelvin
\newcommand{\uKGperMcu}{\uKG / \uMcu} % Kilogram per Meter cubed
\newcommand{\uMsqKperW}{\uMsq \cdot \uK / \uW} % Meter squared Kelvin per Watt
\newcommand{\uWperMsqperK}{\uW / \left( \uMsq \cdot \uK \right)} % Watt per Meter squared per Kelvin
\newcommand{\uKWHperMsqperYEAR}{\uKWH / \left( \uMsq \cdot \uYEAR \right)} % Kilo Watt hour per Meter squared per Year
\newcommand{\uKWHperMsq}{\uKWH / \uMsq} % Kilo Watt hour per Meter squared
\newcommand{\uLMperW}{\uLM / \uW} % Lumen per Watt
\newcommand{\uKWperMsq}{\uKW / \uMsq} % Kilo Watt per Meter squared
\newcommand{\uCCYperYEAR}{\uCCY / \uYEAR} % Currency per Year
\newcommand{\uKWHperYEAR}{\uKWH / \uYEAR} % Kilo Watt hour per Year
\newcommand{\uNOUNITperYEAR}{\uNOUNIT / \uYEAR} % No unit per Year
\newcommand{\uCCYperKWH}{\uCCY / \uKWH} % Currency per Kilo Watt hour
\newcommand{\uCZKperYEAR}{\uCZK / \uYEAR} % Czech Crown per Year
\newcommand{\uCZKperKWH}{\uCZK / \uKWH} % Czech Crown per Kilo Watt hour
\newcommand{\uCZKperMONTH}{\uCZK / \uMONTH} % Czech Crown per Month



% --------------------
% Unit equations
% --------------------

\newcommand{\ueqM}{$\uM$}
\newcommand{\ueqMsq}{$\uMsq$}
\newcommand{\ueqMcu}{$\uMcu$}
\newcommand{\ueqS}{$\uS$}
\newcommand{\ueqJ}{$\uJ$}
\newcommand{\ueqK}{$\uK$}
\newcommand{\ueqDEGREE}{$\uDEGREE$}
\newcommand{\ueqCELS}{$\uCELS$}
\newcommand{\ueqW}{$\uW$}
\newcommand{\ueqMW}{$\uMW$}
\newcommand{\ueqKW}{$\uKW$}
\newcommand{\ueqWH}{$\uWH$}
\newcommand{\ueqKWH}{$\uKWH$}
\newcommand{\ueqCCY}{$\uCCY$}
\newcommand{\ueqCZK}{$\uCZK$}
\newcommand{\ueqPERCENT}{$\uPERCENT$}
\newcommand{\ueqNOUNIT}{$\uNOUNIT$}
\newcommand{\ueqYEAR}{$\uYEAR$}
\newcommand{\ueqMONTH}{$\uMONTH$}
\newcommand{\ueqHOUR}{$\uHOUR$}
\newcommand{\ueqLM}{$\uLM$}
\newcommand{\ueqHinv}{$\uHinv$}

\newcommand{\ueqJperK}{$\uJperK$}
\newcommand{\ueqKperM}{$\uKperM$}
\newcommand{\ueqKperS}{$\uKperS$}
\newcommand{\ueqMsqperS}{$\uMsqperS$}
\newcommand{\ueqWperMsq}{$\uWperMsq$}
\newcommand{\ueqWperMperK}{$\uWperMperK$}
\newcommand{\ueqJperKGperK}{$\uJperKGperK$}
\newcommand{\ueqKperMsq}{$\uKperMsq$}
\newcommand{\ueqJperSperMperK}{$\uJperSperMperK$}
\newcommand{\ueqKGperMcu}{$\uKGperMcu$}
\newcommand{\ueqMsqKperW}{$\uMsqKperW$}
\newcommand{\ueqWperMsqperK}{$\uWperMsqperK$}
\newcommand{\ueqKWHperMsqperYEAR}{$\uKWHperMsqperYEAR$}
\newcommand{\ueqKWHperMsq}{$\uKWHperMsq$}
\newcommand{\ueqLMperW}{$\uLMperW$}
\newcommand{\ueqKWperMsq}{$\uKWperMsq$}
\newcommand{\ueqCCYperYEAR}{$\uCCYperYEAR$}
\newcommand{\ueqKWHperYEAR}{$\uKWHperYEAR$}
\newcommand{\ueqNOUNITperYEAR}{$\uNOUNITperYEAR$}
\newcommand{\ueqCCYperKWH}{$\uCCYperKWH$}
\newcommand{\ueqCZKperYEAR}{$\uCZKperYEAR$}
\newcommand{\ueqCZKperKWH}{$\uCZKperKWH$}
\newcommand{\ueqCZKperMONTH}{$\uCZKperMONTH$}


\head{Teorie}

\begin{document}

\maketitle
\tableofcontents
\newpage



\section{Značení}

\begin{itemize}
    \item $T$ - teplota (\ueqK \fs -- kelvin / \ueqCELS \fs - stupeň celsia)
    \item $\Delta T$ - rozdíl teplot (\ueqK \fs -- kelvin)
    \item $l$ - délka (\ueqM \fs -- metr)
    \item $h$ - výška (\ueqM \fs -- metr)
    \item $d$ - tloušťka / Průměr (\ueqM \fs -- metr)
    \item $r$ - poloměr (\ueqM \fs -- metr)
    \item $S$ - plocha (\ueqMsq \fs -- metr čtvereční)
    \item $c$ - měrná tepelná kapacita (\ueqJandKGinvKinv \fs -- joule na kilogram na kelvin)
    \item $\rho$ - hustota (\ueqKGandMinvcu \fs -- kilogram na metr krychlový)
    \item $m$ - hmotnost (\ueqKG \fs -- kilogram)
    \item $\dot{q}$ - měrný tepelný tok (\ueqWandMinvsq \fs -- watt)
    \item $q$ - měrná tepelná energie (\ueqJandMinvsq \fs -- joule na kilogram)
    \item $\dot{Q}$ - tepelný tok (\ueqW \fs -- watt)
    \item $Q$ - tepelná energie (\ueqJ \fs -- joule)
    \item $v$ - rychlost (\ueqMandSinv \fs -- metr za sekundu)
    \item $a$ - zrychlení (\ueqMandSinvsq \fs -- metr za sekundu na druhou)
    \item $g$ - gravitační zrychlení (\ueqMandSinvsq \fs -- metr za sekundu na druhou)
    \item $V$ - objem (\ueqMcu \fs -- metr krychlový)
    \item $P$ - výkon (\ueqW \fs -- watt)
    \item $t$ - čas (\ueqS \fs -- sekunda)
    \item $\lambda$ - tepelná vodivost (\ueqWandMinvKinv \fs -- watt na metr na kelvin)
    \item $\alpha_{\vartheta}$ - součinitel prostupu tepla (\ueqWandMinvsqKinv \fs -- watt na metr čtvereční na kelvin)
    \item $R_{\vartheta}$ - tepelný odpor (\ueqMsqKandWinv \fs -- metr čtvereční kelvin na watt)
    \item $\rho_e$ - měrný elektrický odpor (\ueqOHMandMinv \fs -- ohm na metr)
    \item $\sigma_e$ - měrná elektrická vodivost (\ueqMandOHMinv \fs -- metr na ohm)
    \item $U$ - elektrické napětí (\ueqV \fs -- volt)
    \item $I$ - elektrický proud (\ueqA \fs -- ampér)
    \item $R_e$ - elektrický odpor (\ueqOHM \fs -- ohm)
    \item $E$ - energie (\ueqJ \fs -- joule)
    \item $E_p$ - potenciální energie (\ueqJ \fs -- joule)
    \item $E_k$ - kinetická energie (\ueqJ \fs -- joule)
\end{itemize}

\newpage



\section{Konstanty}

\begin{itemize}
    \item gravitační zrychlení: $g = 9,81$ \ueqMandSinvsq \fs -- metr za sekundu na druhou
    \item boltzmannova konstanta: $k_B = 1,38 \cdot 10^{-23}$ \ueqJandKinv \fs -- joule na kelvin
\end{itemize}

\begin{table}[H]
    \centering
    \begin{tabular}{l|ll}
        \hline
        Mateiál    & $\rho$ (\ueqKGandMinvcu) & $c$ (\ueqJandKGinvKinv) \\
        \hline
        Voda (H2O) & 1 000                    & 4 186                   \\
        Ocel       & 7 750                    & 450                     \\
        Zlato      & 19 320                   & 129                     \\
        \hline
    \end{tabular}
    \caption {Hustota a měrná tepelná kapacita materiálů}
\end{table}

\newpage



\section{Energie}


\subsection{Potenciální energie}
Potenciální energie je energie, kterou má těleso v důsledku své polohy v gravitačním poli. Vztah pro výpočet potenciální energie je:
\begin{equation}
    E_p = m \cdot g \cdot h,
    \tagaddtext{(\ueqJ)}
\end{equation}
kde:\\
$E_p$ -- potenciální energie (\ueqJ),\\
$m$ -- hmotnost (\ueqKG),\\
$g$ -- gravitační zrychlení (\ueqMandSinvsq),\\
$h$ -- výška (\ueqM).


\subsection{Kinetická energie}
Kinetická energie je energie, kterou má těleso v důsledku své rychlosti. Vztah pro výpočet kinetické energie je:
\begin{equation}
    E_k = \frac{1}{2} \cdot m \cdot v^2,
    \tagaddtext{(\ueqJ)}
\end{equation}
kde:\\
$E_k$ -- kinetická energie (\ueqJ),\\
$m$ -- hmotnost (\ueqKG),\\
$v$ -- rychlost (\ueqMandSinv).


\subsection{Měrná tepelná kapacita}
Měrná tepelná kapacita je definována jako množství tepla, které je potřeba k ohřátí jednoho kilogramu látky o jeden stupeň Kelvina:
\begin{equation}
    Q = m \cdot c \cdot \Delta T,
    \tagaddtext{(\ueqJ)}
\end{equation}
kde:\\
$Q$ -- tepelná energie (\ueqJ),\\
$m$ -- hmotnost (\ueqKG),\\
$c$ -- měrná tepelná kapacita (\ueqJandKGinvKinv),\\
$\Delta T$ -- rozdíl teplot (\ueqK).


\subsection{Výkon}
Výkon je definován jako množství práce vykonané za jednotku času:
\begin{equation}
    P = \frac{dW}{dt},
    \tagaddtext{(\ueqW)}
\end{equation}
kde:\\
$P$ -- výkon (\ueqW),\\
$dW$ -- infinitesimální práce (\ueqJ),\\
$dt$ -- infinitesimální čas (\ueqS).

\newpage



\section{Sdílení tepla}


\subsection{Fourierova-Kirchhoffova rovnice}
Fourierova-Kirchhoffova rovnice je základní rovnicí pro popis toku tepla:
\begin{equation}
    \rho \cdot c \cdot \left( \frac{\partial T}{\partial t} + \vec{v} \cdot \vec{\nabla} T \right) = \nabla \cdot \left( \lambda \cdot \vec{\nabla} T \right) + Q_v,
    \tagaddtext{(\ueqWandMinvcu)}
\end{equation}
kde:\\
$\rho$ -- hustota (\ueqKGandMinvcu),\\
$c$ -- měrná tepelná kapacita (\ueqJandKGinvKinv),\\
$T$ -- teplota (\ueqK),\\
$t$ -- čas (\ueqS),\\
$\vec{v}$ -- rychlost (\ueqMandSinv),\\
$\vec{\nabla} T$ -- gradient teploty (\ueqKandMinv),\\
$\nabla \cdot$ -- divergence (\ueqMinv),\\
$\lambda$ -- tepelná vodivost (\ueqWandMinvKinv),\\
$Q_v$ -- objemový zdroj tepla (\ueqWandMinvcu).


\subsection{Fourieruv zákon}
Fourieruv zákon je základní rovnicí pro popis toku tepla:
\begin{equation}
    \vec{\dot{q}} = - \lambda \cdot \vec{\nabla} T,
    \tagaddtext{(\ueqWandMinvsq)}
\end{equation}
kde:\\
$\vec{\dot{q}}$ -- měrný tepelný tok (\ueqWandMinvsq),\\
$\lambda$ -- tepelná vodivost (\ueqWandMinvKinv),\\
$\vec{\nabla} T$ -- gradient teploty (\ueqKandMinv).

\newpage



\section{Symetrizace}

\newpage



\section{Aplikace}




\end{document}
