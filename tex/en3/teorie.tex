\documentclass{article}
\usepackage[czech]{babel} % Czech language
\usepackage[shortlabels]{enumitem} % Custom enumeration
\usepackage{graphicx} % Import images
\usepackage{float} % Use [H] to force figure position
\usepackage{indentfirst} % Indent first paragraph
\usepackage{emoji} % Emojis
\usepackage{pgffor} % Loops
\usepackage{tikz} % TikZ
\usepackage{pgfplots} % TikZ plots
\usepackage{amsmath} % Math
\usetikzlibrary{arrows.meta}

\makeatletter
\providecommand\add@text{}
\newcommand\tagaddtext[1]{%
    \gdef\add@text{#1\gdef\add@text{}}}% 
\renewcommand\tagform@[1]{%
    \maketag@@@{\llap{\add@text\quad}(\ignorespaces#1\unskip\@@italiccorr)}%
}
\makeatother

\newcommand{\cvHead}[1]{\head{Cvičení #1}}

\newcommand{\head}[1]{
    \title{\textbf{#1}\\Elektroenergetika 3}
    \author{Petr Jílek}
    \date{2024}
}

\newcommand{\spicy}{\emoji{hot-pepper}}

% --------------------
% --------------------
% Units
% --------------------
% --------------------

% --------------------
% Basic units
% --------------------  

% no unit
\newcommand{\uNOUNIT}{\te{--}} % no unit
\newcommand{\uPERCENT}{\te{\%}} % percent
\newcommand{\uDEGREE}{^\circ} % degree
\newcommand{\uRAD}{\te{rad}} % radian

% meter
\newcommand{\uM}{\te{m}} % meter
\newcommand{\uMsq}{\uM^\te{2}} % meter squared
\newcommand{\uMcu}{\uM^\te{3}} % meter cubed
\newcommand{\uMquar}{\uM^\te{4}} % meter quartic
\newcommand{\uMinv}{\uM^{-1}} % meter inverse
\newcommand{\uMinvsq}{\uM^{-2}} % meter inverse squared
\newcommand{\uMinvcu}{\uM^{-3}} % meter inverse cubed
\newcommand{\uMinvquar}{\uM^{-4}} % meter inverse quartic
\newcommand{\uMM}{\te{mm}} % millimeter
\newcommand{\uCM}{\te{cm}} % centimeter
\newcommand{\uKM}{\te{km}} % kilometer

% liter
\newcommand{\uL}{\te{l}} % liter
\newcommand{\uML}{\te{ml}} % milliliter

% second
\newcommand{\uS}{\te{s}} % second
\newcommand{\uSinv}{\uS^{-1}} % second inverse
\newcommand{\uSinvsq}{\uS^{-2}} % second inverse squared

% hour
\newcommand{\uH}{\te{h}} % hour
\newcommand{\uHinv}{\te{h}^{-1}} % hour inverse

% kilogram
\newcommand{\uKG}{\te{kg}} % kilogram
\newcommand{\uKGinv}{\uKG^{-1}} % kilogram inverse
\newcommand{\uKGinvsq}{\uKG^{-2}} % kilogram inverse squared
\newcommand{\uG}{\te{g}} % gram

% lumen
\newcommand{\uLM}{\te{lm}} % lumen

% kelvin
\newcommand{\uK}{\te{K}} % kelvin
\newcommand{\uKsq}{\uK^\te{2}} % kelvin squared
\newcommand{\uKcu}{\uK^\te{3}} % kelvin cubed
\newcommand{\uKquar}{\uK^\te{4}} % kelvin quartic
\newcommand{\uKinv}{\uK^{-1}} % kelvin inverse
\newcommand{\uKinvsq}{\uK^{-2}} % kelvin inverse squared
\newcommand{\uKinvcu}{\uK^{-3}} % kelvin inverse cubed
\newcommand{\uKinvquar}{\uK^{-4}} % kelvin inverse quartic

% degree celsius
\newcommand{\uCELS}{\uDEGREE \te{C}} % degree celsius

% watt
\newcommand{\uW}{\te{W}} % watt
\newcommand{\uKW}{\te{kW}} % kilowatt
\newcommand{\uMW}{\te{MW}} % megawatt
\newcommand{\uGW}{\te{GW}} % gigawatt
\newcommand{\uWinv}{\uW^{-1}} % watt inverse

% joule
\newcommand{\uJ}{\te{J}} % joule
\newcommand{\uKJ}{\te{kJ}} % kilojoule
\newcommand{\uMJ}{\te{MJ}} % megajoule
\newcommand{\uCAL}{\te{cal}} % calorie
\newcommand{\uKCAL}{\te{kcal}} % kilocalorie
\newcommand{\uWH}{\te{Wh}} % watt hour
\newcommand{\uKWH}{\te{kWh}} % kilowatt hour
\newcommand{\uWandS}{\te{Ws}} % watt second

% ampere
\newcommand{\uA}{\te{A}} % ampere

% volt
\newcommand{\uV}{\te{V}} % volt

% ohm
\newcommand{\uOHM}{\te{\Omega}} % ohm
\newcommand{\uOHMinv}{\uOHM^{-1}} % ohm inverse

% siemens
\newcommand{\uSIE}{\te{S}} % siemens
\newcommand{\uSIEinv}{\uSIE^{-1}} % siemens inverse

% newton
\newcommand{\uN}{\te{N}} % newton

% currency
\newcommand{\uCCY}{\te{CCY}} % currency
\newcommand{\uCZK}{\te{CZK}} % czech crown

% --------------------
% Compound units
% --------------------

% velocity-acceleration
\newcommand{\uKGandMinvcu}{\uKG \cdot \uMinvcu} % kilogram per meter cubed
\newcommand{\uMandSinv}{\uM \cdot \uSinv} % meter per second
\newcommand{\uMcuSinv}{\uMcu \cdot \uSinv} % meter cubed per second
\newcommand{\uMandSinvsq}{\uM \cdot \uSinvsq} % meter per second squared

% power-energy
\newcommand{\uWandMinvsq}{\uW \cdot \uMinvsq} % watt per meter squared
\newcommand{\uWandMinvcu}{\uW \cdot \uMinvcu} % watt per meter cubed
\newcommand{\uJandMinvsq}{\uJ \cdot \uMinvsq} % joule per meter squared
\newcommand{\uJandMinvcu}{\uJ \cdot \uMinvcu} % joule per meter cubed

% heat
\newcommand{\uJandKGinvKinv}{\uJ \cdot \uKGinv \cdot \uKinv} % joule per kilogram per kelvin
\newcommand{\uWandMinvKinv}{\uW \cdot \uMinv \cdot \uKinv} % watt per meter per kelvin
\newcommand{\uMsqKandWinv}{\uMsq \cdot \uK \cdot \uWinv} % meter squared kelvin per watt
\newcommand{\uKandWinv}{\uK \cdot \uWinv} % kelvin per watt inverse
\newcommand{\uWandMinvsqKinv}{\uW \cdot \uMinvsq \cdot \uKinv} % watt per meter squared per kelvin
\newcommand{\uWandKinv}{\uW \cdot \uKinv} % watt per kelvin inverse

% electricity
\newcommand{\uOHMandMinv}{\uOHM \cdot \uMinv} % ohm per meter
\newcommand{\uSIEandMinv}{\uSIE \cdot \uMinv} % siemens per meter
\newcommand{\uAandMinvsq}{\uA \cdot \uMinvsq} % ampere per meter squared
\newcommand{\uVandMinv}{\uV \cdot \uMinv} % volt per meter


% --------------------
% --------------------
% Units in equation
% --------------------
% --------------------

% --------------------
% Basic units
% --------------------

% no unit
\newcommand{\ueqNOUNIT}{$\uNOUNIT$} % no unit
\newcommand{\ueqPERCENT}{$\uPERCENT$} % percent
\newcommand{\ueqDEGREE}{$\uDEGREE$} % degree
\newcommand{\ueqRAD}{$\uRAD$} % radian

% meter
\newcommand{\ueqM}{$\uM$} % meter
\newcommand{\ueqMsq}{$\uMsq$} % meter squared
\newcommand{\ueqMcu}{$\uMcu$} % meter cubed
\newcommand{\ueqMquar}{$\uMquar$} % meter quartic
\newcommand{\ueqMinv}{$\uMinv$} % meter inverse
\newcommand{\ueqMinvsq}{$\uMinvsq$} % meter inverse squared
\newcommand{\ueqMinvcu}{$\uMinvcu$} % meter inverse cubed
\newcommand{\ueqMinvquar}{$\uMinvquar$} % meter inverse quartic
\newcommand{\ueqMM}{$\uMM$} % millimeter
\newcommand{\ueqCM}{$\uCM$} % centimeter
\newcommand{\ueqKM}{$\uKM$} % kilometer

% liter
\newcommand{\ueqL}{$\uL$} % liter
\newcommand{\ueqML}{$\uML$} % milliliter

% second
\newcommand{\ueqS}{$\uS$} % second
\newcommand{\ueqSinv}{$\uSinv$} % second inverse
\newcommand{\ueqSinvsq}{$\uSinvsq$} % second inverse squared

% hour
\newcommand{\ueqH}{$\uH$} % hour
\newcommand{\ueqHinv}{$\uHinv$} % hour inverse

% kilogram
\newcommand{\ueqKG}{$\uKG$} % kilogram
\newcommand{\ueqKGinv}{$\uKGinv$} % kilogram inverse
\newcommand{\ueqKGinvsq}{$\uKGinvsq$} % kilogram inverse squared
\newcommand{\ueqG}{$\uG$} % gram

% lumen
\newcommand{\ueqLM}{$\uLM$} % lumen

% kelvin
\newcommand{\ueqK}{$\uK$} % kelvin
\newcommand{\ueqKsq}{$\uKsq$} % kelvin squared
\newcommand{\ueqKcu}{$\uKcv$} % kelvin cubed
\newcommand{\ueqKquar}{$\uKquar$} % kelvin quartic
\newcommand{\ueqKinv}{$\uKinv$} % kelvin inverse
\newcommand{\ueqKinvsq}{$\uKinvsq$} % kelvin inverse squared
\newcommand{\ueqKinvcu}{$\uKinvcu$} % kelvin inverse cubed
\newcommand{\ueqKinvquar}{$\uuKinvquar$} % kelvin inverse quartic

% degree celsius
\newcommand{\ueqCELS}{$\uCELS$} % degree celsius

% watt
\newcommand{\ueqW}{$\uW$} % watt
\newcommand{\ueqKW}{$\uKW$} % kilowatt
\newcommand{\ueqMW}{$\uMW$} % megawatt
\newcommand{\ueqGW}{$\uGW$} % gigawatt
\newcommand{\ueqWinv}{$\uWinv$} % watt inverse

% joule
\newcommand{\ueqJ}{$\uJ$} % joule
\newcommand{\ueqKJ}{$\uKJ$} % kilojoule
\newcommand{\ueqMJ}{$\uMJ$} % megajoule
\newcommand{\ueqCAL}{$\uCAL$} % calorie
\newcommand{\ueqKCAL}{$\uKCAL$} % kilocalorie
\newcommand{\ueqWH}{$\uWH$} % watt hour
\newcommand{\ueqKWH}{$\uKWH$} % kilowatt hour
\newcommand{\ueqWandS}{$\uWandS$} % watt second

% ampere
\newcommand{\ueqA}{$\uA$} % ampere

% volt
\newcommand{\ueqV}{$\uV$} % volt

% ohm
\newcommand{\ueqOHM}{$\uOHM$} % ohm
\newcommand{\ueqOHMinv}{$\uOHMinv$} % ohm inverse

% siemens
\newcommand{\ueqSIE}{$\uSIE$} % siemens
\newcommand{\ueqSIEinv}{$\uSIEinv$} % siemens inverse

% newton
\newcommand{\ueqN}{$\uN$} % newton

% currency
\newcommand{\ueqCCY}{$\uCCY$} % currency
\newcommand{\ueqCZK}{$\uCZK$} % czech crown

% --------------------
% Compound units
% --------------------

% velocity-acceleration
\newcommand{\ueqKGandMinvcu}{$\uKGandMinvcu$} % kilogram per meter cubed
\newcommand{\ueqMandSinv}{$\uMandSinv$} % meter per second
\newcommand{\ueqMcuSinv}{$\uMcuSinv$} % meter cubed per second
\newcommand{\ueqMandSinvsq}{$\uMandSinvsq$} % meter per second squared

% power-energy
\newcommand{\ueqWandMinvsq}{$\uWandMinvsq$} % watt per meter squared
\newcommand{\ueqWandMinvcu}{$\uWandMinvcu$} % watt per meter cubed
\newcommand{\ueqJandMinvsq}{$\uJandMinvsq$} % joule per meter squared
\newcommand{\ueqJandMinvcu}{$\uJandMinvcu$} % joule per meter cubed

% heat
\newcommand{\ueqJandKGinvKinv}{$\uJandKGinvKinv$} % joule per kilogram per kelvin
\newcommand{\ueqWandMinvKinv}{$\uWandMinvKinv$} % watt per meter per kelvin
\newcommand{\ueqMsqKandWinv}{$\uMsqKandWinv$} % meter squared kelvin per watt
\newcommand{\ueqKandWinv}{$\uKandWinv$} % kelvin per watt inverse
\newcommand{\ueqWandMinvsqKinv}{$\uWandMinvsqKinv$} % watt per meter squared per kelvin
\newcommand{\ueqWandKinv}{$\uWandKinv$} % watt per kelvin inverse

% electricity
\newcommand{\ueqOHMandMinv}{$\uOHMandMinv$} % ohm per meter
\newcommand{\ueqSIEandMinv}{$\uSIEandMinv$} % siemens per meter
\newcommand{\ueqAandMinvsq}{$\uAandMinvsq$} % ampere per meter squared
\newcommand{\ueqVandMinv}{$\uVandMinv$} % volt per meter


\head{Teorie}

\begin{document}

\maketitle
\tableofcontents
\newpage



\section{Značení}

\begin{itemize}
    \item $t$ - čas (\ueqS \fs -- sekunda)
    \item $l$ - délka (\ueqM \fs -- metr)
    \item $h$ - výška (\ueqM \fs -- metr)
    \item $r$ - poloměr (\ueqM \fs -- metr)
    \item $d$ - tloušťka / průměr (\ueqM \fs -- metr)
    \item $S$ - plocha (\ueqMsq \fs -- metr čtvereční)
    \item $V$ - objem (\ueqMcu \fs -- metr krychlový)
    \item $m$ - hmotnost (\ueqKG \fs -- kilogram)
    \item $\rho$ - hustota (\ueqKGandMinvcu \fs -- kilogram na metr krychlový)
    \item $v$ - rychlost (\ueqMandSinv \fs -- metr za sekundu)
    \item $a$ - zrychlení (\ueqMandSinvsq \fs -- metr za sekundu na druhou)
    \item $\dot{V}$ - objemový průtok (\ueqMcuSinv \fs -- metr krychlový za sekundu)
    \item $E_p$ - potenciální energie (\ueqJ \fs -- joule)
    \item $E_k$ - kinetická energie (\ueqJ \fs -- joule)
    \item $P$ - výkon (\ueqW \fs -- watt)
    \item $T$ - teplota (\ueqK \fs -- kelvin / \ueqCELS \fs - stupeň celsia)
    \item $\Delta T$ - rozdíl teplot (\ueqK \fs -- kelvin)
    \item $\dot{q}$ - měrný tepelný tok (\ueqWandMinvsq \fs -- watt na metr čtvereční)
    \item $\dot{Q}$ - tepelný tok (\ueqW \fs -- watt)
    \item $q$ - měrná tepelná energie (\ueqJandMinvsq \fs -- joule na metr čtvereční)
    \item $Q$ - tepelná energie (\ueqJ \fs -- joule)
    \item $Q_v$ - objemový zdroj tepla (\ueqWandMinvcu \fs -- watt na metr krychlový)
    \item $c$ - měrná tepelná kapacita (\ueqJandKGinvKinv \fs -- joule na kilogram na kelvin)
    \item $\lambda$ - tepelná vodivost (\ueqWandMinvKinv \fs -- watt na metr na kelvin)
    \item $R_{\vartheta}$ - (měrný) tepelný odpor (\ueqMsqKandWinv \fs -- metr čtvereční kelvin na watt)
    \item $R_{\vartheta A}$ - (absolutní) tepelný odpor (\ueqKandWinv \fs -- kelvin na watt)
    \item $\alpha_{\vartheta}$ - součinitel přestupu tepla (\ueqWandMinvsqKinv \fs -- watt na metr čtvereční na kelvin)
    \item $U_{\vartheta}$ - součinitel prostupu tepla (\ueqWandMinvsqKinv \fs -- watt na metr čtvereční na kelvin)
    \item $U_{\vartheta A}$ - prostupu tepla (\ueqWandKinv \fs -- watt na kelvin)
    \item $\rho_e$ - měrný elektrický odpor (\ueqOHMandMinv \fs -- ohm na metr)
    \item $\sigma_e$ - měrná elektrická vodivost (\ueqSIEandMinv \fs -- siemens na metr)
    \item $J_e$ - elektrická proudová hustota (\ueqAandMinvsq \fs -- ampér na metr čtvereční)
    \item $E_e$ - intenzita elektrického pole (\ueqVandMinv \fs -- volt na metr)
    \item $U_e$ - elektrické napětí (\ueqV \fs -- volt)
    \item $I_e$ - elektrický proud (\ueqA \fs -- ampér)
    \item $R_e$ - elektrický odpor (\ueqOHM \fs -- ohm)
\end{itemize}

\newpage



\section{Konstanty}

\begin{itemize}
    \item gravitační zrychlení: $g = 9,81$ \ueqMandSinvsq \fs -- metr za sekundu na druhou
    \item Stefanova-Boltzmannova konstanta: $\sigma = 5,67 \cdot 10^{-8}$ \ueqWandMinvsqKinvquar \fs -- watt na metr čtvereční na kelvin na čtvrtou
\end{itemize}

\begin{table}[H]
    \centering
    \begin{tabular}{l|ll}
        \hline
        Mateiál    & $\rho$ (\ueqKGandMinvcu) & $c$ (\ueqJandKGinvKinv) \\
        \hline
        Voda (H2O) & 1 000                    & 4 186                   \\
        Ocel       & 7 750                    & 450                     \\
        Zlato      & 19 320                   & 129                     \\
        \hline
    \end{tabular}
    \caption {Hustota a měrná tepelná kapacita materiálů}
\end{table}

\newpage



\section{Energie}


\subsection{Potenciální energie}
Potenciální energie je energie, kterou má těleso v důsledku své polohy v gravitačním poli. Vztah pro výpočet potenciální energie je:
\begin{equation}
    E_p = m \cdot g \cdot h,
    \tagaddtext{(\ueqJ)}
\end{equation}
kde:\\
$E_p$ -- potenciální energie (\ueqJ),\\
$m$ -- hmotnost (\ueqKG),\\
$g$ -- gravitační zrychlení (\ueqMandSinvsq),\\
$h$ -- výška (\ueqM).


\subsection{Kinetická energie}
Kinetická energie je energie, kterou má těleso v důsledku své rychlosti. Vztah pro výpočet kinetické energie je:
\begin{equation}
    E_k = \frac{1}{2} \cdot m \cdot v^2,
    \tagaddtext{(\ueqJ)}
\end{equation}
kde:\\
$E_k$ -- kinetická energie (\ueqJ),\\
$m$ -- hmotnost (\ueqKG),\\
$v$ -- rychlost (\ueqMandSinv).


\subsection{Měrná tepelná kapacita}
Měrná tepelná kapacita je definována jako množství tepla, které je potřeba k ohřátí jednoho kilogramu látky o jeden stupeň Kelvina:
\begin{equation}
    Q = m \cdot c \cdot \Delta T,
    \tagaddtext{(\ueqJ)}
\end{equation}
kde:\\
$Q$ -- tepelná energie (\ueqJ),\\
$m$ -- hmotnost (\ueqKG),\\
$c$ -- měrná tepelná kapacita (\ueqJandKGinvKinv),\\
$\Delta T$ -- rozdíl teplot (\ueqK).


\subsection{Výkon}
Výkon je definován jako množství práce vykonané za jednotku času:
\begin{equation}
    P = \frac{dW}{dt},
    \tagaddtext{(\ueqW)}
\end{equation}
kde:\\
$P$ -- výkon (\ueqW),\\
$dW$ -- infinitesimální práce (\ueqJ),\\
$dt$ -- infinitesimální čas (\ueqS).

\newpage



\section{Sdílení tepla}


\subsection{Fourierova-Kirchhoffova rovnice}
Fourierova-Kirchhoffova rovnice je základní rovnicí pro popis toku tepla:
\begin{equation}
    \rho \cdot c \cdot \left( \frac{\partial T}{\partial t} + \vec{v} \cdot \vec{\nabla} T \right) = \nabla \cdot \left( \lambda \cdot \vec{\nabla} T \right) + Q_v,
    \tagaddtext{(\ueqWandMinvcu)}
\end{equation}
kde:\\
$\rho$ -- hustota (\ueqKGandMinvcu),\\
$c$ -- měrná tepelná kapacita (\ueqJandKGinvKinv),\\
$T$ -- teplota (\ueqK),\\
$t$ -- čas (\ueqS),\\
$\vec{v}$ -- rychlost (\ueqMandSinv),\\
$\vec{\nabla} T$ -- gradient teploty (\ueqKandMinv),\\
$\nabla \cdot$ -- divergence (\ueqMinv),\\
$\lambda$ -- tepelná vodivost (\ueqWandMinvKinv),\\
$Q_v$ -- objemový zdroj tepla (\ueqWandMinvcu).


\subsection{Fourieruv zákon}
Fourieruv zákon je základní rovnicí pro popis toku tepla:
\begin{equation}
    \vec{\dot{q}} = - \lambda \cdot \vec{\nabla} T,
    \tagaddtext{(\ueqWandMinvsq)}
\end{equation}
kde:\\
$\vec{\dot{q}}$ -- měrný tepelný tok (\ueqWandMinvsq),\\
$\lambda$ -- tepelná vodivost (\ueqWandMinvKinv),\\
$\vec{\nabla} T$ -- gradient teploty (\ueqKandMinv).

\end{document}
