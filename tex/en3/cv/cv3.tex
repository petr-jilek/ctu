\documentclass{article}
\usepackage[czech]{babel} % Czech language
\usepackage[shortlabels]{enumitem} % Custom enumeration
\usepackage{graphicx} % Import images
\usepackage{float} % Use [H] to force figure position
\usepackage{indentfirst} % Indent first paragraph
\usepackage{emoji} % Emojis
\usepackage{pgffor} % Loops
\usepackage{tikz} % TikZ
\usepackage{pgfplots} % TikZ plots
\usepackage{circuitikz} % Use circuitikz for circuit diagrams
\usepackage{amsmath} % Math
\usetikzlibrary{arrows.meta}

\makeatletter
\providecommand\add@text{}
\newcommand\tagaddtext[1]{%
    \gdef\add@text{#1\gdef\add@text{}}}% 
\renewcommand\tagform@[1]{%
    \maketag@@@{\llap{\add@text\quad}(\ignorespaces#1\unskip\@@italiccorr)}%
}
\makeatother

\newcommand{\cvHead}[1]{\head{Cvičení #1}}

\newcommand{\head}[1]{
    \title{\textbf{#1}\\Elektroenergetika 3}
    \author{Petr Jílek}
    \date{2024}
}

\newcommand{\spicy}{\emoji{hot-pepper}}

% My font space
\newcommand{\myFS}{\;}

% Text in math mode
\newcommand{\te}[1]{\textrm{#1}}

% --------------------
% Units
% --------------------

\newcommand{\uM}{\textrm{m}} % Meter
\newcommand{\uMsq}{\uM^\textrm{2}} % Meter squared
\newcommand{\uMcu}{\uM^\textrm{3}} % Meter cubed
\newcommand{\uS}{\textrm{s}} % Second
\newcommand{\uKG}{\textrm{kg}} % Kilogram
\newcommand{\uJ}{\textrm{J}} % Joule
\newcommand{\uK}{\textrm{K}} % Kelvin
\newcommand{\uDEGREE}{^\circ} % Degree
\newcommand{\uCELS}{\uDEGREE \textrm{C}} % Celsius
\newcommand{\uW}{\textrm{W}} % Watt
\newcommand{\uMW}{\textrm{MW}} % Mega Watt
\newcommand{\uKW}{\textrm{kW}} % Kilo Watt
\newcommand{\uWH}{\textrm{Wh}} % Watt hour
\newcommand{\uKWH}{\textrm{kWh}} % Kilo Watt hour
\newcommand{\uCCY}{\textrm{CCY}} % Currency
\newcommand{\uCZK}{\textrm{CZK}} % Czech Crown
\newcommand{\uPERCENT}{\textrm{\%}} % Percent
\newcommand{\uNOUNIT}{\textrm{--}} % No unit
\newcommand{\uYEAR}{\textrm{year}} % Year
\newcommand{\uMONTH}{\textrm{month}} % Month
\newcommand{\uHOUR}{\textrm{hour}} % Hour
\newcommand{\uLM}{\textrm{lm}} % Lumen
\newcommand{\uHinv}{\textrm{h}^{-1}} % Hour inverse

\newcommand{\uJperK}{\uJ / \uK} % Joule per Kelvin
\newcommand{\uKperM}{\uK / \uM} % Kelvin per Meter
\newcommand{\uKperS}{\uK / \uS} % Kelvin per Second
\newcommand{\uMsqperS}{\uMsq/\uS} % Meter squared per Second
\newcommand{\uWperMsq}{\uW / \uMsq} % Watt per Meter squared
\newcommand{\uWperMperK}{\uW / \left( \uM \cdot \uK \right)} % Watt per Meter per Kelvin
\newcommand{\uJperKGperK}{\uJ / \left( \uKG \cdot \uK \right)} % Joule per Kilogram per Kelvin
\newcommand{\uKperMsq}{\uK / \uMsq} % Kelvin per Meter squared
\newcommand{\uJperSperMperK}{\uJ / \left( \uS \cdot \uM \cdot \uK \right)} % Joule per Second per Meter per Kelvin
\newcommand{\uKGperMcu}{\uKG / \uMcu} % Kilogram per Meter cubed
\newcommand{\uMsqKperW}{\uMsq \cdot \uK / \uW} % Meter squared Kelvin per Watt
\newcommand{\uWperMsqperK}{\uW / \left( \uMsq \cdot \uK \right)} % Watt per Meter squared per Kelvin
\newcommand{\uKWHperMsqperYEAR}{\uKWH / \left( \uMsq \cdot \uYEAR \right)} % Kilo Watt hour per Meter squared per Year
\newcommand{\uKWHperMsq}{\uKWH / \uMsq} % Kilo Watt hour per Meter squared
\newcommand{\uLMperW}{\uLM / \uW} % Lumen per Watt
\newcommand{\uKWperMsq}{\uKW / \uMsq} % Kilo Watt per Meter squared
\newcommand{\uCCYperYEAR}{\uCCY / \uYEAR} % Currency per Year
\newcommand{\uKWHperYEAR}{\uKWH / \uYEAR} % Kilo Watt hour per Year
\newcommand{\uNOUNITperYEAR}{\uNOUNIT / \uYEAR} % No unit per Year
\newcommand{\uCCYperKWH}{\uCCY / \uKWH} % Currency per Kilo Watt hour
\newcommand{\uCZKperYEAR}{\uCZK / \uYEAR} % Czech Crown per Year
\newcommand{\uCZKperKWH}{\uCZK / \uKWH} % Czech Crown per Kilo Watt hour
\newcommand{\uCZKperMONTH}{\uCZK / \uMONTH} % Czech Crown per Month



% --------------------
% Unit equations
% --------------------

\newcommand{\ueqM}{$\uM$}
\newcommand{\ueqMsq}{$\uMsq$}
\newcommand{\ueqMcu}{$\uMcu$}
\newcommand{\ueqS}{$\uS$}
\newcommand{\ueqJ}{$\uJ$}
\newcommand{\ueqK}{$\uK$}
\newcommand{\ueqDEGREE}{$\uDEGREE$}
\newcommand{\ueqCELS}{$\uCELS$}
\newcommand{\ueqW}{$\uW$}
\newcommand{\ueqMW}{$\uMW$}
\newcommand{\ueqKW}{$\uKW$}
\newcommand{\ueqWH}{$\uWH$}
\newcommand{\ueqKWH}{$\uKWH$}
\newcommand{\ueqCCY}{$\uCCY$}
\newcommand{\ueqCZK}{$\uCZK$}
\newcommand{\ueqPERCENT}{$\uPERCENT$}
\newcommand{\ueqNOUNIT}{$\uNOUNIT$}
\newcommand{\ueqYEAR}{$\uYEAR$}
\newcommand{\ueqMONTH}{$\uMONTH$}
\newcommand{\ueqHOUR}{$\uHOUR$}
\newcommand{\ueqLM}{$\uLM$}
\newcommand{\ueqHinv}{$\uHinv$}

\newcommand{\ueqJperK}{$\uJperK$}
\newcommand{\ueqKperM}{$\uKperM$}
\newcommand{\ueqKperS}{$\uKperS$}
\newcommand{\ueqMsqperS}{$\uMsqperS$}
\newcommand{\ueqWperMsq}{$\uWperMsq$}
\newcommand{\ueqWperMperK}{$\uWperMperK$}
\newcommand{\ueqJperKGperK}{$\uJperKGperK$}
\newcommand{\ueqKperMsq}{$\uKperMsq$}
\newcommand{\ueqJperSperMperK}{$\uJperSperMperK$}
\newcommand{\ueqKGperMcu}{$\uKGperMcu$}
\newcommand{\ueqMsqKperW}{$\uMsqKperW$}
\newcommand{\ueqWperMsqperK}{$\uWperMsqperK$}
\newcommand{\ueqKWHperMsqperYEAR}{$\uKWHperMsqperYEAR$}
\newcommand{\ueqKWHperMsq}{$\uKWHperMsq$}
\newcommand{\ueqLMperW}{$\uLMperW$}
\newcommand{\ueqKWperMsq}{$\uKWperMsq$}
\newcommand{\ueqCCYperYEAR}{$\uCCYperYEAR$}
\newcommand{\ueqKWHperYEAR}{$\uKWHperYEAR$}
\newcommand{\ueqNOUNITperYEAR}{$\uNOUNITperYEAR$}
\newcommand{\ueqCCYperKWH}{$\uCCYperKWH$}
\newcommand{\ueqCZKperYEAR}{$\uCZKperYEAR$}
\newcommand{\ueqCZKperKWH}{$\uCZKperKWH$}
\newcommand{\ueqCZKperMONTH}{$\uCZKperMONTH$}


\cvHead{3 - Sdílení tepla - Válec a koule}

\begin{document}

\maketitle
\tableofcontents

\section*{Úprava značení}

\begin{itemize}
    \item Absolutní tepelný odpor: $R_{\vartheta A} \rightarrow R \fs (\uKandWinv)$
    \item Součinitel prostupu tepla: $U \rightarrow U \fs (\uWandMinvsqKinv)$
\end{itemize}

\newpage



\section{Válec \spicy \spicy \spicy}


\subsection{Odvození tepelného odporu}

\begin{center}
    \begin{tikzpicture}
        % First circle
        \draw[thick] (0,0) circle [radius=3];
        \draw[thick] (0,0) circle [radius=1];

        % r1 and r2 labels
        \draw[->] (0, 0) -- (0, 1) node[midway, left] {$r_1$};
        \draw[->] (0, 0) -- (2.121,2.121) node[midway, above left] {$r_2$};

        % Circuit
        \draw[thick] (1,0) -- (1.5,0) node[midway, above] {$T_1$};
        \draw[thick] (1.5,-0.3) rectangle (2.5,0.3) node[midway] {$R_\lambda$};
        \draw[thick] (2.5,0) -- (3.5,0);
        \draw[thick] (3.5,-0.3) rectangle (4.5,0.3) node[midway] {$R_U$};
        \draw[thick] (4.5,0) -- (5,0);
        \draw[thick] (5,0) -- (5,-1);
        \draw[thick] (5.5,-1) -- (4.5,-1) node[midway, below] {$T_2$};

        % Triangle arrow
        \draw[-{Triangle}] (2.4,0.7) -- (4.4,.7) node[midway, above] {$Q$};
    \end{tikzpicture}
\end{center}

Tepelný odpor válce se skládá ze dvou částí:
\begin{itemize}
    \item tepelný odpor vedení tepla v materiálu válce $R_\lambda$,
    \item tepelný odpor přenosu tepla z povrchu válce do okolí $R_U$.
\end{itemize}

Tepelné schéma válce bude vypadat následovně:

\begin{center}
    \begin{tikzpicture}
        \draw[thick] (1,0) -- (1,-1);
        \draw[thick] (0.5,-1) -- (1.5,-1) node[midway, below] {$T_1$};
        \draw[thick] (1,0) -- (1.5,0);
        \draw[thick] (1.5,-0.3) rectangle (2.5,0.3) node[midway] {$R_\lambda$};
        \draw[thick] (2.5,0) -- (3.5,0);
        \draw[thick] (3.5,-0.3) rectangle (4.5,0.3) node[midway] {$R_U$};
        \draw[thick] (4.5,0) -- (5,0);
        \draw[thick] (5,0) -- (5,-1);
        \draw[thick] (5.5,-1) -- (4.5,-1) node[midway, below] {$T_2$};

        % Triangle arrow
        \draw[-{Triangle}] (2,0.7) -- (4,.7) node[midway, above] {$Q$};
    \end{tikzpicture}
\end{center}

Pro odvození tepelného odporu $R_\lambda$ využijeme vztah pro absolutní tepelný odpor $R$:
$$
    R = \frac{d}{\lambda \cdot S},
$$
kde:\\
$d$ - tloušťka materiálu (\uM),\\
$\lambda$ - tepelná vodivost materiálu (\ueqWandMinvKinv),\\
$S$ - plocha přenosu tepla (\ueqMsq).\\

Zde $d$ nahradíme nekonečně malou částí poloměru válce:
$$
    d \rightarrow dr.
$$

Dále $S$ nahradíme plochou válce:
$$
    S \rightarrow 2 \pi r \cdot l \cdot dr,
$$
kde:\\
$r$ - poloměr válce (m),\\
$l$ - délka válce (m).\\

Tepelný odpor $dR_\lambda$ válce bude tedy:
$$
    dR_\lambda = \frac{dr}{\lambda \cdot 2 \pi r \cdot l \cdot dr}.
$$

\begin{center}
    \begin{tikzpicture}
        % First circle
        \draw[thick] (0,0) circle [radius=3];
        \draw[thick] (0,0) circle [radius=1];
        \draw[thick] (0, 0) circle [radius=2.2];
        \draw[thick] (0, 0) circle [radius=2.4];

        % Second circle
        \draw[thick] (6,1) circle [radius=2.4];

        % Lines
        \draw[thick] (0,3) -- (6,3.4);
        \draw[thick] (0.6,-2.94) -- (6.6,-1.33) node[midway, below right] {$l$};

        % r1 and r2 labels
        \draw[->] (0, 0) -- (0, 1) node[midway, right] {$r_1$};
        \draw[->] (0, 0) -- (-3, 0) node[midway, below right] {$r_2$};

        % dr label
        \draw[->] (0,2.7) -- (0,2.4);
        \draw (0,2.4) -- (0,2.2);
        \draw[<-] (0,2.2) -- (0,1.9) node[midway, below right] {$dr$};
    \end{tikzpicture}
\end{center}

Pro získání celkového odporu $R_\lambda$ je třeba provést integraci od vnitřního poloměru izolace $r_1$ po vnější poloměr válce $r_2$:
$$
    R_\lambda = \int_{r_1}^{r_2} \frac{1}{\lambda \cdot 2 \pi r \cdot l} dr = \frac{1}{\lambda \cdot 2 \pi \cdot l} \int_{r_1}^{r_2} \frac{1}{r} dr = \frac{1}{\lambda \cdot 2 \pi \cdot l} \left[ \ln |r| \right]_{r_1}^{r_2}
$$

Jelikož poloměry $r_1$ a $r_2$ jsou kladné (záporné poloměry nedávají fyzikální smysl), můžeme absolutní hodnotu u logaritmu zanedbat:
$$
    R_\lambda = \frac{1}{\lambda \cdot 2 \pi \cdot l} \left[ \ln r \right]_{r_1}^{r_2} = \frac{1}{\lambda \cdot 2 \pi \cdot l} \left( \ln r_2 - \ln r_1 \right) = \frac{1}{\lambda \cdot 2 \pi \cdot l} \ln \frac{r_2}{r_1} = \frac{1}{2 \pi l \lambda} \ln \frac{r_2}{r_1}.
$$

Pro odpor přenosu tepla z povrchu válce do okolí $R_U$ využijeme vztah pro odpor přenosu tepla $R_U$:
$$
    R_U = \frac{1}{U \cdot S},
$$
kde:\\
$U$ - součinitel prostupu tepla (\ueqWandMinvsqKinv).\\

Plochu $S$ nahradíme plochou válce:
$$
    S \rightarrow 2 \pi r_2 \cdot l.
$$

Odpor přenosu tepla z povrchu válce do okolí $R_U$ bude tedy:
$$
    R_U = \frac{1}{U \cdot 2 \pi r_2 \cdot l} = \frac{1}{2 \pi l U r_2}.
$$

Celkový tepelný odpor válce $R_{\vartheta, \Sigma}$ bude součtem obou odporů:
$$
    R_\Sigma = R_\lambda + R_U = \frac{1}{2 \pi l \lambda} \ln \frac{r_2}{r_1} + \frac{1}{2 \pi l U r_2}
$$


\subsection{Minimum tepelného odporu}

Pro nalezení extrému tepelného odporu $R_\Sigma$ podle poloměru válce $r_2$ je třeba zjistit, kdy bude derivace $R_\Sigma$ podle $r_2$ rovna nule:
$$
    \frac{dR_\Sigma}{dr_2} = 0.
$$

Derivace $R_\Sigma$ podle $r_2$ bude:
$$
    \frac{dR_\Sigma}{dr_2} = \frac{dR_\lambda}{dr_2} + \frac{dR_U}{dr_2}.
$$

Derivace $R_\lambda$ podle $r_2$ bude:
$$
    \frac{dR_\lambda}{dr_2} = \frac{d}{dr_2} \left( \frac{1}{2 \pi l \lambda} \ln \frac{r_2}{r_1} \right) = \frac{d}{dr_2} \left( \frac{1}{2 \pi l \lambda} \left(\ln r_2 - \ln r_1 \right) \right) =
$$

$$
    = \frac{d}{dr_2} \left( \frac{1}{2 \pi l \lambda} \ln r_2 \right) = \frac{1}{2 \pi l \lambda r_2}.
$$

Derivace $R_U$ podle $r_2$ bude:
$$
    \frac{dR_U}{dr_2} = \frac{d}{dr_2} \left( \frac{1}{2 \pi l U r_2} \right) = -\frac{1}{2 \pi l U r_2^2}.
$$

Derivace $R_\Sigma$ podle $r_2$ bude tedy:
$$
    \frac{dR_\Sigma}{dr_2} = \frac{1}{2 \pi l \lambda r_2} - \frac{1}{2 \pi l U r_2^2}.
$$

Nyní můžeme zjistit, kdy bude derivace $R_\Sigma$ podle $r_2$ rovna nule:
$$
    \frac{1}{2 \pi l \lambda r_2} - \frac{1}{2 \pi l U r_2^2} = 0
$$

$$
    \frac{1}{\lambda r_2} - \frac{1}{U r_2^2} = 0
$$

$$
    \frac{1}{\lambda r_2} = \frac{1}{U r_2^2}
$$

$$
    \frac{r_2}{\lambda} = \frac{1}{U}
$$

$$
    r_2 = \frac{\lambda}{U}.
$$

V další části budeme zkoumat, zda se jedná o minimum nebo maximum. K tomu využijeme druhou derivaci $R_\Sigma$ podle $r_2$:
$$
    \frac{d^2R_\Sigma}{dr_2^2} = \frac{d}{dr_2} \left( \frac{1}{2 \pi l \lambda r_2} - \frac{1}{2 \pi l U r_2^2} \right) = -\frac{1}{2 \pi l \lambda r_2^2} + \frac{1}{\pi l U r_2^3}.
$$

Nyní dosadíme $r_2 = \frac{\lambda}{U}$:
$$
    -\frac{1}{2 \pi l \lambda \left( \frac{\lambda}{U} \right)^2} + \frac{1}{\pi l U \left( \frac{\lambda}{U} \right)^3} = -\frac{1}{2 \pi l \lambda \frac{\lambda^2}{U^2}} + \frac{1}{\pi l U \frac{\lambda^3}{U^3}} =
$$

$$
    = -\frac{U^2}{2 \pi l \lambda^3} + \frac{U^2}{\pi l \lambda^3} = \frac{-U^2 + 2 U^2}{2 \pi l \lambda^3} = \frac{U^2}{2 \pi l \lambda^3}.
$$

Hodnoty $U$ a $\lambda$ jsou kladné, tudíž druhá derivace je kladná. To znamená, že se jedná o minimum.

\begin{center}
    \begin{tikzpicture}
        \begin{axis}[
                title = {Závislost tepelného odporu $R_\Sigma$ na poloměru válce $r_2$},
                axis lines = middle,
                xlabel = $r_2 \fs (\uM)$,
                ylabel = {$R_\Sigma \fs (\uMsqKandWinv)$},
                samples=100,
                domain=0:0.5,
                xmin=0,
                xmax=0.4,
                ymin=0,
                ymax=50,
                legend pos=outer north east,
                grid=both
            ]
            \addplot[color=blue, thick] {(1/(2*pi*0.04))*ln(x/0.05) + 1/(2*pi*0.5*x)};
        \end{axis}
    \end{tikzpicture}
\end{center}


\subsection{Ekonomie}

Nyní můžeme porovnat rozdíl v rychlosti růstu tepelného odporu $R_\Sigma$ a objemu izolace v závislosti na poloměru válce $r_2$. Řešíme objem izolace, jelikož při konstrukci se platí za objem materiálu. Pro objem izolace platí:
$$
    V = \pi (r_2^2 - r_1^2) \cdot l = \pi r_2^2 \cdot l - \pi r_1^2 \cdot l.
$$

Nyní provedeme zjednodušení vzorce podobně jako se provádí u časové složitosti algoritmů. Zkoumáme rychlost růstu objemu izolace v závislosti na poloměru válce $r_2$, tudíž vyřadíme konstantní členy, čímž dostaneme:
$$
    V \sim \pi r_2^2 \cdot l.
$$

Dále vyřadíme všechny násobící konstanty, čímž dostaneme:
$$
    V \sim r_2^2.
$$

Zde dostáváme, že objem roste s druhou mocninou poloměru válce $r_2$. Pro odpor $R_\Sigma$ platí:
$$
    R_\Sigma = \frac{1}{2 \pi l \lambda} \ln \frac{r_2}{r_1} + \frac{1}{2 \pi l U r_2}.
$$

Zde druhý člen jde do nuly, když $r_2$ jde do nekonečna, tudíž ho můžeme vyřadit. Zbyde nám:
$$
    R_\Sigma \sim \frac{1}{2 \pi l \lambda} \ln \frac{r_2}{r_1}.
$$

Dále můžeme vyřadit konstantní členy a dostaneme:
$$
    R_\Sigma \sim \ln r_2.
$$

Zde vidíme, že odpor roste logaritmicky s poloměrem válce $r_2$. Pokud porovnáme rychlost růstu objemu izolace a rychlost růstu odporu $R_\Sigma$, zjistíme, že objem izolace roste rychleji než odpor $R_\Sigma$. To znamená, že při dimenzování izolace je třeba brát v potaz i ekonomické hledisko, jelikož při velkém přidání izolace nám dramaticky může narůst cena, ale odpor $R_\Sigma$ se nám příliš nezmění.

\begin{center}
    \begin{tikzpicture}
        \begin{axis}[
                axis lines = middle,
                xlabel = $x$,
                ylabel = {$y$},
                samples=100,
                domain=0.1:3, % Adjust the domain for better view
                legend pos=outer north east,
                grid=both
            ]
            % Plot ln(x)
            \addplot[color=blue, thick] {ln(x)};
            \addlegendentry{$\ln(x)$}

            % Plot x^2
            \addplot[color=red, thick, dashed] {x^2};
            \addlegendentry{$x^2$}
        \end{axis}
    \end{tikzpicture}
\end{center}


\subsection{Číselný příklad 1 \spicy \spicy}
Mějme izolovaný vodič v rozvaděči, kde tepelná vodivost izolace je $\lambda = 0.159 \fs \uWandMinvKinv$, součinitel prostupu tepla do okolí je $U = 10 \fs \uWandMinvsqKinv$ Vypočítejte vnější poloměr válce $r_2$, kdy bude tepelný odpor $R_\Sigma$ minimální.

\subsubsection{Řešení}
Dosadíme do vzorce pro minimum tepelného odporu $R_\Sigma$:
$$
    r_2 = \frac{\lambda}{U} = \frac{0.159}{10} = 0.0159 \fs \uM = 1,59 \fs \te{cm}.
$$

Z výsledku můžeme říct, že pokud máme vodič o poloměru $r_1$, který je menší, než $1,59 \fs \te{cm}$, pak pokud přidáme izolaci tak, aby vnější poloměr byl $1,59 \fs \te{cm}$, tak bude izolovaný vodič lépe odvádět teplo do okolí.


\subsection{Číselný příklad 2 \spicy \spicy}
Izolace horkovodního potrubí má tepelnou vodivost $\lambda = 0.02 \fs \uWandMinvKinv$ a součinitel prostupu tepla do okolí je $U = 5 \fs \uWandMinvsqKinv$. Vypočítejte vnější poloměr válce $r_2$, kdy bude tepelný odpor $R_\Sigma$ minimální.

\subsubsection{Řešení}
Dosadíme do vzorce pro minimum tepelného odporu $R_\Sigma$:
$$
    r_2 = \frac{\lambda}{U} = \frac{0.02}{5} = 0.004 \fs \uM = 0.4 \fs \te{cm}.
$$

Z výsledku můžeme říct, že by vnitřní poloměr potrubí $r_1$ měl být menší než $0.4 \fs \te{cm}$, aby bylo výhodné přidat izolaci. Nicméně takto malý poloměr potrubí se v praxi nevyskytuje.

\newpage



\section{Koule \spicy \spicy \spicy}


\subsection{Odvození tepelného odporu}

\begin{center}
    \begin{tikzpicture}
        % Draw the sphere
        \shade[ball color=gray!30] (0,0) circle (3);
        \shade[ball color=orange!50] (0,0) circle (1);

        % Draw the radii
        \draw[thick,->] (0,0) -- (0,1) node[midway, left]{$r_1$};
        \draw[thick,->] (0,0) -- (2.121,2.121) node[midway, above left]{$r_2$};

        % Draw center point
        \filldraw (0,0) circle (2pt);

        % Circuit
        \draw[thick] (1,0) -- (1.5,0) node[midway, above] {$T_1$};
        \draw[thick] (1.5,-0.3) rectangle (2.5,0.3) node[midway] {$R_\lambda$};
        \draw[thick] (2.5,0) -- (3.5,0);
        \draw[thick] (3.5,-0.3) rectangle (4.5,0.3) node[midway] {$R_U$};
        \draw[thick] (4.5,0) -- (5,0);
        \draw[thick] (5,0) -- (5,-1);
        \draw[thick] (5.5,-1) -- (4.5,-1) node[midway, below] {$T_2$};

        % Triangle arrow
        \draw[-{Triangle}] (2.4,0.7) -- (4.4,.7) node[midway, above] {$Q$};
    \end{tikzpicture}
\end{center}

Tepelný odpor koule se skládá ze dvou částí:
\begin{itemize}
    \item odpor vedení tepla v materiálu koule $R_\lambda$,
    \item odpor přenosu tepla z povrchu koule do okolí $R_U$.
\end{itemize}

Tepelné schéma koule bude vypadat následovně:

\begin{center}
    \begin{tikzpicture}
        \draw[thick] (1,0) -- (1,-1);
        \draw[thick] (0.5,-1) -- (1.5,-1) node[midway, below] {$T_1$};
        \draw[thick] (1,0) -- (1.5,0);
        \draw[thick] (1.5,-0.3) rectangle (2.5,0.3) node[midway] {$R_\lambda$};
        \draw[thick] (2.5,0) -- (3.5,0);
        \draw[thick] (3.5,-0.3) rectangle (4.5,0.3) node[midway] {$R_U$};
        \draw[thick] (4.5,0) -- (5,0);
        \draw[thick] (5,0) -- (5,-1);
        \draw[thick] (5.5,-1) -- (4.5,-1) node[midway, below] {$T_2$};

        % Triangle arrow
        \draw[-{Triangle}] (2.4,0.7) -- (4.4,.7) node[midway, above] {$Q$};
    \end{tikzpicture}
\end{center}


Pro odvození tepelného odporu $R_{\lambda}$ využijeme vztah pro tepelný odpor $R$:
$$
    R_{\vartheta} = \frac{d}{\lambda \cdot S},
$$
kde:\\
$d$ - tloušťka materiálu (\ueqM),\\
$\lambda$ - tepelná vodivost materiálu (\ueqWandMinvKinv),\\
$S$ - plocha přenosu tepla (\ueqMsq).\\

Zde $d$ nahradíme nekonečně malou částí poloměru koule:
$$
    d \rightarrow dr.
$$

Dále $S$ nahradíme plochou koule:
$$
    S \rightarrow 4 \pi r^2.
$$

Tepelný odpor $dR_{\vartheta}$ koule bude tedy:
$$
    dR_\lambda = \frac{dr}{\lambda \cdot 4 \pi r^2}.
$$

Pro získání celkového odporu $R_{\lambda}$ je třeba provést integraci od vnitřního poloměru izolace $r_1$ po vnější poloměr koule $r_2$:
$$
    R_\lambda = \int_{r_1}^{r_2} \frac{1}{\lambda \cdot 4 \pi r^2} dr = \frac{1}{\lambda \cdot 4 \pi} \int_{r_1}^{r_2} \frac{1}{r^2} dr = \frac{1}{\lambda \cdot 4 \pi} \left[ -\frac{1}{r} \right]_{r_1}^{r_2}
$$

Po dosazení mezí integrace dostaneme:
$$
    R_\lambda = \frac{1}{\lambda \cdot 4 \pi} \left( -\frac{1}{r_2} + \frac{1}{r_1} \right) = \frac{1}{\lambda \cdot 4 \pi} \left( \frac{1}{r_1} - \frac{1}{r_2} \right) = \frac{1}{4 \pi \lambda} \left( \frac{1}{r_1} - \frac{1}{r_2} \right)
$$

Pro odpor přenosu tepla z povrchu koule do okolí $R_{U}$ využijeme vztah pro odpor přenosu tepla $R_{U}$:
$$
    R_U = \frac{1}{U \cdot S},
$$
kde:\\
$U$ - koeficient prostupu tepla (W/m$^2 \cdot$K).\\

Plochu $S$ nahradíme plochou koule:
$$
    S \rightarrow 4 \pi r_2^2.
$$

Odpor přenosu tepla z povrchu koule do okolí $R_U$ bude tedy:
$$
    R_U = \frac{1}{U \cdot 4 \pi r_2^2} = \frac{1}{4 \pi U r_2^2}.
$$

Celkový tepelný odpor koule $R_\Sigma$ bude součtem obou odporů:
$$
    R_\Sigma = R_{\lambda} + R_U = \frac{1}{4 \pi \lambda} \left( \frac{1}{r_1} - \frac{1}{r_2} \right) + \frac{1}{4 \pi U r_2^2}
$$

Nyní pojďme vyšetřit limitu odporu pro $r_2$ jdoucí do nekonečna:
$$
    \lim_{r_2 \to \infty} R_\Sigma = \lim_{r_2 \to \infty} \left( \frac{1}{4 \pi \lambda} \left( \frac{1}{r_1} - \frac{1}{r_2} \right) + \frac{1}{4 \pi U r_2^2} \right)
$$

Členy, kde se vyskytuje $r_2$ ve jmenovateli, jdou do nuly, čímž dostaneme:
$$
    \lim_{r_2 \to \infty} R_\Sigma = \frac{1}{4 \pi \lambda r_1}.
$$

Zde vidíme, že koule nelze úplně izolovat, jelikož při nekonečné izolaci bude mít koule stále nějaký odpor.


\subsection{Minimum tepelného odporu}

Pro nalezení extrému tepelného odporu $R_\Sigma$ podle poloměru koule $r_2$ je třeba zjistit, kdy bude derivace $R_\Sigma$ podle $r_2$ rovna nule:
$$
    \frac{dR_\Sigma}{dr_2} = 0.
$$

Derivace $R_\Sigma$ podle $r_2$ bude:
$$
    \frac{dR_\Sigma}{dr_2} = \frac{dR_\lambda}{dr_2} + \frac{dR_U}{dr_2}.
$$

Derivace $R_\lambda$ podle $r_2$ bude:
$$
    \frac{dR_\lambda}{dr_2} = \frac{d}{dr_2} \left( \frac{1}{4 \pi \lambda} \left( \frac{1}{r_1} - \frac{1}{r_2} \right) \right) = \frac{1}{4 \pi \lambda r_2^2}
$$

Derivace $R_U$ podle $r_2$ bude:
$$
    \frac{dR_U}{dr_2} = \frac{d}{dr_2} \left( \frac{1}{4 \pi U r_2^2} \right) = -\frac{2}{4 \pi U r_2^3} = -\frac{1}{2 \pi U r_2^3}.
$$

Derivace $R_\Sigma$ podle $r_2$ bude tedy:
$$
    \frac{dR_\Sigma}{dr_2} = \frac{1}{4 \pi \lambda r_2^2} - \frac{1}{2 \pi U r_2^3}.
$$

Nyní můžeme zjistit, kdy bude derivace $R_\Sigma$ podle $r_2$ rovna nule:
$$
    \frac{1}{4 \pi \lambda r_2^2} - \frac{1}{2 \pi U r_2^3} = 0
$$

$$
    \frac{1}{4 \lambda r_2^2} - \frac{1}{2 U r_2^3} = 0
$$

$$
    \frac{1}{4 \lambda r_2^2} = \frac{1}{2 U r_2^3}
$$

$$
    \frac{1}{\lambda r_2^2} = \frac{2}{U r_2^3}
$$

$$
    \frac{r_2}{\lambda} = \frac{2}{U}
$$

$$
    r_2 = \frac{2 \lambda}{U}.
$$

V další části budeme zkoumat, zda se jedná o minimum nebo maximum. K tomu využijeme druhou derivaci $R_\Sigma$ podle $r_2$:

\begin{center}
    \begin{tikzpicture}
        \begin{axis}[
                title = {Závislost tepelného odporu $R_\Sigma$ na poloměru koule $r_2$},
                axis lines = middle,
                xlabel = $r_2 \fs (\uM)$,
                ylabel = {$R_\Sigma \fs (\uMsqKandWinv)$},
                samples=100,
                domain=0:0.5,
                xmin=0,
                xmax=0.5,
                ymin=0,
                ymax=300,
                legend pos=outer north east,
                grid=both
            ]
            \addplot[color=blue, thick] {(1/(4*pi*0.04))*(1/0.05-1/x) + 1/(4*pi*2*x^2)};
        \end{axis}
    \end{tikzpicture}
\end{center}


\subsection{Ekonomie}

Nyní můžeme porovnat rozdíl v rychlosti růstu tepelného odporu $R_\Sigma$ a objemu izolace v závislosti na poloměru koule $r_2$.
$$
    V = \frac{4}{3} \pi (r_2^3 - r_1^3).
$$

Po zjednodušení dostaneme:
$$
    V \sim r_2^3.
$$

Rychlost růstu objemu izolace je kubická a jde do někonečna, zatímco tepelný odpor dosáhne limitní hodnoty.


\subsection{Číselný příklad}
Mějme izolovanou kouli, kde tepelná vodivost izolace je $\lambda = 0.159 \fs \uWandMinvKinv$, součinitel prostupu tepla do okolí je $U = 10 \fs \uWandMinvsqKinv$ Vypočítejte vnější poloměr koule $r_2$, kdy bude tepelný odpor $R_\Sigma$ minimální.

\subsubsection{Řešení}
Dosadíme do vzorce pro minimum tepelného odporu $R_\Sigma$:
$$
    r_2 = \frac{2 \lambda}{U} = \frac{2 \cdot 0.159}{10} = 0.0318 \fs \uM = 3.18 \fs \te{cm}.
$$

\end{document}
