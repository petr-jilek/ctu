\documentclass{article}
\usepackage[czech]{babel} % Czech language
\usepackage[shortlabels]{enumitem} % Custom enumeration
\usepackage{graphicx} % Import images
\usepackage{float} % Use [H] to force figure position
\usepackage{indentfirst} % Indent first paragraph
\usepackage{emoji} % Emojis
\usepackage{pgffor} % Loops
\usepackage{tikz} % TikZ
\usepackage{pgfplots} % TikZ plots
\usepackage{circuitikz} % Use circuitikz for circuit diagrams
\usepackage{amsmath} % Math
\usetikzlibrary{arrows.meta}

\makeatletter
\providecommand\add@text{}
\newcommand\tagaddtext[1]{%
    \gdef\add@text{#1\gdef\add@text{}}}% 
\renewcommand\tagform@[1]{%
    \maketag@@@{\llap{\add@text\quad}(\ignorespaces#1\unskip\@@italiccorr)}%
}
\makeatother

\newcommand{\cvHead}[1]{\head{Cvičení #1}}

\newcommand{\head}[1]{
    \title{\textbf{#1}\\Elektroenergetika 3}
    \author{Petr Jílek}
    \date{2024}
}

\newcommand{\spicy}{\emoji{hot-pepper}}

% My font space
\newcommand{\myFS}{\;}

% Text in math mode
\newcommand{\te}[1]{\textrm{#1}}


\cvHead{3 - Sdílení tepla - Válec a koule}

\begin{document}

\newcommand{\cvHead}[1]{\head{Cvičení #1}}

\newcommand{\head}[1]{
    \title{\textbf{#1}\\Elektroenergetika 3}
    \author{Petr Jílek}
    \date{2024}
}

\newcommand{\spicy}{\emoji{hot-pepper}}

% My font space
\newcommand{\myFS}{\;}

% Text in math mode
\newcommand{\te}[1]{\textrm{#1}}

% --------------------
% Units
% --------------------

\newcommand{\uM}{\textrm{m}} % Meter
\newcommand{\uMsq}{\uM^\textrm{2}} % Meter squared
\newcommand{\uMcu}{\uM^\textrm{3}} % Meter cubed
\newcommand{\uS}{\textrm{s}} % Second
\newcommand{\uKG}{\textrm{kg}} % Kilogram
\newcommand{\uJ}{\textrm{J}} % Joule
\newcommand{\uK}{\textrm{K}} % Kelvin
\newcommand{\uDEGREE}{^\circ} % Degree
\newcommand{\uCELS}{\uDEGREE \textrm{C}} % Celsius
\newcommand{\uW}{\textrm{W}} % Watt
\newcommand{\uMW}{\textrm{MW}} % Mega Watt
\newcommand{\uKW}{\textrm{kW}} % Kilo Watt
\newcommand{\uWH}{\textrm{Wh}} % Watt hour
\newcommand{\uKWH}{\textrm{kWh}} % Kilo Watt hour
\newcommand{\uCCY}{\textrm{CCY}} % Currency
\newcommand{\uCZK}{\textrm{CZK}} % Czech Crown
\newcommand{\uPERCENT}{\textrm{\%}} % Percent
\newcommand{\uNOUNIT}{\textrm{--}} % No unit
\newcommand{\uYEAR}{\textrm{year}} % Year
\newcommand{\uMONTH}{\textrm{month}} % Month
\newcommand{\uHOUR}{\textrm{hour}} % Hour
\newcommand{\uLM}{\textrm{lm}} % Lumen
\newcommand{\uHinv}{\textrm{h}^{-1}} % Hour inverse

\newcommand{\uJperK}{\uJ / \uK} % Joule per Kelvin
\newcommand{\uKperM}{\uK / \uM} % Kelvin per Meter
\newcommand{\uKperS}{\uK / \uS} % Kelvin per Second
\newcommand{\uMsqperS}{\uMsq/\uS} % Meter squared per Second
\newcommand{\uWperMsq}{\uW / \uMsq} % Watt per Meter squared
\newcommand{\uWperMperK}{\uW / \left( \uM \cdot \uK \right)} % Watt per Meter per Kelvin
\newcommand{\uJperKGperK}{\uJ / \left( \uKG \cdot \uK \right)} % Joule per Kilogram per Kelvin
\newcommand{\uKperMsq}{\uK / \uMsq} % Kelvin per Meter squared
\newcommand{\uJperSperMperK}{\uJ / \left( \uS \cdot \uM \cdot \uK \right)} % Joule per Second per Meter per Kelvin
\newcommand{\uKGperMcu}{\uKG / \uMcu} % Kilogram per Meter cubed
\newcommand{\uMsqKperW}{\uMsq \cdot \uK / \uW} % Meter squared Kelvin per Watt
\newcommand{\uWperMsqperK}{\uW / \left( \uMsq \cdot \uK \right)} % Watt per Meter squared per Kelvin
\newcommand{\uKWHperMsqperYEAR}{\uKWH / \left( \uMsq \cdot \uYEAR \right)} % Kilo Watt hour per Meter squared per Year
\newcommand{\uKWHperMsq}{\uKWH / \uMsq} % Kilo Watt hour per Meter squared
\newcommand{\uLMperW}{\uLM / \uW} % Lumen per Watt
\newcommand{\uKWperMsq}{\uKW / \uMsq} % Kilo Watt per Meter squared
\newcommand{\uCCYperYEAR}{\uCCY / \uYEAR} % Currency per Year
\newcommand{\uKWHperYEAR}{\uKWH / \uYEAR} % Kilo Watt hour per Year
\newcommand{\uNOUNITperYEAR}{\uNOUNIT / \uYEAR} % No unit per Year
\newcommand{\uCCYperKWH}{\uCCY / \uKWH} % Currency per Kilo Watt hour
\newcommand{\uCZKperYEAR}{\uCZK / \uYEAR} % Czech Crown per Year
\newcommand{\uCZKperKWH}{\uCZK / \uKWH} % Czech Crown per Kilo Watt hour
\newcommand{\uCZKperMONTH}{\uCZK / \uMONTH} % Czech Crown per Month



% --------------------
% Unit equations
% --------------------

\newcommand{\ueqM}{$\uM$}
\newcommand{\ueqMsq}{$\uMsq$}
\newcommand{\ueqMcu}{$\uMcu$}
\newcommand{\ueqS}{$\uS$}
\newcommand{\ueqJ}{$\uJ$}
\newcommand{\ueqK}{$\uK$}
\newcommand{\ueqDEGREE}{$\uDEGREE$}
\newcommand{\ueqCELS}{$\uCELS$}
\newcommand{\ueqW}{$\uW$}
\newcommand{\ueqMW}{$\uMW$}
\newcommand{\ueqKW}{$\uKW$}
\newcommand{\ueqWH}{$\uWH$}
\newcommand{\ueqKWH}{$\uKWH$}
\newcommand{\ueqCCY}{$\uCCY$}
\newcommand{\ueqCZK}{$\uCZK$}
\newcommand{\ueqPERCENT}{$\uPERCENT$}
\newcommand{\ueqNOUNIT}{$\uNOUNIT$}
\newcommand{\ueqYEAR}{$\uYEAR$}
\newcommand{\ueqMONTH}{$\uMONTH$}
\newcommand{\ueqHOUR}{$\uHOUR$}
\newcommand{\ueqLM}{$\uLM$}
\newcommand{\ueqHinv}{$\uHinv$}

\newcommand{\ueqJperK}{$\uJperK$}
\newcommand{\ueqKperM}{$\uKperM$}
\newcommand{\ueqKperS}{$\uKperS$}
\newcommand{\ueqMsqperS}{$\uMsqperS$}
\newcommand{\ueqWperMsq}{$\uWperMsq$}
\newcommand{\ueqWperMperK}{$\uWperMperK$}
\newcommand{\ueqJperKGperK}{$\uJperKGperK$}
\newcommand{\ueqKperMsq}{$\uKperMsq$}
\newcommand{\ueqJperSperMperK}{$\uJperSperMperK$}
\newcommand{\ueqKGperMcu}{$\uKGperMcu$}
\newcommand{\ueqMsqKperW}{$\uMsqKperW$}
\newcommand{\ueqWperMsqperK}{$\uWperMsqperK$}
\newcommand{\ueqKWHperMsqperYEAR}{$\uKWHperMsqperYEAR$}
\newcommand{\ueqKWHperMsq}{$\uKWHperMsq$}
\newcommand{\ueqLMperW}{$\uLMperW$}
\newcommand{\ueqKWperMsq}{$\uKWperMsq$}
\newcommand{\ueqCCYperYEAR}{$\uCCYperYEAR$}
\newcommand{\ueqKWHperYEAR}{$\uKWHperYEAR$}
\newcommand{\ueqNOUNITperYEAR}{$\uNOUNITperYEAR$}
\newcommand{\ueqCCYperKWH}{$\uCCYperKWH$}
\newcommand{\ueqCZKperYEAR}{$\uCZKperYEAR$}
\newcommand{\ueqCZKperKWH}{$\uCZKperKWH$}
\newcommand{\ueqCZKperMONTH}{$\uCZKperMONTH$}


\maketitle
\tableofcontents
\newpage



\section{Válec \spicy \spicy \spicy}


\subsection{Odvození tepelného odporu}

Tepelný odpor válce se skládá ze dvou částí:
\begin{itemize}
    \item odpor vedení tepla v materiálu válce $R_{\lambda}$,
    \item odpor přenosu tepla z povrchu válce do okolí $R_{\alpha}$.
\end{itemize}

Pro odvození tepelného odporu $R_{\lambda}$ využijeme vztah pro tepelný odpor $R$:
$$
    R = \frac{d}{\lambda \cdot S},
$$
kde:\\
$d$ - tloušťka materiálu (m),\\
$\lambda$ - tepelná vodivost materiálu (W/m$\cdot$K),\\
$S$ - plocha přenosu tepla (m$^2$).\\

Zde $d$ nahradíme nekonečně malou částí poloměru válce:
$$
    d \rightarrow dr.
$$

Dále $S$ nahradíme plochou válce:
$$
    S \rightarrow 2 \pi r \cdot l \cdot dr,
$$
kde:\\
$r$ - poloměr válce (m),\\
$l$ - délka válce (m).\\

Tepelný odpor $dR$ válce bude tedy:
$$
    dR = \frac{dr}{\lambda \cdot 2 \pi r \cdot l \cdot dr}.
$$

Pro získání celkového odporu $R_{\lambda}$ je třeba provést integraci od vnitřního poloměru izolace $r_1$ po vnější poloměr válce $r_2$:
$$
    R_{\lambda} = \int_{r_1}^{r_2} \frac{1}{\lambda \cdot 2 \pi r \cdot l} dr = \frac{1}{\lambda \cdot 2 \pi \cdot l} \int_{r_1}^{r_2} \frac{1}{r} dr = \frac{1}{\lambda \cdot 2 \pi \cdot l} \left[ \ln |r| \right]_{r_1}^{r_2}
$$

Jelikož poloměry $r_1$ a $r_2$ jsou kladné (záporné poloměry nedávají fyzikální smysl), můžeme absolutní hodnotu u logaritmu zanedbat:
$$
    R_{\lambda} = \frac{1}{\lambda \cdot 2 \pi \cdot l} \left[ \ln r \right]_{r_1}^{r_2} = \frac{1}{\lambda \cdot 2 \pi \cdot l} \left( \ln r_2 - \ln r_1 \right) = \frac{1}{\lambda \cdot 2 \pi \cdot l} \ln \frac{r_2}{r_1} = \frac{1}{2 \pi l \lambda} \ln \frac{r_2}{r_1}.
$$

Pro odpor přenosu tepla z povrchu válce do okolí $R_{\alpha}$ využijeme vztah pro odpor přenosu tepla $R_{\alpha}$:
$$
    R_{\alpha} = \frac{1}{\alpha \cdot S},
$$
kde:\\
$\alpha$ - koeficient prostupu tepla (W/m$^2 \cdot$K).\\

Plochu $S$ nahradíme plochou válce:
$$
    S \rightarrow 2 \pi r_2 \cdot l.
$$

Odpor přenosu tepla z povrchu válce do okolí $R_{\alpha}$ bude tedy:
$$
    R_{\alpha} = \frac{1}{\alpha \cdot 2 \pi r_2 \cdot l} = \frac{1}{2 \pi l \alpha r_2}.
$$

Celkový tepelný odpor válce $R_{\Sigma}$ bude součtem obou odporů:
$$
    R_{\Sigma} = R_{\lambda} + R_{\alpha} = \frac{1}{2 \pi l \lambda} \ln \frac{r_2}{r_1} + \frac{1}{2 \pi l \alpha r_2}
$$


\subsection{Minimum tepelného odporu}

Pro nalezení extrému tepelného odporu $R_{\Sigma}$ podle poloměru válce $r_2$ je třeba zjistit, kdy bude derivace $R_{\Sigma}$ podle $r_2$ rovna nule:
$$
    \frac{dR_{\Sigma}}{dr_2} = 0.
$$

Derivace $R_{\Sigma}$ podle $r_2$ bude:
$$
    \frac{dR_{\Sigma}}{dr_2} = \frac{dR_{\lambda}}{dr_2} + \frac{dR_{\alpha}}{dr_2}.
$$

Derivace $R_{\lambda}$ podle $r_2$ bude:
$$
    \frac{dR_{\lambda}}{dr_2} = \frac{d}{dr_2} \left( \frac{1}{2 \pi l \lambda} \ln \frac{r_2}{r_1} \right) = \frac{d}{dr_2} \left( \frac{1}{2 \pi l \lambda} \left(\ln r_2 - \ln r_1 \right) \right) =
$$

$$
    = \frac{d}{dr_2} \left( \frac{1}{2 \pi l \lambda} \ln r_2 \right) = \frac{1}{2 \pi l \lambda r_2}.
$$

Derivace $R_{\alpha}$ podle $r_2$ bude:
$$
    \frac{dR_{\alpha}}{dr_2} = \frac{d}{dr_2} \left( \frac{1}{2 \pi l \alpha r_2} \right) = -\frac{1}{2 \pi l \alpha r_2^2}.
$$

Derivace $R_{\Sigma}$ podle $r_2$ bude tedy:
$$
    \frac{dR_{\Sigma}}{dr_2} = \frac{1}{2 \pi l \lambda r_2} - \frac{1}{2 \pi l \alpha r_2^2}.
$$

Nyní můžeme zjistit, kdy bude derivace $R_{\Sigma}$ podle $r_2$ rovna nule:
$$
    \frac{1}{2 \pi l \lambda r_2} - \frac{1}{2 \pi l \alpha r_2^2} = 0
$$

$$
    \frac{1}{\lambda r_2} - \frac{1}{\alpha r_2^2} = 0
$$

$$
    \frac{1}{\lambda r_2} = \frac{1}{\alpha r_2^2}
$$

$$
    \frac{r_2}{\lambda} = \frac{1}{\alpha}
$$

$$
    r_2 = \frac{\lambda}{\alpha}.
$$

V další části budeme zkoumat, zda se jedná o minimum nebo maximum. K tomu využijeme druhou derivaci $R_{\Sigma}$ podle $r_2$:
$$
    \frac{d^2R_{\Sigma}}{dr_2^2} = \frac{d}{dr_2} \left( \frac{1}{2 \pi l \lambda r_2} - \frac{1}{2 \pi l \alpha r_2^2} \right) = -\frac{1}{2 \pi l \lambda r_2^2} + \frac{1}{\pi l \alpha r_2^3}.
$$

Nyní dosadíme $r_2 = \frac{\lambda}{\alpha}$:
$$
    -\frac{1}{2 \pi l \lambda \left( \frac{\lambda}{\alpha} \right)^2} + \frac{1}{\pi l \alpha \left( \frac{\lambda}{\alpha} \right)^3} = -\frac{1}{2 \pi l \lambda \frac{\lambda^2}{\alpha^2}} + \frac{1}{\pi l \alpha \frac{\lambda^3}{\alpha^3}} =
$$

$$
    = -\frac{\alpha^2}{2 \pi l \lambda^3} + \frac{\alpha^2}{\pi l \lambda^3} = \frac{-\alpha^2 + 2 \alpha^2}{2 \pi l \lambda^3} = \frac{\alpha^2}{2 \pi l \lambda^3}.
$$

Hodnoty $\alpha$ a $\lambda$ jsou kladné, tudíž druhá derivace je kladná. To znamená, že se jedná o minimum.


\subsection{Ekonomie}

Nyní můžeme porovnat rozdíl v rychlosti růstu tepelného odporu $R_{\Sigma}$ a objemu izolace v závislosti na poloměru válce $r_2$. Řešíme objem izolace, jelikož při konstrukci se platí za objem materiálu. Pro objem izolace platí:
$$
    V = \pi (r_2^2 - r_1^2) \cdot l = \pi r_2^2 \cdot l - \pi r_1^2 \cdot l.
$$

Nyní provedeme zjednodušení vzorce podobně jako se provádí u časové složitosti algoritmů. Zkoumáme rychlost růstu objemu izolace v závislosti na poloměru válce $r_2$, tudíž vyřadíme konstantní členy, čímž dostaneme:
$$
    V \sim \pi r_2^2 \cdot l.
$$

Dále vyřadíme všechny násobící konstanty, čímž dostaneme:
$$
    V \sim r_2^2.
$$

Zde dostáváme, že objem roste s druhou mocninou poloměru válce $r_2$. Pro odpor $R_{\Sigma}$ platí:
$$
    R_{\Sigma} = \frac{1}{2 \pi l \lambda} \ln \frac{r_2}{r_1} + \frac{1}{2 \pi l \alpha r_2}.
$$

Zde druhý člen jde do nuly, když $r_2$ jde do nekonečna, tudíž ho můžeme vyřadit. Zbyde nám:
$$
    R_{\Sigma} \sim \frac{1}{2 \pi l \lambda} \ln \frac{r_2}{r_1}.
$$

Dále můžeme vyřadit konstantní členy a dostaneme:
$$
    R_{\Sigma} \sim \ln r_2.
$$

Zde vidíme, že odpor roste logaritmicky s poloměrem válce $r_2$. Pokud porovnáme rychlost růstu objemu izolace a rychlost růstu odporu $R_{\Sigma}$, zjistíme, že objem izolace roste rychleji než odpor $R_{\Sigma}$. To znamená, že při dimenzování izolace je třeba brát v potaz i ekonomické hledisko, jelikož při velkém přidání izolace nám dramaticky může narůst cena, ale odpor $R_{\Sigma}$ se nám příliš nezmění.


\subsection{Číselný příklad 1 \spicy \spicy}



\subsection{Číselný příklad 2 \spicy \spicy}



\section{Koule \spicy \spicy \spicy}


\subsection{Odvození tepelného odporu}

Tepelný odpor koule se skládá ze dvou částí:
\begin{itemize}
    \item odpor vedení tepla v materiálu koule $R_{\lambda}$,
    \item odpor přenosu tepla z povrchu koule do okolí $R_{\alpha}$.
\end{itemize}

Pro odvození tepelného odporu $R_{\lambda}$ využijeme vztah pro tepelný odpor $R$:
$$
    R = \frac{d}{\lambda \cdot S},
$$
kde:\\
$d$ - tloušťka materiálu (m),\\
$\lambda$ - tepelná vodivost materiálu (W/m$\cdot$K),\\
$S$ - plocha přenosu tepla (m$^2$).\\

Zde $d$ nahradíme nekonečně malou částí poloměru koule:
$$
    d \rightarrow dr.
$$

Dále $S$ nahradíme plochou koule:
$$
    S \rightarrow 4 \pi r^2.
$$

Tepelný odpor $dR$ koule bude tedy:
$$
    dR = \frac{dr}{\lambda \cdot 4 \pi r^2}.
$$

Pro získání celkového odporu $R_{\lambda}$ je třeba provést integraci od vnitřního poloměru izolace $r_1$ po vnější poloměr koule $r_2$:
$$
    R_{\lambda} = \int_{r_1}^{r_2} \frac{1}{\lambda \cdot 4 \pi r^2} dr = \frac{1}{\lambda \cdot 4 \pi} \int_{r_1}^{r_2} \frac{1}{r^2} dr = \frac{1}{\lambda \cdot 4 \pi} \left[ -\frac{1}{r} \right]_{r_1}^{r_2}
$$

Po dosazení mezí integrace dostaneme:
$$
    R_{\lambda} = \frac{1}{\lambda \cdot 4 \pi} \left( -\frac{1}{r_2} + \frac{1}{r_1} \right) = \frac{1}{\lambda \cdot 4 \pi} \left( \frac{1}{r_1} - \frac{1}{r_2} \right) = \frac{1}{4 \pi \lambda} \left( \frac{1}{r_1} - \frac{1}{r_2} \right)
$$

Pro odpor přenosu tepla z povrchu koule do okolí $R_{\alpha}$ využijeme vztah pro odpor přenosu tepla $R_{\alpha}$:
$$
    R_{\alpha} = \frac{1}{\alpha \cdot S},
$$
kde:\\
$\alpha$ - koeficient prostupu tepla (W/m$^2 \cdot$K).\\

Plochu $S$ nahradíme plochou koule:
$$
    S \rightarrow 4 \pi r_2^2.
$$

Odpor přenosu tepla z povrchu koule do okolí $R_{\alpha}$ bude tedy:
$$
    R_{\alpha} = \frac{1}{\alpha \cdot 4 \pi r_2^2} = \frac{1}{4 \pi \alpha r_2^2}.
$$

Celkový tepelný odpor koule $R_{\Sigma}$ bude součtem obou odporů:
$$
    R_{\Sigma} = R_{\lambda} + R_{\alpha} = \frac{1}{4 \pi \lambda} \left( \frac{1}{r_1} - \frac{1}{r_2} \right) + \frac{1}{4 \pi \alpha r_2^2}
$$

Nyní pojďme vyšetřit limitu odporu pro $r_2$ jdoucí do nekonečna:
$$
    \lim_{r_2 \to \infty} R_{\Sigma} = \lim_{r_2 \to \infty} \left( \frac{1}{4 \pi \lambda} \left( \frac{1}{r_1} - \frac{1}{r_2} \right) + \frac{1}{4 \pi \alpha r_2^2} \right)
$$

Členy, kde se vyskytuje $r_2$ ve jmenovateli, jdou do nuly, čímž dostaneme:
$$
    \lim_{r_2 \to \infty} R_{\Sigma} = \frac{1}{4 \pi \lambda r_1}.
$$

Zde vidíme, že koule nelze úplně izolovat, jelikož při nekonečné izolaci bude mít koule stále nějaký odpor.


\subsection{Minimum tepelného odporu}

Pro nalezení extrému tepelného odporu $R_{\Sigma}$ podle poloměru koule $r_2$ je třeba zjistit, kdy bude derivace $R_{\Sigma}$ podle $r_2$ rovna nule:
$$
    \frac{dR_{\Sigma}}{dr_2} = 0.
$$

Derivace $R_{\Sigma}$ podle $r_2$ bude:
$$
    \frac{dR_{\Sigma}}{dr_2} = \frac{dR_{\lambda}}{dr_2} + \frac{dR_{\alpha}}{dr_2}.
$$

Derivace $R_{\lambda}$ podle $r_2$ bude:
$$
    \frac{dR_{\lambda}}{dr_2} = \frac{d}{dr_2} \left( \frac{1}{4 \pi \lambda} \left( \frac{1}{r_1} - \frac{1}{r_2} \right) \right) = \frac{1}{4 \pi \lambda r_2^2}
$$

Derivace $R_{\alpha}$ podle $r_2$ bude:
$$
    \frac{dR_{\alpha}}{dr_2} = \frac{d}{dr_2} \left( \frac{1}{4 \pi \alpha r_2^2} \right) = -\frac{2}{4 \pi \alpha r_2^3} = -\frac{1}{2 \pi \alpha r_2^3}.
$$

Derivace $R_{\Sigma}$ podle $r_2$ bude tedy:
$$
    \frac{dR_{\Sigma}}{dr_2} = \frac{1}{4 \pi \lambda r_2^2} - \frac{1}{2 \pi \alpha r_2^3}.
$$

Nyní můžeme zjistit, kdy bude derivace $R_{\Sigma}$ podle $r_2$ rovna nule:
$$
    \frac{1}{4 \pi \lambda r_2^2} - \frac{1}{2 \pi \alpha r_2^3} = 0
$$

$$
    \frac{1}{4 \lambda r_2^2} - \frac{1}{2 \alpha r_2^3} = 0
$$

$$
    \frac{1}{4 \lambda r_2^2} = \frac{1}{2 \alpha r_2^3}
$$

$$
    \frac{1}{\lambda r_2^2} = \frac{2}{\alpha r_2^3}
$$

$$
    \frac{r_2}{\lambda} = \frac{2}{\alpha}
$$

$$
    r_2 = \frac{2 \lambda}{\alpha}.
$$

V další části budeme zkoumat, zda se jedná o minimum nebo maximum. K tomu využijeme druhou derivaci $R_{\Sigma}$ podle $r_2$:


\subsection{Ekonomie}


\subsection{Číselný příklad}



\end{document}
