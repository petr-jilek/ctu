\documentclass{article}
\usepackage[czech]{babel} % Czech language
\usepackage[shortlabels]{enumitem} % Custom enumeration
\usepackage{graphicx} % Import images
\usepackage{float} % Use [H] to force figure position
\usepackage{indentfirst} % Indent first paragraph
\usepackage{emoji} % Emojis
\usepackage{pgffor} % Loops
\usepackage{tikz} % TikZ
\usepackage{pgfplots} % TikZ plots
\usepackage{amsmath} % Math
\usetikzlibrary{arrows.meta}

\makeatletter
\providecommand\add@text{}
\newcommand\tagaddtext[1]{%
    \gdef\add@text{#1\gdef\add@text{}}}% 
\renewcommand\tagform@[1]{%
    \maketag@@@{\llap{\add@text\quad}(\ignorespaces#1\unskip\@@italiccorr)}%
}
\makeatother

\newcommand{\cvHead}[1]{\head{Cvičení #1}}

\newcommand{\head}[1]{
    \title{\textbf{#1}\\Elektroenergetika 3}
    \author{Petr Jílek}
    \date{2024}
}

\newcommand{\spicy}{\emoji{hot-pepper}}

% --------------------
% --------------------
% Units
% --------------------
% --------------------

% --------------------
% Basic units
% --------------------  

% no unit
\newcommand{\uNOUNIT}{\te{--}} % no unit
\newcommand{\uPERCENT}{\te{\%}} % percent
\newcommand{\uDEGREE}{^\circ} % degree
\newcommand{\uRAD}{\te{rad}} % radian

% meter
\newcommand{\uM}{\te{m}} % meter
\newcommand{\uMsq}{\uM^\te{2}} % meter squared
\newcommand{\uMcu}{\uM^\te{3}} % meter cubed
\newcommand{\uMquar}{\uM^\te{4}} % meter quartic
\newcommand{\uMinv}{\uM^{-1}} % meter inverse
\newcommand{\uMinvsq}{\uM^{-2}} % meter inverse squared
\newcommand{\uMinvcu}{\uM^{-3}} % meter inverse cubed
\newcommand{\uMinvquar}{\uM^{-4}} % meter inverse quartic
\newcommand{\uMM}{\te{mm}} % millimeter
\newcommand{\uCM}{\te{cm}} % centimeter
\newcommand{\uKM}{\te{km}} % kilometer

% liter
\newcommand{\uL}{\te{l}} % liter
\newcommand{\uML}{\te{ml}} % milliliter

% second
\newcommand{\uS}{\te{s}} % second
\newcommand{\uSinv}{\uS^{-1}} % second inverse
\newcommand{\uSinvsq}{\uS^{-2}} % second inverse squared

% hour
\newcommand{\uH}{\te{h}} % hour
\newcommand{\uHinv}{\te{h}^{-1}} % hour inverse

% kilogram
\newcommand{\uKG}{\te{kg}} % kilogram
\newcommand{\uKGinv}{\uKG^{-1}} % kilogram inverse
\newcommand{\uKGinvsq}{\uKG^{-2}} % kilogram inverse squared
\newcommand{\uG}{\te{g}} % gram

% lumen
\newcommand{\uLM}{\te{lm}} % lumen

% kelvin
\newcommand{\uK}{\te{K}} % kelvin
\newcommand{\uKsq}{\uK^\te{2}} % kelvin squared
\newcommand{\uKcu}{\uK^\te{3}} % kelvin cubed
\newcommand{\uKquar}{\uK^\te{4}} % kelvin quartic
\newcommand{\uKinv}{\uK^{-1}} % kelvin inverse
\newcommand{\uKinvsq}{\uK^{-2}} % kelvin inverse squared
\newcommand{\uKinvcu}{\uK^{-3}} % kelvin inverse cubed
\newcommand{\uKinvquar}{\uK^{-4}} % kelvin inverse quartic

% degree celsius
\newcommand{\uCELS}{\uDEGREE \te{C}} % degree celsius

% watt
\newcommand{\uW}{\te{W}} % watt
\newcommand{\uKW}{\te{kW}} % kilowatt
\newcommand{\uMW}{\te{MW}} % megawatt
\newcommand{\uGW}{\te{GW}} % gigawatt
\newcommand{\uWinv}{\uW^{-1}} % watt inverse

% joule
\newcommand{\uJ}{\te{J}} % joule
\newcommand{\uKJ}{\te{kJ}} % kilojoule
\newcommand{\uMJ}{\te{MJ}} % megajoule
\newcommand{\uCAL}{\te{cal}} % calorie
\newcommand{\uKCAL}{\te{kcal}} % kilocalorie
\newcommand{\uWH}{\te{Wh}} % watt hour
\newcommand{\uKWH}{\te{kWh}} % kilowatt hour
\newcommand{\uWandS}{\te{Ws}} % watt second

% ampere
\newcommand{\uA}{\te{A}} % ampere

% volt
\newcommand{\uV}{\te{V}} % volt

% ohm
\newcommand{\uOHM}{\te{\Omega}} % ohm
\newcommand{\uOHMinv}{\uOHM^{-1}} % ohm inverse

% siemens
\newcommand{\uSIE}{\te{S}} % siemens
\newcommand{\uSIEinv}{\uSIE^{-1}} % siemens inverse

% newton
\newcommand{\uN}{\te{N}} % newton

% currency
\newcommand{\uCCY}{\te{CCY}} % currency
\newcommand{\uCZK}{\te{CZK}} % czech crown

% --------------------
% Compound units
% --------------------

% velocity-acceleration
\newcommand{\uKGandMinvcu}{\uKG \cdot \uMinvcu} % kilogram per meter cubed
\newcommand{\uMandSinv}{\uM \cdot \uSinv} % meter per second
\newcommand{\uMcuSinv}{\uMcu \cdot \uSinv} % meter cubed per second
\newcommand{\uMandSinvsq}{\uM \cdot \uSinvsq} % meter per second squared

% power-energy
\newcommand{\uWandMinvsq}{\uW \cdot \uMinvsq} % watt per meter squared
\newcommand{\uWandMinvcu}{\uW \cdot \uMinvcu} % watt per meter cubed
\newcommand{\uJandMinvsq}{\uJ \cdot \uMinvsq} % joule per meter squared
\newcommand{\uJandMinvcu}{\uJ \cdot \uMinvcu} % joule per meter cubed

% heat
\newcommand{\uJandKGinvKinv}{\uJ \cdot \uKGinv \cdot \uKinv} % joule per kilogram per kelvin
\newcommand{\uWandMinvKinv}{\uW \cdot \uMinv \cdot \uKinv} % watt per meter per kelvin
\newcommand{\uMsqKandWinv}{\uMsq \cdot \uK \cdot \uWinv} % meter squared kelvin per watt
\newcommand{\uKandWinv}{\uK \cdot \uWinv} % kelvin per watt inverse
\newcommand{\uWandMinvsqKinv}{\uW \cdot \uMinvsq \cdot \uKinv} % watt per meter squared per kelvin
\newcommand{\uWandKinv}{\uW \cdot \uKinv} % watt per kelvin inverse

% electricity
\newcommand{\uOHMandMinv}{\uOHM \cdot \uMinv} % ohm per meter
\newcommand{\uSIEandMinv}{\uSIE \cdot \uMinv} % siemens per meter
\newcommand{\uAandMinvsq}{\uA \cdot \uMinvsq} % ampere per meter squared
\newcommand{\uVandMinv}{\uV \cdot \uMinv} % volt per meter


% --------------------
% --------------------
% Units in equation
% --------------------
% --------------------

% --------------------
% Basic units
% --------------------

% no unit
\newcommand{\ueqNOUNIT}{$\uNOUNIT$} % no unit
\newcommand{\ueqPERCENT}{$\uPERCENT$} % percent
\newcommand{\ueqDEGREE}{$\uDEGREE$} % degree
\newcommand{\ueqRAD}{$\uRAD$} % radian

% meter
\newcommand{\ueqM}{$\uM$} % meter
\newcommand{\ueqMsq}{$\uMsq$} % meter squared
\newcommand{\ueqMcu}{$\uMcu$} % meter cubed
\newcommand{\ueqMquar}{$\uMquar$} % meter quartic
\newcommand{\ueqMinv}{$\uMinv$} % meter inverse
\newcommand{\ueqMinvsq}{$\uMinvsq$} % meter inverse squared
\newcommand{\ueqMinvcu}{$\uMinvcu$} % meter inverse cubed
\newcommand{\ueqMinvquar}{$\uMinvquar$} % meter inverse quartic
\newcommand{\ueqMM}{$\uMM$} % millimeter
\newcommand{\ueqCM}{$\uCM$} % centimeter
\newcommand{\ueqKM}{$\uKM$} % kilometer

% liter
\newcommand{\ueqL}{$\uL$} % liter
\newcommand{\ueqML}{$\uML$} % milliliter

% second
\newcommand{\ueqS}{$\uS$} % second
\newcommand{\ueqSinv}{$\uSinv$} % second inverse
\newcommand{\ueqSinvsq}{$\uSinvsq$} % second inverse squared

% hour
\newcommand{\ueqH}{$\uH$} % hour
\newcommand{\ueqHinv}{$\uHinv$} % hour inverse

% kilogram
\newcommand{\ueqKG}{$\uKG$} % kilogram
\newcommand{\ueqKGinv}{$\uKGinv$} % kilogram inverse
\newcommand{\ueqKGinvsq}{$\uKGinvsq$} % kilogram inverse squared
\newcommand{\ueqG}{$\uG$} % gram

% lumen
\newcommand{\ueqLM}{$\uLM$} % lumen

% kelvin
\newcommand{\ueqK}{$\uK$} % kelvin
\newcommand{\ueqKsq}{$\uKsq$} % kelvin squared
\newcommand{\ueqKcu}{$\uKcv$} % kelvin cubed
\newcommand{\ueqKquar}{$\uKquar$} % kelvin quartic
\newcommand{\ueqKinv}{$\uKinv$} % kelvin inverse
\newcommand{\ueqKinvsq}{$\uKinvsq$} % kelvin inverse squared
\newcommand{\ueqKinvcu}{$\uKinvcu$} % kelvin inverse cubed
\newcommand{\ueqKinvquar}{$\uuKinvquar$} % kelvin inverse quartic

% degree celsius
\newcommand{\ueqCELS}{$\uCELS$} % degree celsius

% watt
\newcommand{\ueqW}{$\uW$} % watt
\newcommand{\ueqKW}{$\uKW$} % kilowatt
\newcommand{\ueqMW}{$\uMW$} % megawatt
\newcommand{\ueqGW}{$\uGW$} % gigawatt
\newcommand{\ueqWinv}{$\uWinv$} % watt inverse

% joule
\newcommand{\ueqJ}{$\uJ$} % joule
\newcommand{\ueqKJ}{$\uKJ$} % kilojoule
\newcommand{\ueqMJ}{$\uMJ$} % megajoule
\newcommand{\ueqCAL}{$\uCAL$} % calorie
\newcommand{\ueqKCAL}{$\uKCAL$} % kilocalorie
\newcommand{\ueqWH}{$\uWH$} % watt hour
\newcommand{\ueqKWH}{$\uKWH$} % kilowatt hour
\newcommand{\ueqWandS}{$\uWandS$} % watt second

% ampere
\newcommand{\ueqA}{$\uA$} % ampere

% volt
\newcommand{\ueqV}{$\uV$} % volt

% ohm
\newcommand{\ueqOHM}{$\uOHM$} % ohm
\newcommand{\ueqOHMinv}{$\uOHMinv$} % ohm inverse

% siemens
\newcommand{\ueqSIE}{$\uSIE$} % siemens
\newcommand{\ueqSIEinv}{$\uSIEinv$} % siemens inverse

% newton
\newcommand{\ueqN}{$\uN$} % newton

% currency
\newcommand{\ueqCCY}{$\uCCY$} % currency
\newcommand{\ueqCZK}{$\uCZK$} % czech crown

% --------------------
% Compound units
% --------------------

% velocity-acceleration
\newcommand{\ueqKGandMinvcu}{$\uKGandMinvcu$} % kilogram per meter cubed
\newcommand{\ueqMandSinv}{$\uMandSinv$} % meter per second
\newcommand{\ueqMcuSinv}{$\uMcuSinv$} % meter cubed per second
\newcommand{\ueqMandSinvsq}{$\uMandSinvsq$} % meter per second squared

% power-energy
\newcommand{\ueqWandMinvsq}{$\uWandMinvsq$} % watt per meter squared
\newcommand{\ueqWandMinvcu}{$\uWandMinvcu$} % watt per meter cubed
\newcommand{\ueqJandMinvsq}{$\uJandMinvsq$} % joule per meter squared
\newcommand{\ueqJandMinvcu}{$\uJandMinvcu$} % joule per meter cubed

% heat
\newcommand{\ueqJandKGinvKinv}{$\uJandKGinvKinv$} % joule per kilogram per kelvin
\newcommand{\ueqWandMinvKinv}{$\uWandMinvKinv$} % watt per meter per kelvin
\newcommand{\ueqMsqKandWinv}{$\uMsqKandWinv$} % meter squared kelvin per watt
\newcommand{\ueqKandWinv}{$\uKandWinv$} % kelvin per watt inverse
\newcommand{\ueqWandMinvsqKinv}{$\uWandMinvsqKinv$} % watt per meter squared per kelvin
\newcommand{\ueqWandKinv}{$\uWandKinv$} % watt per kelvin inverse

% electricity
\newcommand{\ueqOHMandMinv}{$\uOHMandMinv$} % ohm per meter
\newcommand{\ueqSIEandMinv}{$\uSIEandMinv$} % siemens per meter
\newcommand{\ueqAandMinvsq}{$\uAandMinvsq$} % ampere per meter squared
\newcommand{\ueqVandMinv}{$\uVandMinv$} % volt per meter


\cvHead{2 - Sdílení tepla}

\begin{document}

\maketitle
\tableofcontents
\newpage



\section{\emoji{brick} Zeď \spicy \spicy}


\subsection{Fourierova-Kirchhoffova rovnice \spicy \spicy \spicy}
Pro zeď o tloušťce $d$ budeme uvažovat následující zjednodušející předpoklady:
\begin{itemize}
    \item $\frac{\partial T}{\partial t}$ = 0,
    \item $\vec{v}$ = 0,
    \item $Q_v$ = 0,
    \item $\lambda$ = konstanta,
    \item $T = T(x)$ - teplota závislá pouze na x.
\end{itemize}

Poté Fourierova-Kirchhoffova rovnice bude mít tvar:
$$
    0 = \nabla \cdot \left( \lambda \cdot \vec{\nabla} T \right) = \lambda \cdot \frac{d^2 T}{d x^2}.
$$

Rovnici můžeme vydělit $\lambda$ ($\lambda > 0$) a získáme:
$$
    \frac{d^2 T}{d x^2} = 0.
$$

Tuto rovnici můžeme řešit dvojí integrací. První integrace bude vypadat následovně:
$$
    \frac{d T}{d x} = \int 0 \cdot dx = 0 + c_1 = c_1.
$$

Druhá integrace bude vypadat následovně:
$$
    T(x) = \int c_1 \cdot dx = c_1 \cdot x + c_2.
$$

Obecné řešení bude tedy:
$$
    T(x) = c_1 \cdot x + c_2.
$$

S okrajovými podmínkami:
\begin{itemize}
    \item $T(0) = T_1$:
          $$
              T_1 = T(0) = c_1 \cdot 0 + c_2 = c_2 \Rightarrow c_2 = T_1.
          $$
    \item $T(d) = T_2$:
          $$
              T_2 = T(d) = c_1 \cdot d + c_2 \Rightarrow c_1 = \frac{T_2 - c_2}{d} = \frac{T_2 - T_1}{d}.
          $$
\end{itemize}

Řešení je tedy:
$$
    T(x) = \frac{T_2 - T_1}{d} \cdot x + T_1.
$$

Tuto situaci můžeme znázornit následujícím obrázkem:

\begin{center}
    \begin{tikzpicture}
        % Variables
        \def\TOne{4} % T1
        \def\TTwo{1} % T2
        \def\d{5} % width of the wall, d

        % Axes
        \draw[->] (0,0) -- (\d + 1,0) node[anchor=north] {$x$}; % x-axis
        \draw[->] (0,0) -- (0,\TOne + 1) node[anchor=east] {$T(x)$}; % y-axis

        % Line representing T(x)
        \draw[thick, red] (-2,\TOne) -- (0,\TOne);
        \draw[thick, red] (0,\TOne) -- (\d,\TTwo);
        \draw[thick, red] (\d,\TTwo) -- (\d + 2,\TTwo);

        % Dotted line
        \draw[dashed] (0,\TOne) -- (1,\TOne);
        \draw[dashed] (0,\TTwo) -- (\d,\TTwo);

        % Wall lines
        \draw[very thick] (0,0) -- (0,\TOne + 0.5);
        \draw[very thick] (\d,0) -- (\d,\TOne - 1.5);
        \draw[very thick] (\d,\TOne - 0.5) -- (\d,\TOne + 0.5);

        % Labels
        \node[anchor=south] at (4, 2.7) {$T(x) = \frac{T_2 - T_1}{d} \cdot x + T_1$};
        \node[anchor=west] at (1, \TOne) {$T_1$};
        \node[anchor=east] at (0, \TTwo) {$T_2$};
        \node[anchor=north] at (\d, 0) {$d$};
        \node[anchor=north] at (0, 0) {$0$};
    \end{tikzpicture}
\end{center}


\subsection{Fourieruv zákon \spicy \spicy \spicy}
Nyní dosadíme řešení z předchozího příkladu do Fourierova zákona, kde za gradient teploty dosadíme derivaci teploty podle osy x, čímž získáme měrný tepelný tok $q_x$:
$$
    \dot{q_x} = - \lambda \cdot \frac{dT(x)}{dx} = - \lambda \cdot \frac{d}{dx} \left( \frac{T_2 - T_1}{d} \cdot x + T_1 \right) = - \lambda \cdot \frac{T_2 - T_1}{d} = \lambda \cdot \frac{T_1 - T_2}{d} = \frac{\Delta T}{\frac{d}{\lambda}}.
$$


\subsection{Tepelný odpor}
Dolní výraz z předchozí sekce $\frac{d}{\lambda}$ je tepelný odpor:
\begin{equation}
    R_{\vartheta} = \frac{d}{\lambda}.
    \unit{(\ueqMsqKandWinv)}
\end{equation}

Měrný tepelný tok $q_x$ můžeme tedy zapsat jako:
\begin{equation}
    \dot{q_x} = \frac{\Delta T}{R_{\vartheta}}.
    \unit{(\ueqWandMinvsq)}
\end{equation}

Tepelný tok $\dot{Q_x}$ dostaneme vynásobením měrného tepelného toku plochou průřezu $S$:
\begin{equation}
    \dot{Q_x} = \dot{q_x} \cdot S.
    \unit{(\ueqW)}
\end{equation}

Tepelný tok $\dot{Q_x}$ můžeme dále rozepsat:
$$
    \dot{Q_x} = \dot{q_x} \cdot S = \lambda \frac{T_1 - T_2}{d} \cdot S = \lambda \frac{\Delta T}{d} \cdot S = \frac{\Delta T}{\frac{d}{\lambda \cdot S}}.
$$

Dolní výraz $\frac{d}{\lambda \cdot S}$ je absolutní tepelný odpor:
\begin{equation}
    R_{\vartheta A} = \frac{d}{\lambda \cdot S} = \frac{R_{\vartheta}}{S}.
    \unit{(\ueqKandWinv)}
\end{equation}

\textit{Poznámka}\\

Výpočet tepelného odporu je analogický jako výpočet elektrického odporu. Elektrický odpor $R_e$ vypočteme jako:
\begin{equation}
    R_e = \frac{l}{\sigma_e \cdot S},
    \unit{(\ueqOHM)}
\end{equation}
kde:\\
$l$ -- délka vodiče (\ueqM),\\
$\sigma_e$ -- měrná elektrická vodivost (\ueqSIEandMinv),\\
$S$ -- průřez vodiče (\ueqMsq).\\

Analogie jsou:
\begin{itemize}
    \item $R_{\vartheta A}$ -- absolutní tepelný odpor (\ueqKandWinv) $\rightarrow$ $R_e$ -- elektrický odpor (\ueqOHM),
    \item $d$ -- tloušťka stěny (\ueqM) $\rightarrow$ $l$ -- délka vodiče (m),
    \item $\lambda$ -- tepelná vodivost (\ueqWandMinvKinv) $\rightarrow$ $\sigma_e$ -- měrná elektrická vodivost (\ueqSIEandMinv),
    \item $S$ -- plocha průřezu (\ueqMsq) $\rightarrow$ $S$ -- průřez vodiče (m2).
\end{itemize}

Výpočet tepelného toku je analogický s Ohmovým zákonem:
\begin{equation}
    I = \frac{U}{R_e},
    \unit{(\ueqA)}
\end{equation}
kde:\\
$I$ -- proud (\ueqA),\\
$U$ -- napětí (\ueqV),\\
$R_e$ -- elektrický odpor (\ueqOHM).\\

Analogie jsou:
\begin{itemize}
    \item $\dot{Q_x}$ -- tepelný tok (\ueqW) $\rightarrow$ $I$ -- proud (\ueqA),
    \item $\Delta T$ -- rozdíl teplot (\ueqK) $\rightarrow$ $U$ -- napětí (\ueqV),
    \item $R_{\vartheta A}$ -- absolutní tepelný odpor (\ueqKandWinv) $\rightarrow$ $R_e$ -- elektrický odpor (\ueqOHM).
\end{itemize}

Elektrické schéma můžeme znázornit následujícím obrázkem:

\begin{center}
    \begin{tikzpicture}
        \draw[thick] (0.5,-1) -- (1.5,-1) node[midway, below] {$U$};
        \draw[thick] (1,0) -- (1,-1);
        \draw[thick] (1,0) -- (2.5,0);
        \draw[thick] (2.5,-0.3) rectangle (3.5,0.3) node[midway] {$R_e$};
        \draw[thick] (3.5,0) -- (5,0);
        \draw[thick] (5,0) -- (5,-1);
        \draw[thick] (5.5,-1) -- (4.5,-1) node[midway, below] {$0$};

        % Triangle arrow
        \draw[-{Triangle}] (2,0.7) -- (4,.7) node[midway, above] {$I$};
    \end{tikzpicture}
\end{center}

Analogicky můžeme vytvořit tepelné schéma:

\begin{center}
    \begin{tikzpicture}
        \draw[thick] (0.5,-1) -- (1.5,-1) node[midway, below] {$T_1$};
        \draw[thick] (1,0) -- (1,-1);
        \draw[thick] (1,0) -- (2.5,0);
        \draw[thick] (2.5,-0.3) rectangle (3.5,0.3) node[midway] {$R_{\vartheta}$};
        \draw[thick] (3.5,0) -- (5,0);
        \draw[thick] (5,0) -- (5,-1);
        \draw[thick] (5.5,-1) -- (4.5,-1) node[midway, below] {$T_2$};

        % Triangle arrow
        \draw[-{Triangle}] (2,0.7) -- (4,.7) node[midway, above] {$\dot{Q_x}$};
    \end{tikzpicture}
\end{center}


\subsection{Součinitel prostupu tepla}
Součinitel prostupu tepla $U_\vartheta$ je inverzní hodnota tepelného odporu:
\begin{equation}
    U_\vartheta = \frac{1}{R_{\vartheta}} = \frac{\lambda}{d}.
    \unit{(\ueqWandMinvsqKinv)}
\end{equation}

Měrný tepelný tok $q_x$ se poté může zapsat jako:
\begin{equation}
    \dot{q_x} = U_\vartheta \cdot \Delta T.
    \unit{(\ueqWandMinvsq)}
\end{equation}

Prostup tepla $U_{\vartheta A}$ je inverzní hodnota absolutního tepelného odporu:
\begin{equation}
    U_{\vartheta A} = \frac{1}{R_{\vartheta A}} = \frac{\lambda \cdot S}{d} = U_\vartheta \cdot S.
    \unit{(\ueqWandKinv)}
\end{equation}

Tepelný tok $\dot{Q_x}$ se poté může zapsat jako:
\begin{equation}
    \dot{Q_x} = U_{\vartheta A} \cdot \Delta T.
    \unit{(\ueqW)}
\end{equation}


\subsection{Číselný příklad}
Teplota na začátku zdi je $T_1 = 20 \fs \uCELS$, teplota na konci zdi je $T_2 = -10 \fs \uCELS$. Zeď má tloušťku $d_{cihla} = 45 \fs \uCM$ a je tvořená obyčejnou cihlou s tepelná vodivostí $\lambda_{cihla} = 0,8 \fs \uWandMinvKinv$. Plocha průřezu zdi je $S = 20 \fs \uMsq$.

\begin{enumerate}[a)]
    \item Vypočítejte $R_{\vartheta}$, $R_{\vartheta A}$, $U_\vartheta$, $U_{\vartheta A}$, $\dot{q_x}$ a $\dot{Q_x}$.
    \item Uvažujte polystyrénovou izolaci s tepelnou vodivostí $\lambda_{izol} = 0,04 \fs \uWandMinvKinv$. Vypočítejte tloušťku izolace $d_{izol}$, která zajistí stejný měrný tepelný odpor (tím také zajistí stejný měrný tepelný tok).
\end{enumerate}

\subsubsection{a}
Tepelný odpor $R_{\vartheta}$ vypočteme jako:
$$
    R_{\vartheta} = \frac{d_{cihla}}{\lambda_{cihla}} = \frac{0,45 \fs \uM}{0,8 \fs \uWandMinvKinv} = 0,5625 \fs \uMsqKandWinv.
$$

Absolutní tepelný odpor $R_{\vartheta A}$ vypočteme jako:
$$
    R_{\vartheta A} = \frac{R_{\vartheta}}{S} = \frac{0,5625}{20} \approx 0,028 \fs \uKandWinv.
$$

Součinitel prostupu tepla $U_\vartheta$ vypočteme jako:
$$
    U_\vartheta = \frac{1}{R_{\vartheta}} = \frac{1}{0,5625} \approx 1,778 \fs \uWandMinvsqKinv.
$$

Prostup tepla $U_{\vartheta A}$ vypočteme jako:
$$
    U_{\vartheta A} = \frac{1}{R_{\vartheta A}} = \frac{1}{0,028} \approx 35,714 \fs \uKandWinv.
$$

Měrný tepelný tok $\dot{q_x}$ vypočteme jako:
$$
    \dot{q_x} = \frac{\Delta T}{R_{\vartheta}} = \frac{T_1 - T_2}{R_{\vartheta}} = \frac{20 - (-10)}{0,5625} = \frac{30}{0,5625} \approx 53,33 \fs \uWandMinvsq.
$$

Tepelný tok $\dot{Q_x}$ vypočteme jako:
$$
    \dot{Q_x} = \dot{q_x} \cdot S = 53,33 \fs \uWandMinvsq \cdot 20 \fs \uMsq = 1066,6 \fs \uW \approx 1 \fs \uKW.
$$

\subsubsection{b}
Aby se měrné tepelné odpory rovnaly, musí platit:
$$
    R_{\vartheta,cihla} = R_{\vartheta,izol}.
$$
$$
    \frac{d_{cihla}}{\lambda_{cihla}} = \frac{d_{izol}}{\lambda_{izol}}.
$$

Tloušťku izolace $d_{izol}$ vypočteme jako:
$$
    d_{izol} = \frac{d_{cihla} \cdot \lambda_{izol}}{\lambda_{cihla}} = \frac{0,45 \fs \uM \cdot 0,04 \fs \uWandMinvKinv}{0,8 \fs \uWandMinvKinv} = 0,0225 \fs \uM = 2,25 \fs \uCM.
$$

\newpage



\section{\emoji{building-construction} Skládaná zeď \spicy \spicy}


\subsection{Tepelné schéma}
U skládané zdi máme několik vrstev zdi s různou tepelnou vodivostí a tloušťkou. Navíc počítáme se součinitely přestupu tepla na začátku (ze vnitř do zdi) a na konci (ze zdi ven).

Mějme tedy n vrstev zdi, kde $i$-tá vrstva má tepelnou vodivost $\lambda_i$ a tloušťku $d_i$. Mějme součinitel přestupu tepla na začátku $\alpha_{\vartheta,in}$ a na konci $\alpha_{\vartheta,out}$. Potom můžeme pro tuto situaci nakreslit následující tepelné schéma:

\begin{center}
    \begin{tikzpicture}
        \draw[thick] (0.5,-2) -- (1.5,-2) node[midway, below] {$T_1$};
        \draw[thick] (1,0) -- (1,-2);

        \draw[thick] (1,0) -- (1.5,0);
        \draw[thick] (1.5,-0.3) rectangle (2.5,0.3) node[midway] {$R_{\vartheta,in}$};
        \draw[thick] (2.5,0) -- (3,0);
        \draw[thick] (3,-0.3) rectangle (4,0.3) node[midway] {$R_{\vartheta,1}$};
        \draw[thick] (4,0) -- (4.5,0);

        \draw[dotted] (5,0) -- (6,0);

        \draw[thick] (6.5,0) -- (7,0);
        \draw[thick] (7,-0.3) rectangle (8,0.3) node[midway] {$R_{\vartheta,n}$};
        \draw[thick] (8,0) -- (8.5,0);
        \draw[thick] (8.5,-0.3) rectangle (9.5,0.3) node[midway] {$R_{\vartheta,out}$};
        \draw[thick] (9.5,0) -- (10,0);

        \draw[thick] (10,0) -- (10,-2);
        \draw[thick] (9.5,-2) -- (10.5,-2) node[midway, below] {$T_2$};

        % Triangle arrow
        \draw[-{Triangle}] (3,0.7) -- (8,.7) node[midway, above] {$\dot{Q_x}$};
    \end{tikzpicture}
\end{center}

Tepelný odpor $R_{\vartheta,in}$ vypočteme jako:
$$
    R_{\vartheta,in} = \frac{1}{\alpha_{\vartheta,in}}.
$$

Tepelný odpor $R_{\vartheta,out}$ vypočteme jako:
$$
    R_{\vartheta,out} = \frac{1}{\alpha_{\vartheta,out}}.
$$

Tepelný odpor $R_{\vartheta,i}$ vypočteme jako:
$$
    R_{\vartheta,i} = \frac{d_i}{\lambda_i}.
$$

Celkový tepelný odpor $R_{\vartheta,\Sigma}$ zdi s přechody vypočteme jako:
$$
    R_{\vartheta,\Sigma} = R_{\vartheta,in} + \sum_{i=1}^{n} R_{\vartheta,i} + R_{\vartheta,out}.
$$

Celkový absolutní tepelný odpor $R_{\vartheta A,\Sigma}$ zdi s přechody vypočteme jako:
$$
    R_{\vartheta A,\Sigma} = \frac{R_{\vartheta,\Sigma}}{S}.
$$

Součinitel prostupu tepla $U_{\vartheta,\Sigma}$ zdi s přechody vypočteme jako:
$$
    U_{\vartheta,\Sigma} = \frac{1}{R_{\vartheta,\Sigma}} = \frac{1}{R_{\vartheta,in} + \sum_{i=1}^{n} R_{\vartheta,i} + R_{\vartheta,out}} = \frac{1}{\frac{1}{\alpha_{\vartheta,in}} + \sum_{i=1}^{n} \frac{d_i}{\lambda_i} + \frac{1}{\alpha_{\vartheta,out}}} =
$$
$$
    = \left ( \frac{1}{\alpha_{\vartheta,in}} + \sum_{i=1}^{n} \frac{d_i}{\lambda_i} + \frac{1}{\alpha_{\vartheta,out}} \right )^{-1}
$$

Prostup tepla $U_{\vartheta A,\Sigma}$ zdi s přechody vypočteme jako:
$$
    U_{\vartheta A,\Sigma} = U_{\vartheta,\Sigma} \cdot S.
$$

Měrný tepelný tok $\dot{q_x}$ zdi s přechody vypočteme pomocí měrných odporů jako:
$$
    \dot{q_x} = \frac{\Delta T}{R_{\vartheta,\Sigma}} = \frac{T_1 - T_2}{R_{\vartheta,\Sigma}}.
$$

Pomocí součinitele prostupu tepla $U_{\vartheta,\Sigma}$ můžeme vypočítat měrný tepelný tok $\dot{q_x}$ jako:
$$
    \dot{q_x} = U_{\vartheta,\Sigma} \cdot \Delta T = U_{\vartheta,\Sigma} \cdot (T_1 - T_2).
$$

Celkový tepelný tok $\dot{Q_x}$ zdi s přechody vypočteme jako:
$$
    \dot{Q_x} = \dot{q_x} \cdot S.
$$


\subsection{Číselný příklad}
Teplota na začátku zdi je $T_1 = 20 \fs \uCELS$, teplota na konci zdi je $T_2 = -10 \fs \uCELS$. Plocha průřezu zdi je $S = 20 \fs \uMsq$. Uvažujme dvouvrstvou zeď složenou z cihly a polystyrénu. Parametry cihly jsou:
\begin{itemize}
    \item $\lambda_{cihla} = 0,8 \fs \uWandMinvKinv$,
    \item $d_{cihla} = 45 \fs \uCM$.
\end{itemize}

Parametry polystyrénu jsou:
\begin{itemize}
    \item $\lambda_{izol} = 0,04 \fs \uWandMinvKinv$,
    \item $d_{izol} = 5 \fs \uCM$.
\end{itemize}

Uvažujte, že izolace je na konci zdi (z venčí). Zanedbejte součinitele přestupu tepla na začátku a na konci zdi.

\begin{enumerate}[a)]
    \item Nakreslete tepelné schéma a vypočítejte celkový tepelný odpor $R_{\vartheta,\Sigma}$, celkový absolutní tepelný odpor $R_{\vartheta A,\Sigma}$, součinitel prostupu tepla $U_{\vartheta,\Sigma}$, absolutní součinitel prostupu tepla $U_{\vartheta A,\Sigma}$, měrný tepelný tok $\dot{q_x}$ a tepelný tok $\dot{Q_x}$.
    \item Nakreslete graf závislosti teploty na ose x pro případ izolace z venčí a pro případ izolace zevnitř. Diskutujte výhody a nevýhody obou případů.
\end{enumerate}

\subsubsection{a}
Tepelné schéma bude vypadat následovně:

\begin{center}
    \begin{tikzpicture}
        \draw[thick] (0.5,-2) -- (1.5,-2) node[midway, below] {$T_1$};
        \draw[thick] (1,0) -- (1,-2);

        \draw[thick] (1,0) -- (1.5,0);
        \draw[thick] (1.5,-0.3) rectangle (3,0.3) node[midway] {$R_{\vartheta,cihla}$};
        \draw[thick] (3,0) -- (4,0);
        \draw[thick] (4,-0.3) rectangle (5.5,0.3) node[midway] {$R_{\vartheta,izol}$};
        \draw[thick] (5.5,0) -- (6,0);

        \draw[thick] (6,0) -- (6,-2);
        \draw[thick] (5.5,-2) -- (6.5,-2) node[midway, below] {$T_2$};

        % Triangle arrow
        \draw[-{Triangle}] (2.5,0.7) -- (4.5,.7) node[midway, above] {$\dot{Q_x}$};
    \end{tikzpicture}
\end{center}

Tepelný odpor cihly $R_{\vartheta,cihla}$ vypočteme jako:
$$
    R_{\vartheta,cihla} = \frac{d_{cihla}}{\lambda_{cihla}} = \frac{0,45 \fs \uM}{0,8 \fs \uWandMinvKinv} = 0,5625 \fs \uMsqKandWinv.
$$

Tepelný odpor izolace $R_{\vartheta,izol}$ vypočteme jako:
$$
    R_{\vartheta,izol} = \frac{d_{izol}}{\lambda_{izol}} = \frac{0,05 \fs \uM}{0,04 \fs \uWandMinvKinv} = 1,25 \fs \uMsqKandWinv.
$$

Absolutní tepelný odpor cihly $R_{\vartheta A,cihla}$ vypočteme jako:
$$
    R_{\vartheta A,cihla} = \frac{R_{\vartheta,cihla}}{S} = \frac{0,5625}{20} \approx 0,028 \fs \uKandWinv.
$$

Absolutní tepelný odpor izolace $R_{\vartheta A,izol}$ vypočteme jako:
$$
    R_{\vartheta A,izol} = \frac{R_{\vartheta,izol}}{S} = \frac{1,25}{20} = 0,063 \fs \uKandWinv.
$$

Celkový tepelný odpor $R_{\vartheta,\Sigma}$ vypočteme jako:
$$
    R_{\vartheta,\Sigma} = R_{\vartheta,cihla} + R_{\vartheta,izol} = 0,5625 + 1,25 = 1,8125 \fs \uMsqKandWinv.
$$

Celkový absolutní tepelný odpor $R_{\vartheta A,\Sigma}$ vypočteme jako:
$$
    R_{\vartheta A,\Sigma} = R_{\vartheta A,cihla} + R_{\vartheta A,izol} = 0,028 + 0,063 = 0,091 \fs \uKandWinv.
$$

Součinitel prostupu tepla $U_{\vartheta,\Sigma}$ vypočteme jako:
$$
    U_{\vartheta,\Sigma} = \frac{1}{R_{\vartheta,\Sigma}} = \frac{1}{1,8125} \approx 0,5517 \fs \uWandMinvsqKinv.
$$

Absolutní součinitel prostupu tepla $U_{\vartheta A,\Sigma}$ vypočteme jako:
$$
    U_{\vartheta A,\Sigma} = U_{\vartheta,\Sigma} \cdot S = 0,5517 \fs \uWandMinvsqKinv \cdot 20 \fs \uMsq = 11,034 \fs \uWandKinv.
$$

Měrný tepelný tok $\dot{q_x}$ vypočteme jako:
$$
    \dot{q_x} = \frac{\Delta T}{R_{\vartheta,\Sigma}} = \frac{T_1 - T_2}{R_{\vartheta,\Sigma}} = \frac{20 - (-10)}{1,8125} = \frac{30}{1,8125} \approx 16,55 \fs \uWandMinvsq.
$$

Tepelný tok $\dot{Q_x}$ vypočteme jako:
$$
    \dot{Q_x} = \dot{q_x} \cdot S = 16,55 \uWandMinvsq \cdot 20 \fs \uMsq = 331 \fs \uW = 0,331 \fs \uKW.
$$


\subsubsection{b}
Změnu teploty v cihle vypočteme jako:
$$
    \Delta T_{cihla} = \dot{q_x} \cdot R_{\vartheta,cihla} = 16,55 \fs \uWandMinvsq \cdot 0,5625 \fs \uMsqKandWinv = 9,31 \fs \uK.
$$

Změnu teploty v izolaci vypočteme jako:
$$
    \Delta T_{izol} = \dot{q_x} \cdot R_{\vartheta,izol} = 16,55 \fs \uWandMinvsq \cdot 1,25 \fs \uMsqKandWinv = 20,69 \fs \uK.
$$

Obrázek pro případ izolace z venčí bude vypadat následovně:

\begin{center}
    \begin{tikzpicture}
        % Variables
        \def\kk{0.18}
        \def\kkT{1.2}
        \def\dOne{45 * \kk} % d1
        \def\dTwo{5 * \kk} % d2
        \def\TOne{20 * \kk * \kkT} % T1
        \def\TTwo{-10 * \kk * \kkT} % T2
        \def\deltaTcihla{9.31 * \kk * \kkT} % delta T cihla
        \def\deltaTizol{20.69 * \kk * \kkT} % delta T izol

        % Axes
        \draw[->] (0,\TTwo - 1) -- (\dOne + \dTwo + 1,\TTwo - 1) node[anchor=north] {$x$}; % x-axis
        \draw[->] (0,\TTwo - 1) -- (0,\TOne * 1.1) node[anchor=east] {$T(x)$}; % y-axis

        % Wall lines
        \draw[thick] (0,\TOne) -- (0,\TTwo - 1);
        \draw[thick] (\dOne,\TOne) -- (\dOne,\TTwo - 1);
        \draw[thick] (\dOne + \dTwo,\TOne) -- (\dOne + \dTwo,\TTwo - 1);

        % Line representing T(x)
        \draw[thick, red] (-1,\TOne) -- (0,\TOne);
        \draw[thick, red] (0,\TOne) -- (\dOne,\TOne - \deltaTcihla);
        \draw[thick, red] (\dOne,\TOne - \deltaTcihla) -- (\dOne + \dTwo,\TTwo);
        \draw[thick, red] (\dOne + \dTwo,\TTwo) -- (\dOne + \dTwo + 1,\TTwo);

        % Dashed lines
        \draw[dashed] (0,\TOne) -- (2,\TOne) node[anchor=west] {$T_1 = 20 \fs \uCELS$};
        \draw[dashed] (\dOne,\TOne - \deltaTcihla) -- (5,\TOne - \deltaTcihla) node[anchor=east] {$T_2 - \Delta T_{cihla} = 10,69 \fs \uCELS$};
        \draw[dashed] (\dOne + \dTwo,\TTwo) -- (3,\TTwo) node[anchor=east] {$T_2 = -10 \fs \uCELS$};
        \draw[dashed] (\dOne + \dTwo, \TTwo - 1) -- (\dOne + \dTwo, \TTwo - 2) node[anchor=north] {$d_{cihla} + d_{izol}$};

        % % Labels
        \node[anchor=north] at (0, \TTwo - 1) {$0$};
        \node[anchor=north] at (\dOne, \TTwo - 1) {$d_{cihla}$};
    \end{tikzpicture}
\end{center}

Výhody toho položení je, že pokud je zeď zevnitř, tak funguje jako akumulátor tepla. Je to vhodné pro dlouhodobé vytápění. Nevýhodou je, že pokud se například jedná o chalupu, kam se jezdí pouze na víkend, tak nějákou dobu trvá, než se teplo naakumuluje a v místnoti bude teplo. Tento typ izolace se používá časteji.\\

Obrázek pro případ izolace zevnitř bude vypadat následovně:

\begin{center}
    \begin{tikzpicture}
        % Variables
        \def\kk{0.18}
        \def\kkT{1.2}
        \def\dOne{45 * \kk} % d1
        \def\dTwo{5 * \kk} % d2
        \def\TOne{20 * \kk * \kkT} % T1
        \def\TTwo{-10 * \kk * \kkT} % T2
        \def\deltaTcihla{9.31 * \kk * \kkT} % delta T cihla
        \def\deltaTizol{20.69 * \kk * \kkT} % delta T izol

        % Axes
        \draw[->] (0,\TTwo - 1) -- (\dOne + \dTwo + 1,\TTwo - 1) node[anchor=north] {$x$}; % x-axis
        \draw[->] (0,\TTwo - 1) -- (0,\TOne * 1.1) node[anchor=east] {$T(x)$}; % y-axis

        % Wall lines
        \draw[thick] (0,\TOne) -- (0,\TTwo - 1);
        \draw[thick] (\dTwo,\TOne) -- (\dTwo,\TTwo - 1);
        \draw[thick] (\dOne + \dTwo,\TOne) -- (\dOne + \dTwo,\TTwo - 1);

        % Line representing T(x)
        \draw[thick, red] (-1,\TOne) -- (0,\TOne);
        \draw[thick, red] (0,\TOne) -- (\dTwo,\TOne - \deltaTizol);
        \draw[thick, red] (\dTwo,\TOne - \deltaTizol) -- (\dOne + \dTwo,\TTwo);
        \draw[thick, red] (\dOne + \dTwo,\TTwo) -- (\dOne + \dTwo + 1,\TTwo);

        % Dashed lines
        \draw[dashed] (0,\TOne) -- (2,\TOne) node[anchor=west] {$T_1 = 20 \fs \uCELS$};
        \draw[dashed] (\dTwo,\TOne - \deltaTizol) -- (3,\TOne - \deltaTizol) node[anchor=west] {$T_2 - \Delta T_{izol} = -0,69 \fs \uCELS$};
        \draw[dashed] (\dOne + \dTwo,\TTwo) -- (4,\TTwo) node[anchor=east] {$T_2 = -10 \fs \uCELS$};
        \draw[dashed] (\dOne + \dTwo, \TTwo - 1) -- (\dOne + \dTwo, \TTwo - 2) node[anchor=north] {$d_{izol} + d_{cihla}$};

        % % Labels
        \node[anchor=north] at (0, \TTwo - 1) {$0$};
        \node[anchor=north] at (\dTwo, \TTwo - 1) {$d_{izol}$};
    \end{tikzpicture}
\end{center}

Toto položení se rychleji vytopí, ale také se rychleji ochladí, jelikož izolace nefunguje jako dobrý akumulátor tepla. Pokud například zasvítí slunce, tak se místnost rychleji zahřeje. Je zde riziko kondenzace a tvoření vlhkosti a plísní.

\newpage



\section{\emoji{scroll} PENB \spicy \spicy}
Mějme dům s plochou střechou o parametrech:

\begin{itemize}
    \item plocha střechy $S_{strecha} = 100 \fs \uMsq$,
    \item součinitel prostupu tepla střechy $U_{\vartheta,strecha} = 0,3 \fs \uWandMinvsqKinv$,
    \item plocha podlahy $S_{podlaha} = 100 \fs \uMsq$,
    \item měrný součinitel prostupu tepla podlahy $U_{\vartheta,podlaha} = 0,8 \fs \uWandMinvsqKinv$,
    \item plocha obvodových stěn $S_{zed} = 4 \cdot 10 \cdot 3 \fs \uMsq$ (4 stěny o výšce 2 metry a délce 10 metrů),
    \item obvodová stěna je tvořena cihlou s tepelnou vodivostí $\lambda_{cihla} = 0,8 \fs \uWandMinvKinv$ a tloušťkou $d_{cihla} = 45 \fs \uCM$.
\end{itemize}

Zanedbejte součinitele přestupu tepla, ztráty tepla na oknech a dveřích. Berte v úvahu, že venkovní teplota je okolo celého domu včetně podlahy stejná.

\begin{enumerate}[a)]
    \item Vypočítejte průměrný měrný součinitel prostupu tepla $U_{\vartheta,avg}$ pro celý dům.
    \item Vypočítejte jak silnou izolaci $d_{izol}$ s tepelnou vodivostí $\lambda_{izol} = 0,04 \fs \uWandMinvKinv$ musíte použít, aby průměrný součinitel prostupu tepla $U_{\vartheta,avg}^{*}$ byl $0,5 \fs \uWandMinvsqKinv$.
\end{enumerate}


\subsubsection{a}
Plochu obvodových stěn $S_{zed}$ vypočteme jako:
$$
    S_{zed} = 4 \cdot 10 \cdot 3 = 120 \fs \uMsq.
$$

Celkovou plochu $S_{\Sigma}$ vypočteme jako:
$$
    S_{\Sigma} = S_{strecha} + S_{podlaha} + S_{zed} = 100 + 100 + 120 = 320 \fs \uMsq.
$$

Součinitel prostupu tepla zdi $U_{\vartheta,cihla}$ vypočteme jako:
$$
    U_{\vartheta,cihla} = \frac{\lambda_{cihla}}{d_{cihla}} = \frac{0,8 \fs \uWandMinvKinv}{0,45 \fs \uM} = 1,778 \fs \uWandMinvsqKinv.
$$

Průměrný součinitel prostupu tepla $U_{\vartheta,avg}$ vypočteme jako:
$$
    U_{\vartheta,avg} = \frac{1}{S_{\Sigma}} \cdot (U_{\vartheta,strecha} \cdot S_{strecha} + U_{\vartheta,podlaha} \cdot S_{podlaha} + U_{\vartheta,cihla} \cdot S_{zed}) =
$$
$$
    = \frac{1}{320} \cdot \left( 0,3 \cdot 100 + 0,8 \cdot 100 + 1,778 \cdot 120 \right) = \frac{1}{320} \cdot (30 + 80 + 213,36) =
$$
$$
    \frac{323,36}{320} \approx 1,01 \fs \uWandMinvsqKinv.
$$


\subsubsection{b}
Součinitel prostupu tepla zdi s izolací $U_{\vartheta,zed}$ vypočteme jako:
$$
    U_{\vartheta,zed} = \left( \frac{d_{izol}}{\lambda_{izol}} + \frac{d_{cihla}}{\lambda_{cihla}} \right)^{-1}
$$

Průměrný součinitel prostupu tepla $U_{\vartheta,avg}^{*}$ vypočteme jako:
$$
    U_{\vartheta,avg}^{*} = \frac{1}{S_{\Sigma}} \cdot (U_{\vartheta,strecha} \cdot S_{strecha} + U_{\vartheta,podlaha} \cdot S_{podlaha} + U_{\vartheta,zed} \cdot S_{zed}) =
$$
$$
    = \frac{1}{S_{\Sigma}} \cdot (U_{\vartheta,strecha} \cdot S_{strecha} + U_{\vartheta,podlaha} \cdot S_{podlaha} + \left( \frac{d_{izol}}{\lambda_{izol}} + \frac{d_{cihla}}{\lambda_{cihla}} \right)^{-1} \cdot S_{zed})
$$

Nyní je třeba vyřešit rovnici pro $d_{izol}$:
$$
    U_{\vartheta,avg}^{*} = \frac{1}{S_{\Sigma}} \cdot (U_{\vartheta,strecha} \cdot S_{strecha} + U_{\vartheta,podlaha} \cdot S_{podlaha} + \left( \frac{d_{izol}}{\lambda_{izol}} + \frac{d_{cihla}}{\lambda_{cihla}} \right)^{-1} \cdot S_{zed})
$$
$$
    U_{\vartheta,avg}^{*} \cdot S_{\Sigma} =  (U_{\vartheta,strecha} \cdot S_{strecha} + U_{\vartheta,podlaha} \cdot S_{podlaha} + \left( \frac{d_{izol}}{\lambda_{izol}} + \frac{d_{cihla}}{\lambda_{cihla}} \right)^{-1} \cdot S_{zed})
$$
$$
    U_{\vartheta,avg}^{*} \cdot S_{\Sigma} - U_{\vartheta,strecha} \cdot S_{strecha} - U_{\vartheta,podlaha} \cdot S_{podlaha} = \left( \frac{d_{izol}}{\lambda_{izol}} + \frac{d_{cihla}}{\lambda_{cihla}} \right)^{-1} \cdot S_{zed}
$$
$$
    \left( \frac{d_{izol}}{\lambda_{izol}} + \frac{d_{cihla}}{\lambda_{cihla}} \right)^{-1} = \frac{U_{\vartheta,avg}^{*} \cdot S_{\Sigma} - U_{\vartheta,strecha} \cdot S_{strecha} - u_{podlaha} \cdot S_{podlaha}}{S_{zed}}
$$
$$
    \frac{d_{izol}}{\lambda_{izol}} + \frac{d_{cihla}}{\lambda_{cihla}} = \left( \frac{U_{\vartheta,avg}^{*} \cdot S_{\Sigma} - U_{\vartheta,strecha} \cdot S_{strecha} - U_{\vartheta,podlaha} \cdot S_{podlaha}}{S_{zed}} \right)^{-1}
$$
$$
    \frac{d_{izol}}{\lambda_{izol}} = \left( \frac{U_{\vartheta,avg}^{*} \cdot S_{\Sigma} - U_{\vartheta,strecha} \cdot S_{strecha} - U_{\vartheta,podlaha} \cdot S_{podlaha}}{S_{zed}} \right)^{-1} - \frac{d_{cihla}}{\lambda_{cihla}}
$$
$$
    d_{izol} = \lambda_{izol} \cdot \left( \left( \frac{U_{\vartheta,avg}^{*} \cdot S_{\Sigma} - U_{\vartheta,strecha} \cdot S_{strecha} - U_{\vartheta,podlaha} \cdot S_{podlaha}}{S_{zed}} \right)^{-1} - \frac{d_{cihla}}{\lambda_{cihla}} \right)
$$

Nyní můžeme dosadit a vypočítat:
$$
    d_{izol} = 0,04 \cdot \left( \left( \frac{0,5 \cdot 320 - 0,3 \cdot 100 - 0,8 \cdot 100}{120} \right)^{-1} - \frac{0,45}{0,8} \right) =
$$
$$
    = 0,04 \cdot \left( \left( \frac{160 - 30 - 80}{120} \right)^{-1} - \frac{0,45}{0,8} \right) =
$$
$$
    = 0,04 \cdot \left( \left( \frac{50}{120} \right)^{-1} - \frac{0,45}{0,8} \right) = 0,04 \cdot (2,4 - 0,5625) =
$$
$$
    0,04 \cdot 1,8375 = 0,0735 \fs \uM = 7,35 \fs \uCM.
$$

\newpage



\section{\emoji{snowflake} Topná sezóna \spicy \spicy \spicy}
Průměrná venkonví teplota v topné sezóně je $\overline{T}_{out} = 5 \fs \uCELS$. Vnitřní teplota je $T_{in} = 20 \fs \uCELS$. Doba topné sezóny je 200 dní. Celková plocha je $S = 300 \fs \uMsq$. Součinitel prostupu tepla je $U_\vartheta = 0,5 \fs \uWandMinvsqKinv$. Jaký množství tepla projde stěnami za celou topnou sezónu?

\subsection{Řešení}
Celkové množství tepla, které projde stěnami za celou topnou sezónu $Q$ vypočteme jako:
$$
    Q = \int_{t_1}^{t_2} \dot{Q} \cdot dt,
$$
kde:\\
$\dot{Q}$ je tepelný tok (\ueqW),\\
$t_1$ je začátek topné sezóny (\uH),\\
$t_2$ je konec topné sezóny (\uH).

Integrál můžeme rozepsat jako:
$$
    Q = \int_{t_1}^{t_2} U_\vartheta \cdot S \cdot (T_{in} - T_{out} (t)) \cdot dt = U_\vartheta \cdot S \cdot \int_{t_1}^{t_2} (T_{in} - T_{out} (t)) \cdot dt =
$$
$$
    = U_\vartheta \cdot S \cdot T_{in} \cdot \int_{t_1}^{t_2} dt - U_\vartheta \cdot S \cdot \int_{t_1}^{t_2} T_{out} (t) \cdot dt =
$$
$$
    = U_\vartheta \cdot S \cdot T_{in} \cdot (t_2 - t_1) - U_\vartheta \cdot S \cdot \int_{t_1}^{t_2} T_{out} (t) \cdot dt.
$$

Nyní uděláme odbočku, kde vyjádříme průměrnou venkovní teplotu $\overline{T}_{out}$ jako:
$$
    \overline{T}_{out} = \frac{1}{t_2 - t_1} \cdot \int_{t_1}^{t_2} T_{out} (t) \cdot dt.
$$

Vyjádříme daný integrál:
$$
    \int_{t_1}^{t_2} T_{out} (t) \cdot dt = \overline{T}_{out} \cdot (t_2 - t_1).
$$

Dosadíme zpět do původní rovnice:
$$
    Q = U_\vartheta \cdot S \cdot T_{in} \cdot (t_2 - t_1) - U_\vartheta \cdot S \cdot \overline{T}_{out} \cdot (t_2 - t_1) =
$$
$$
    = U_\vartheta \cdot S \cdot (T_{in} - \overline{T}_{out}) \cdot (t_2 - t_1).
$$

Nyní můžeme dosadit a vypočítat:
$$
    Q = 0,5 \cdot 300 \cdot (20 - 5) \cdot (200 \cdot 24 \cdot 3 \fs 600 - 0) = 0,5 \cdot 300 \cdot 15 \cdot 200 \cdot 24 \cdot 3 \fs 600 =
$$
$$
    = 38,9 \fs \uGJ.
$$

\newpage



\section{\emoji{factory} Cihlová pec \spicy \spicy \spicy \spicy}
U cihlové pece je vysoký rozdíl teplot, tudíž tepelnou vodivost je třeba uvažovat jako proměnnou. Tepelnou vodivost $\lambda$ můžeme zapsat jako:
$$
    \lambda (x) = \lambda_0 + \lambda_1 \cdot T,
$$

Pro zeď o tloušťce $d$ budeme uvažovat následující zjednodušející předpoklady:
\begin{itemize}
    \item $\frac{\partial T}{\partial t}$ = 0,
    \item $\vec{v}$ = 0,
    \item $Q_v$ = 0,
    \item $\lambda (x) = \lambda_0 + \lambda_1 \cdot T$,
    \item $T = T(x)$ - teplota závislá pouze na ose x.
\end{itemize}

Poté Fourierova-Kirchhoffova rovnice bude mít tvar:
$$
    0 = \frac{d}{dx} \left ( \lambda (T) \cdot \frac{dT}{dx} \right ).
$$

Fourieruv zákon pro tepelný tok bude mít tvar:
$$
    \dot{q_x} = - \lambda (T) \cdot \frac{dT}{dx}.
$$

Vynásobme rovnici mínus jedna:
$$
    - \dot{q_x} = \lambda (T) \cdot \frac{dT}{dx}.
$$

Pokud dosadíme do Fourierova-Kirchhoffovy rovnice, dostaneme:
$$
    0 = \frac{d}{dx} \left ( - \dot{q_x} \right ).
$$

Z toho vidíme, že měrný tepelný tok $\dot{q_x}$ je konstantní a nezávisí na poloze x. Můžeme tedy rozepsat Fourierův zákon jako:
$$
    - \dot{q_x} = \left( \lambda_0 + \lambda_1 \cdot T \right) \cdot \frac{dT}{dx}.
$$

Můžeme řešit tuto diferenciální rovnici separací proměnných:
$$
    - \dot{q_x} \cdot dx = \left( \lambda_0 + \lambda_1 \cdot T \right) \cdot dT.
$$

Nyní meze budou jak pro x, tak pro T:
\begin{itemize}
    \item $x = 0 \Rightarrow T = T_1$,
    \item $x = d \Rightarrow T = T_2$.
\end{itemize}

Rovnici můžeme integrovat:
$$
    - \int_{0}^{d} \dot{q_x} \cdot dx = \int_{T_1}^{T_2} \left( \lambda_0 + \lambda_1 \cdot T \right) \cdot dT
$$
$$
    - \dot{q_x} \left[ x \right]_{0}^{d} = \lambda_0 \left[ T \right]_{T_1}^{T_2} + \frac{\lambda_1}{2} \left[ T^2 \right]_{T_1}^{T_2}
$$
$$
    - \dot{q_x} \cdot d = \lambda_0 \cdot (T_2 - T_1) + \frac{\lambda_1}{2} \cdot (T_2^2 - T_1^2)
$$
$$
    - \dot{q_x} \cdot d = \lambda_0 \cdot (T_2 - T_1) + \frac{\lambda_1}{2} \cdot (T_2 - T_1) \cdot (T_2 + T_1)
$$
$$
    - \dot{q_x} \cdot d = (T_2 - T_1) \cdot \left( \lambda_0 + \frac{\lambda_1}{2} \cdot (T_2 + T_1) \right).
$$

Nyní odvoďme střední hodnotu tepelné vodivosti $\overline{\lambda}$:
$$
    \overline{\lambda} = \frac{1}{T_2 - T_1} \cdot \int_{T_1}^{T_2} \lambda (T) \cdot dT = \frac{1}{T_2 - T_1} \cdot \int_{T_1}^{T_2} \left( \lambda_0 + \lambda_1 \cdot T \right) \cdot dT
$$
$$
    = \frac{1}{T_2 - T_1} \cdot \left[ \lambda_0 \cdot T + \frac{\lambda_1}{2} \cdot T^2 \right]_{T_1}^{T_2} = \frac{1}{T_2 - T_1} \cdot \left( \lambda_0 \cdot (T_2 - T_1) + \frac{\lambda_1}{2} \cdot (T_2^2 - T_1^2) \right) =
$$
$$
    = \lambda_0 + \frac{\lambda_1}{2} \cdot (T_2 + T_1).
$$

Nyní můžeme videt, že člen v závorce u vyřešené diferenciální rovnice je roven střední hodnotě tepelné vodivosti $\overline{\lambda}$:
$$
    - \dot{q_x} \cdot d = (T_2 - T_1) \cdot \overline{\lambda}.
$$

Nyní můžeme odvodit vztah pro měrný tepelný tok $\dot{q_x}$:
$$
    \dot{q_x} = \frac{T_1 - T_2}{d} \cdot \overline{\lambda}.
$$

\newpage



\section{\emoji{sun} Sálavá clona \spicy \spicy}
Mějme dvě desky. První deska má teplotu $T_1$ a druhá deska má teplotu $T_2$. Obě desky mají stejnou plochu $S$ a stejnou emisivitu $\varepsilon$.

\begin{enumerate}[a)]
    \item Jaký je sálavý výkon $P_{1 \rightarrow 2} \fs (\uW)$ sálání první desky na druhou desku?
    \item Dejme mezi desky sálavou clonu o stejné ploše $S$ a emisivitě $\varepsilon$. Jaký je sálavý výkon $P_{1 \rightarrow 2} \fs (\uW)$ sálání první desky na druhou desku přes clonu a jaký je poměr tohoto výkonu k výkonu z bodu a)?
\end{enumerate}


\subsection{a}
Sálavý výkon $P_{1 \rightarrow 2}$ můžeme vypočítat pomocí:
$$
    P_{1 \rightarrow 2} = \frac{S \cdot \sigma \left( T_1^4 - T_2^4 \right)}{\frac{1}{\varepsilon_1} + \frac{1}{\varepsilon_2} - 1},
$$
kde:\\
$S$ je plocha desek (\ueqMsq),\\
$\sigma$ je Stefanova-Boltzmannova konstanta ($5,67 \cdot 10^{-8} \fs \uWandMinvsqKinv$),\\
$T_1$ je teplota první desky (\uK),\\
$T_2$ je teplota druhé desky (\uK),\\
$\varepsilon_1$ je emisivita první desky,\\
$\varepsilon_2$ je emisivita druhé desky.\\

Emisivity $\varepsilon_1$ a $\varepsilon_2$ jsou stejné, tudíž:
$$
    P_{1 \rightarrow 2} = \frac{S \cdot \sigma \left( T_1^4 - T_2^4 \right)}{\frac{1}{\varepsilon} + \frac{1}{\varepsilon} - 1} = \frac{S \cdot \sigma \left( T_1^4 - T_2^4 \right)}{\frac{2}{\varepsilon} - 1} = k \cdot \left( T_1^4 - T_2^4 \right).
$$


\subsection{b}
Sálavý výkon z destičky 1 na clonu:
$$
    P_{1 \rightarrow clona} = k \cdot \left( T_1^4 - T_{clona}^4 \right).
$$

Sálavý výkon z clony na destičku 2:
$$
    P_{clona \rightarrow 2} = k \cdot \left( T_{clona}^4 - T_2^4 \right).
$$

Tyto výkony se musí rovnat:
$$
    P_{1 \rightarrow clona} = P_{clona \rightarrow 2}
$$
$$
    k \cdot \left( T_1^4 - T_{clona}^4 \right) = k \cdot \left( T_{clona}^4 - T_2^4 \right)
$$
$$
    T_1^4 - T_{clona}^4 = T_{clona}^4 - T_2^4
$$
$$
    2 \cdot T_{clona}^4 = T_1^4 + T_2^4
$$
$$
    T_{clona}^4 = \frac{T_1^4 + T_2^4}{2}
$$
$$
    T_{clona} = \sqrt[4]{\frac{T_1^4 + T_2^4}{2}}.
$$

Sálavý výkon z destičky 1 na destičku 2 přes clonu je poté roven sálavému výkonu z clony na destičku 2, který je zároveň roven sálavému výkonu z destičky 1 na clonu:
$$
    P_{1 \rightarrow 2,b} = P_{1 \rightarrow clona} = P_{clona \rightarrow 2}.
$$

Ověřme výpočet pro obě možnosti. Sálavý výkon z destičky 1 na clonu:
$$
    P_{1 \rightarrow clona} = k \cdot \left( T_1^4 - T_{clona}^4 \right) = k \cdot \left( T_1^4 - \frac{T_1^4 + T_2^4}{2} \right) =
$$
$$
    = k \cdot \frac{2 \cdot T_1^4 - T_1^4 - T_2^4}{2} = k \cdot \frac{T_1^4 - T_2^4}{2} = \frac{k}{2} \cdot \left( T_1^4 - T_2^4 \right).
$$

Sálavý výkon z clony na destičku 2:
$$
    P_{clona \rightarrow 2} = k \cdot \left( T_{clona}^4 - T_2^4 \right) = k \cdot \left( \frac{T_1^4 + T_2^4}{2} - T_2^4 \right) =
$$
$$
    = k \cdot \frac{T_1^4 + T_2^4 - 2 \cdot T_2^4}{2} = k \cdot \frac{T_1^4 - T_2^4}{2} = \frac{k}{2} \cdot \left( T_1^4 - T_2^4 \right).
$$

Výkony jsou si rovny, tudíž sálavý výkon z destičky 1 na destičku 2 přes clonu je roven:
$$
    P_{1 \rightarrow 2,b} = \frac{k}{2} \cdot \left( T_1^4 - T_2^4 \right).
$$

Poměr tohoto výkonu k výkonu z bodu a):
$$
    \frac{P_{1 \rightarrow 2,b}}{P_{1 \rightarrow 2,a}} = \frac{\frac{k}{2} \cdot \left( T_1^4 - T_2^4 \right)}{k \cdot \left( T_1^4 - T_2^4 \right)} = \frac{1}{2}.
$$

\newpage



\section{\emoji{milky-way} Destička ve vesmíru \spicy \spicy}
Mějme destičku ve vesmíru, na kterou dopadá sluneční záření s následujícími parametry:
\begin{itemize}
    \item plocha $S = 1 \fs \uMsq$,
    \item intenzita slunečního záření $\dot{q} = 1348 \fs \uWandMinvsq$,
    \item emisivita $\varepsilon$,
    \item pohltivost $A$.
\end{itemize}

Destička má stejnou teplotu na obouch stranách (je nekonečně tenká). Sluneční záření má intenzitu $\dot{q} = 1348 \fs \uWandMinvsq$. Stefanova-Boltzmannova konstanta je $\sigma = 5,67 \cdot 10^{-8} \fs \uWandMinvsqKinvquar$. Teplota vesmíru je $T_{space} = 3 \fs \uK$.\\

Odvoďte vzorec pro rovnovážnou teplotu destičky $T$ v závislosti na úhlu natočení vůči slunci $\alpha$ a vypočítejte teplotu pro hodnoty $\alpha$: 0, 45 a 90 \ueqDEGREE.

\subsubsection{Řešení}
Tepelný to $\dot{Q}_{dop} \fs (\uW)$, který na destičku dopadne:
$$
    \dot{Q}_{dop} = \dot{q} \cdot S \cdot \cos \alpha
$$

Tepelný tok, který destička příjme:
$$
    \dot{Q}_{in} = A \cdot \dot{Q}_{dop} = A \cdot \dot{q} \cdot S \cdot \cos \alpha
$$

Tepelný tok, který destička vyzařuje:
$$
    \dot{Q}_{out} = 2 \cdot S \cdot \varepsilon \cdot \sigma \cdot \left( T^4 - T_{space}^4 \right)
$$

Budeme hledat teplotu $T$, při které bude tepelný tok $\dot{Q}_{in}$ roven tepelnému toku $\dot{Q}_{out}$:
$$
    \dot{Q}_{in} = \dot{Q}_{out}
$$
$$
    A \cdot \dot{q} \cdot S \cdot \cos \alpha = 2 \cdot S \cdot \varepsilon \cdot \sigma \cdot \left( T^4 - T_{space}^4 \right)
$$

Při tepelné rovnováze platí:
$$
    A = \varepsilon.
$$

Dostaneme:
$$
    \dot{q} \cdot \cos \alpha = 2 \cdot \sigma \cdot \left( T^4 - T_{space}^4 \right)
$$
$$
    \frac{\dot{q} \cdot \cos \alpha}{2 \cdot \sigma} = T^4 - T_{space}^4
$$
$$
    T^4 = \frac{\dot{q} \cdot \cos \alpha}{2 \cdot \sigma} + T_{space}^4
$$
$$
    T = \sqrt[4]{\frac{\dot{q} \cdot \cos \alpha}{2 \cdot \sigma} + T_{space}^4}
$$

Nyní můžeme dosadit do vzorce a získat vzorec pro teplotu $T$ v závislosti na úhlu natočení $\alpha$:
$$
    T = \sqrt[4]{\frac{1348 \cdot \cos \alpha}{2 \cdot 5,67 \cdot 10^{-8}} + 3^4} = \sqrt[4]{1,189 \cdot 10^{10} \cdot \cos \alpha + 81}.
$$

Nyní můžeme vypočítat teplotu pro hodnoty $\alpha$: 0, 45 a 90 \ueqDEGREE. Pro $\alpha = 0$:
$$
    T = \sqrt[4]{1,189 \cdot 10^{10} \cdot \cos 0 \uDEGREE + 81} = \sqrt[4]{1,189 \cdot 10^{10} + 81} = 330,21 \fs \uK.
$$

Pro $\alpha = 45$:
$$
    T = \sqrt[4]{1,189 \cdot 10^{10} \cdot \cos 45 \uDEGREE + 81} = \sqrt[4]{1,189 \cdot 10^{10} \cdot \frac{\sqrt{2}}{2} + 81} = 302,79 \fs \uK.
$$

Pro $\alpha = 90$:
$$
    T = \sqrt[4]{1,189 \cdot 10^{10} \cdot \cos 90 \uDEGREE + 81} = \sqrt[4]{1,189 \cdot 10^{10} \cdot 0 + 81} = \sqrt[4]{81} = 3 \fs \uK.
$$

\end{document}
