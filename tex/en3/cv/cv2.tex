\documentclass{article}
\usepackage[czech]{babel} % Czech language
\usepackage[shortlabels]{enumitem} % Custom enumeration
\usepackage{graphicx} % Import images
\usepackage{float} % Use [H] to force figure position
\usepackage{indentfirst} % Indent first paragraph
\usepackage{emoji} % Emojis
\usepackage{pgffor} % Loops
\usepackage{tikz} % TikZ
\usepackage{pgfplots} % TikZ plots
\usepackage{circuitikz} % Use circuitikz for circuit diagrams
\usepackage{amsmath} % Math
\usetikzlibrary{arrows.meta}

\makeatletter
\providecommand\add@text{}
\newcommand\tagaddtext[1]{%
    \gdef\add@text{#1\gdef\add@text{}}}% 
\renewcommand\tagform@[1]{%
    \maketag@@@{\llap{\add@text\quad}(\ignorespaces#1\unskip\@@italiccorr)}%
}
\makeatother

\newcommand{\cvHead}[1]{\head{Cvičení #1}}

\newcommand{\head}[1]{
    \title{\textbf{#1}\\Elektroenergetika 3}
    \author{Petr Jílek}
    \date{2024}
}

\newcommand{\spicy}{\emoji{hot-pepper}}

% My font space
\newcommand{\myFS}{\;}

% Text in math mode
\newcommand{\te}[1]{\textrm{#1}}

% --------------------
% Units
% --------------------

\newcommand{\uM}{\textrm{m}} % Meter
\newcommand{\uMsq}{\uM^\textrm{2}} % Meter squared
\newcommand{\uMcu}{\uM^\textrm{3}} % Meter cubed
\newcommand{\uS}{\textrm{s}} % Second
\newcommand{\uKG}{\textrm{kg}} % Kilogram
\newcommand{\uJ}{\textrm{J}} % Joule
\newcommand{\uK}{\textrm{K}} % Kelvin
\newcommand{\uDEGREE}{^\circ} % Degree
\newcommand{\uCELS}{\uDEGREE \textrm{C}} % Celsius
\newcommand{\uW}{\textrm{W}} % Watt
\newcommand{\uMW}{\textrm{MW}} % Mega Watt
\newcommand{\uKW}{\textrm{kW}} % Kilo Watt
\newcommand{\uWH}{\textrm{Wh}} % Watt hour
\newcommand{\uKWH}{\textrm{kWh}} % Kilo Watt hour
\newcommand{\uCCY}{\textrm{CCY}} % Currency
\newcommand{\uCZK}{\textrm{CZK}} % Czech Crown
\newcommand{\uPERCENT}{\textrm{\%}} % Percent
\newcommand{\uNOUNIT}{\textrm{--}} % No unit
\newcommand{\uYEAR}{\textrm{year}} % Year
\newcommand{\uMONTH}{\textrm{month}} % Month
\newcommand{\uHOUR}{\textrm{hour}} % Hour
\newcommand{\uLM}{\textrm{lm}} % Lumen
\newcommand{\uHinv}{\textrm{h}^{-1}} % Hour inverse

\newcommand{\uJperK}{\uJ / \uK} % Joule per Kelvin
\newcommand{\uKperM}{\uK / \uM} % Kelvin per Meter
\newcommand{\uKperS}{\uK / \uS} % Kelvin per Second
\newcommand{\uMsqperS}{\uMsq/\uS} % Meter squared per Second
\newcommand{\uWperMsq}{\uW / \uMsq} % Watt per Meter squared
\newcommand{\uWperMperK}{\uW / \left( \uM \cdot \uK \right)} % Watt per Meter per Kelvin
\newcommand{\uJperKGperK}{\uJ / \left( \uKG \cdot \uK \right)} % Joule per Kilogram per Kelvin
\newcommand{\uKperMsq}{\uK / \uMsq} % Kelvin per Meter squared
\newcommand{\uJperSperMperK}{\uJ / \left( \uS \cdot \uM \cdot \uK \right)} % Joule per Second per Meter per Kelvin
\newcommand{\uKGperMcu}{\uKG / \uMcu} % Kilogram per Meter cubed
\newcommand{\uMsqKperW}{\uMsq \cdot \uK / \uW} % Meter squared Kelvin per Watt
\newcommand{\uWperMsqperK}{\uW / \left( \uMsq \cdot \uK \right)} % Watt per Meter squared per Kelvin
\newcommand{\uKWHperMsqperYEAR}{\uKWH / \left( \uMsq \cdot \uYEAR \right)} % Kilo Watt hour per Meter squared per Year
\newcommand{\uKWHperMsq}{\uKWH / \uMsq} % Kilo Watt hour per Meter squared
\newcommand{\uLMperW}{\uLM / \uW} % Lumen per Watt
\newcommand{\uKWperMsq}{\uKW / \uMsq} % Kilo Watt per Meter squared
\newcommand{\uCCYperYEAR}{\uCCY / \uYEAR} % Currency per Year
\newcommand{\uKWHperYEAR}{\uKWH / \uYEAR} % Kilo Watt hour per Year
\newcommand{\uNOUNITperYEAR}{\uNOUNIT / \uYEAR} % No unit per Year
\newcommand{\uCCYperKWH}{\uCCY / \uKWH} % Currency per Kilo Watt hour
\newcommand{\uCZKperYEAR}{\uCZK / \uYEAR} % Czech Crown per Year
\newcommand{\uCZKperKWH}{\uCZK / \uKWH} % Czech Crown per Kilo Watt hour
\newcommand{\uCZKperMONTH}{\uCZK / \uMONTH} % Czech Crown per Month



% --------------------
% Unit equations
% --------------------

\newcommand{\ueqM}{$\uM$}
\newcommand{\ueqMsq}{$\uMsq$}
\newcommand{\ueqMcu}{$\uMcu$}
\newcommand{\ueqS}{$\uS$}
\newcommand{\ueqJ}{$\uJ$}
\newcommand{\ueqK}{$\uK$}
\newcommand{\ueqDEGREE}{$\uDEGREE$}
\newcommand{\ueqCELS}{$\uCELS$}
\newcommand{\ueqW}{$\uW$}
\newcommand{\ueqMW}{$\uMW$}
\newcommand{\ueqKW}{$\uKW$}
\newcommand{\ueqWH}{$\uWH$}
\newcommand{\ueqKWH}{$\uKWH$}
\newcommand{\ueqCCY}{$\uCCY$}
\newcommand{\ueqCZK}{$\uCZK$}
\newcommand{\ueqPERCENT}{$\uPERCENT$}
\newcommand{\ueqNOUNIT}{$\uNOUNIT$}
\newcommand{\ueqYEAR}{$\uYEAR$}
\newcommand{\ueqMONTH}{$\uMONTH$}
\newcommand{\ueqHOUR}{$\uHOUR$}
\newcommand{\ueqLM}{$\uLM$}
\newcommand{\ueqHinv}{$\uHinv$}

\newcommand{\ueqJperK}{$\uJperK$}
\newcommand{\ueqKperM}{$\uKperM$}
\newcommand{\ueqKperS}{$\uKperS$}
\newcommand{\ueqMsqperS}{$\uMsqperS$}
\newcommand{\ueqWperMsq}{$\uWperMsq$}
\newcommand{\ueqWperMperK}{$\uWperMperK$}
\newcommand{\ueqJperKGperK}{$\uJperKGperK$}
\newcommand{\ueqKperMsq}{$\uKperMsq$}
\newcommand{\ueqJperSperMperK}{$\uJperSperMperK$}
\newcommand{\ueqKGperMcu}{$\uKGperMcu$}
\newcommand{\ueqMsqKperW}{$\uMsqKperW$}
\newcommand{\ueqWperMsqperK}{$\uWperMsqperK$}
\newcommand{\ueqKWHperMsqperYEAR}{$\uKWHperMsqperYEAR$}
\newcommand{\ueqKWHperMsq}{$\uKWHperMsq$}
\newcommand{\ueqLMperW}{$\uLMperW$}
\newcommand{\ueqKWperMsq}{$\uKWperMsq$}
\newcommand{\ueqCCYperYEAR}{$\uCCYperYEAR$}
\newcommand{\ueqKWHperYEAR}{$\uKWHperYEAR$}
\newcommand{\ueqNOUNITperYEAR}{$\uNOUNITperYEAR$}
\newcommand{\ueqCCYperKWH}{$\uCCYperKWH$}
\newcommand{\ueqCZKperYEAR}{$\uCZKperYEAR$}
\newcommand{\ueqCZKperKWH}{$\uCZKperKWH$}
\newcommand{\ueqCZKperMONTH}{$\uCZKperMONTH$}


\cvHead{2 - Sdílení tepla}

\begin{document}

\maketitle
\tableofcontents
\newpage



\section{\emoji{brick} Zeď \spicy \spicy}


\subsection{Fourierova-Kirchhoffova rovnice \spicy \spicy \spicy}
Pro zeď o tloušťce $d$ budeme uvažovat následující zjednodušející předpoklady:
\begin{itemize}
    \item $\lambda$ = konstanta,
    \item $\vec{v}$ = 0,
    \item $Q_v$ = 0,
    \item $\frac{\partial T}{\partial t}$ = 0,
    \item T = T(x) - teplota závislá pouze na ose x.
\end{itemize}

Poté Fourierova-Kirchhoffova rovnice bude mít tvar:
$$
    0 = \nabla \cdot \left( \lambda \cdot \vec{\nabla} T \right) = \lambda \cdot \frac{d^2 T}{d x^2}.
$$

Rovnici můžeme vydělit $\lambda$ ($\lambda > 0$) a získáme:
$$
    \frac{d^2 T}{d x^2} = 0.
$$

Tuto rovnici můžeme řešit dvojí integrací. První integrace bude vypadat následovně:
$$
    \frac{d T}{d x} = \int 0 \cdot dx = 0 + c_1 = c_1.
$$

Druhá integrace bude vypadat následovně:
$$
    T(x) = \int c_1 \cdot dx = c_1 \cdot x + c_2.
$$

Obecné řešení bude tedy:
$$
    T(x) = c_1 \cdot x + c_2.
$$

S okrajovými podmínkami:
\begin{itemize}
    \item $T(0) = T_1$:
          $$
              T_1 = T(0) = c_1 \cdot 0 + c_2 = c_2 \Rightarrow c_2 = T_1.
          $$
    \item $T(d) = T_2$:
          $$
              T_2 = T(d) = c_1 \cdot d + c_2 \Rightarrow c_1 = \frac{T_2 - c_2}{d} = \frac{T_2 - T_1}{d}.
          $$
\end{itemize}

Řešení je tedy:
$$
    T(x) = \frac{T_2 - T_1}{d} \cdot x + T_1.
$$

Tuto situaci můžeme znázornit následujícím obrázkem:

\begin{center}
    \begin{tikzpicture}
        % Variables
        \def\TOne{4} % T1
        \def\TTwo{1} % T2
        \def\d{5} % width of the wall, d

        % Axes
        \draw[->] (0,0) -- (\d + 1,0) node[anchor=north] {$x$}; % x-axis
        \draw[->] (0,0) -- (0,\TOne + 1) node[anchor=east] {$T(x)$}; % y-axis

        % Line representing T(x)
        \draw[thick, red] (-2,\TOne) -- (0,\TOne);
        \draw[thick, red] (0,\TOne) -- (\d,\TTwo);
        \draw[thick, red] (\d,\TTwo) -- (\d + 2,\TTwo);

        % Dotted line
        \draw[dashed] (0,\TOne) -- (1,\TOne);
        \draw[dashed] (0,\TTwo) -- (\d,\TTwo);

        % Wall lines
        \draw[very thick] (0,0) -- (0,\TOne + 0.5);
        \draw[very thick] (\d,0) -- (\d,\TOne - 1.5);
        \draw[very thick] (\d,\TOne - 0.5) -- (\d,\TOne + 0.5);

        % Labels
        \node[anchor=south] at (4, 2.7) {$T(x) = \frac{T_2 - T_1}{d} \cdot x + T_1$};
        \node[anchor=west] at (1, \TOne) {$T_1$};
        \node[anchor=east] at (0, \TTwo) {$T_2$};
        \node[anchor=north] at (\d, 0) {$d$};
    \end{tikzpicture}
\end{center}


\subsection{Fourieruv zákon \spicy \spicy \spicy}
Nyní dosadíme řešení z předchozího příkladu do Fourierova zákona, kde za gradient teploty dosadíme derivaci teploty podle osy x:
$$
    \dot{q_x} = - \lambda  \frac{dT(x)}{dx} = - \lambda  \frac{d}{dx} \left( \frac{T_2 - T_1}{d} \cdot x + T_1 \right) = - \lambda  \frac{T_2 - T_1}{d} = \lambda  \frac{T_1 - T_2}{d} = \frac{\Delta T}{\frac{d}{\lambda}}.
$$


\subsection{Tepelný odpor}
Dolní výraz z předchozí sekce $\frac{d}{\lambda}$ je měrný tepelný odpor $r_{\vartheta}$ s jednotkou \ueqMsqKandWinv. Je tedy definován jako:
$$
    r_{\vartheta} = \frac{d}{\lambda}.
$$

Měrný tepelný tok $q_x$ má jednotku \ueqWandMinvsq. Tepelný tok $\dot{Q_x}$ s jednotkou \ueqW dostaneme vynásobením měrného tepelného toku plochou průřezu $S$:
$$
    \dot{Q_x} = \dot{q_x} \cdot S = \lambda \frac{T_1 - T_2}{d} \cdot S = \lambda \frac{\Delta T}{d} \cdot S = \frac{\Delta T}{\frac{d}{\lambda \cdot S}}.
$$

Dolní výraz $\frac{d}{\lambda \cdot S}$ je tepelný odpor $R_{\vartheta}$ s jednotkou \ueqKandWinv. Je tedy definován jako:
$$
    R_{\vartheta} = \frac{d}{\lambda \cdot S}.
$$

Analogie mezi měrným tepelným odporem a tepelným odporem je:
$$
    R_{\vartheta} = \frac{r_{\vartheta}}{S}.
$$

\textit{Poznámka}\\

Výpočet tepelného odporu je analogický jako výpočet elektrického odporu. Elektrický odpor $R_e$ má jednotku \ueqOHM a je definován jako:
$$
    R_e = \frac{d}{\sigma_e \cdot S},
$$
kde:\\
$d$ -- délka vodiče (m),\\
$\sigma_e$ -- elektrická vodivost (\ueqMandOHMinv),\\
$S$ -- průřez vodiče (m2).\\

Analogie jsou:
\begin{itemize}
    \item $d$ -- tloušťka stěny (\ueqM) $\rightarrow$ $d$ -- délka vodiče (m),
    \item $\lambda$ -- tepelná vodivost (\ueqWandMinvsqKinv) $\rightarrow$ $\sigma_e$ -- elektrická vodivost (\ueqMandOHMinv),
    \item $S$ -- plocha průřezu (\ueqMsq) $\rightarrow$ $S$ -- průřez vodiče (m2).
\end{itemize}

Výpočet tepelného toku je analogický s Ohmovým zákonem:
$$
    I = \frac{U}{R},
$$
kde:\\
$I$ -- proud (\ueqA),\\
$U$ -- napětí (\ueqV),\\
$R$ -- odpor (\ueqOHM).\\

Analogie jsou:
\begin{itemize}
    \item $\dot{Q_x}$ -- tepelný tok (\ueqW) $\rightarrow$ $I$ -- proud (\ueqA),
    \item $\Delta T$ -- rozdíl teplot (\ueqK) $\rightarrow$ $U$ -- napětí (\ueqV),
    \item $R_{\vartheta}$ -- tepelný odpor (\ueqKandWinv) $\rightarrow$ $R$ -- elektrický odpor (\ueqOHM).
\end{itemize}


\subsection{Součinitel prostupu tepla}
Měrný součinitel prostupu tepla $u_{\vartheta}$ má jednotku \ueqWandMinvsqKinv. Je definován jako inverzní hodnota měrného tepelného odporu:
$$
    u_{\vartheta} = \frac{1}{r_{\vartheta}} = \frac{\lambda}{d}.
$$

Měrný tepelný tok $q_x$ se poté může zapsat jako:
$$
    \dot{q_x} = u_{\vartheta} \cdot \Delta T.
$$

Součinitel prostupu tepla $U_{\vartheta}$ má jednotku \ueqKandWinv. Je definován jako inverzní hodnota tepelného odporu:
$$
    U_{\vartheta} = \frac{1}{R_{\vartheta}} = \frac{\lambda \cdot S}{d}.
$$

Tepelný tok $\dot{Q_x}$ se poté může zapsat jako:
$$
    \dot{Q_x} = U_{\vartheta} \cdot \Delta T.
$$

Analogie mezi měrným součinitelem prostupu tepla a součinitelem prostupu tepla je:
$$
    U_{\vartheta} = u_{\vartheta} \cdot S.
$$


\subsection{Číselný příklad}

\newpage



\section{\emoji{building-construction} Skládaná zeď \spicy \spicy}


\newpage



\section{\emoji{scroll} PENB \spicy \spicy}

\newpage



\section{\emoji{snowflake} Topná sezóna \spicy \spicy \spicy}

\newpage



\section{\emoji{factory} Cihlová pec \spicy \spicy \spicy \spicy}

\newpage



\section{\emoji{sun} Sálání (radiace) \spicy \spicy}


\subsection{Sálavá clona}


\subsection{Destička ve vesmíru}


\end{document}
