\documentclass{article}
\usepackage[czech]{babel} % Czech language
\usepackage[shortlabels]{enumitem} % Custom enumeration
\usepackage{graphicx} % Import images
\usepackage{float} % Use [H] to force figure position
\usepackage{indentfirst} % Indent first paragraph
\usepackage{emoji} % Emojis
\usepackage{pgffor} % Loops
\usepackage{tikz} % TikZ
\usepackage{pgfplots} % TikZ plots
\usepackage{amsmath} % Math
\usetikzlibrary{arrows.meta}

\makeatletter
\providecommand\add@text{}
\newcommand\tagaddtext[1]{%
    \gdef\add@text{#1\gdef\add@text{}}}% 
\renewcommand\tagform@[1]{%
    \maketag@@@{\llap{\add@text\quad}(\ignorespaces#1\unskip\@@italiccorr)}%
}
\makeatother

\newcommand{\cvHead}[1]{\head{Cvičení #1}}

\newcommand{\head}[1]{
    \title{\textbf{#1}\\Elektroenergetika 3}
    \author{Petr Jílek}
    \date{2024}
}

\newcommand{\spicy}{\emoji{hot-pepper}}

% --------------------
% --------------------
% Units
% --------------------
% --------------------

% --------------------
% Basic units
% --------------------  

% no unit
\newcommand{\uNOUNIT}{\te{--}} % no unit
\newcommand{\uPERCENT}{\te{\%}} % percent
\newcommand{\uDEGREE}{^\circ} % degree
\newcommand{\uRAD}{\te{rad}} % radian

% meter
\newcommand{\uM}{\te{m}} % meter
\newcommand{\uMsq}{\uM^\te{2}} % meter squared
\newcommand{\uMcu}{\uM^\te{3}} % meter cubed
\newcommand{\uMquar}{\uM^\te{4}} % meter quartic
\newcommand{\uMinv}{\uM^{-1}} % meter inverse
\newcommand{\uMinvsq}{\uM^{-2}} % meter inverse squared
\newcommand{\uMinvcu}{\uM^{-3}} % meter inverse cubed
\newcommand{\uMinvquar}{\uM^{-4}} % meter inverse quartic
\newcommand{\uMM}{\te{mm}} % millimeter
\newcommand{\uCM}{\te{cm}} % centimeter
\newcommand{\uKM}{\te{km}} % kilometer

% liter
\newcommand{\uL}{\te{l}} % liter
\newcommand{\uML}{\te{ml}} % milliliter

% second
\newcommand{\uS}{\te{s}} % second
\newcommand{\uSinv}{\uS^{-1}} % second inverse
\newcommand{\uSinvsq}{\uS^{-2}} % second inverse squared

% hour
\newcommand{\uH}{\te{h}} % hour
\newcommand{\uHinv}{\te{h}^{-1}} % hour inverse

% kilogram
\newcommand{\uKG}{\te{kg}} % kilogram
\newcommand{\uKGinv}{\uKG^{-1}} % kilogram inverse
\newcommand{\uKGinvsq}{\uKG^{-2}} % kilogram inverse squared
\newcommand{\uG}{\te{g}} % gram

% lumen
\newcommand{\uLM}{\te{lm}} % lumen

% kelvin
\newcommand{\uK}{\te{K}} % kelvin
\newcommand{\uKsq}{\uK^\te{2}} % kelvin squared
\newcommand{\uKcu}{\uK^\te{3}} % kelvin cubed
\newcommand{\uKquar}{\uK^\te{4}} % kelvin quartic
\newcommand{\uKinv}{\uK^{-1}} % kelvin inverse
\newcommand{\uKinvsq}{\uK^{-2}} % kelvin inverse squared
\newcommand{\uKinvcu}{\uK^{-3}} % kelvin inverse cubed
\newcommand{\uKinvquar}{\uK^{-4}} % kelvin inverse quartic

% degree celsius
\newcommand{\uCELS}{\uDEGREE \te{C}} % degree celsius

% watt
\newcommand{\uW}{\te{W}} % watt
\newcommand{\uKW}{\te{kW}} % kilowatt
\newcommand{\uMW}{\te{MW}} % megawatt
\newcommand{\uGW}{\te{GW}} % gigawatt
\newcommand{\uWinv}{\uW^{-1}} % watt inverse

% joule
\newcommand{\uJ}{\te{J}} % joule
\newcommand{\uKJ}{\te{kJ}} % kilojoule
\newcommand{\uMJ}{\te{MJ}} % megajoule
\newcommand{\uCAL}{\te{cal}} % calorie
\newcommand{\uKCAL}{\te{kcal}} % kilocalorie
\newcommand{\uWH}{\te{Wh}} % watt hour
\newcommand{\uKWH}{\te{kWh}} % kilowatt hour
\newcommand{\uWandS}{\te{Ws}} % watt second

% ampere
\newcommand{\uA}{\te{A}} % ampere

% volt
\newcommand{\uV}{\te{V}} % volt

% ohm
\newcommand{\uOHM}{\te{\Omega}} % ohm
\newcommand{\uOHMinv}{\uOHM^{-1}} % ohm inverse

% siemens
\newcommand{\uSIE}{\te{S}} % siemens
\newcommand{\uSIEinv}{\uSIE^{-1}} % siemens inverse

% newton
\newcommand{\uN}{\te{N}} % newton

% currency
\newcommand{\uCCY}{\te{CCY}} % currency
\newcommand{\uCZK}{\te{CZK}} % czech crown

% --------------------
% Compound units
% --------------------

% velocity-acceleration
\newcommand{\uKGandMinvcu}{\uKG \cdot \uMinvcu} % kilogram per meter cubed
\newcommand{\uMandSinv}{\uM \cdot \uSinv} % meter per second
\newcommand{\uMcuSinv}{\uMcu \cdot \uSinv} % meter cubed per second
\newcommand{\uMandSinvsq}{\uM \cdot \uSinvsq} % meter per second squared

% power-energy
\newcommand{\uWandMinvsq}{\uW \cdot \uMinvsq} % watt per meter squared
\newcommand{\uWandMinvcu}{\uW \cdot \uMinvcu} % watt per meter cubed
\newcommand{\uJandMinvsq}{\uJ \cdot \uMinvsq} % joule per meter squared
\newcommand{\uJandMinvcu}{\uJ \cdot \uMinvcu} % joule per meter cubed

% heat
\newcommand{\uJandKGinvKinv}{\uJ \cdot \uKGinv \cdot \uKinv} % joule per kilogram per kelvin
\newcommand{\uWandMinvKinv}{\uW \cdot \uMinv \cdot \uKinv} % watt per meter per kelvin
\newcommand{\uMsqKandWinv}{\uMsq \cdot \uK \cdot \uWinv} % meter squared kelvin per watt
\newcommand{\uKandWinv}{\uK \cdot \uWinv} % kelvin per watt inverse
\newcommand{\uWandMinvsqKinv}{\uW \cdot \uMinvsq \cdot \uKinv} % watt per meter squared per kelvin
\newcommand{\uWandKinv}{\uW \cdot \uKinv} % watt per kelvin inverse

% electricity
\newcommand{\uOHMandMinv}{\uOHM \cdot \uMinv} % ohm per meter
\newcommand{\uSIEandMinv}{\uSIE \cdot \uMinv} % siemens per meter
\newcommand{\uAandMinvsq}{\uA \cdot \uMinvsq} % ampere per meter squared
\newcommand{\uVandMinv}{\uV \cdot \uMinv} % volt per meter


% --------------------
% --------------------
% Units in equation
% --------------------
% --------------------

% --------------------
% Basic units
% --------------------

% no unit
\newcommand{\ueqNOUNIT}{$\uNOUNIT$} % no unit
\newcommand{\ueqPERCENT}{$\uPERCENT$} % percent
\newcommand{\ueqDEGREE}{$\uDEGREE$} % degree
\newcommand{\ueqRAD}{$\uRAD$} % radian

% meter
\newcommand{\ueqM}{$\uM$} % meter
\newcommand{\ueqMsq}{$\uMsq$} % meter squared
\newcommand{\ueqMcu}{$\uMcu$} % meter cubed
\newcommand{\ueqMquar}{$\uMquar$} % meter quartic
\newcommand{\ueqMinv}{$\uMinv$} % meter inverse
\newcommand{\ueqMinvsq}{$\uMinvsq$} % meter inverse squared
\newcommand{\ueqMinvcu}{$\uMinvcu$} % meter inverse cubed
\newcommand{\ueqMinvquar}{$\uMinvquar$} % meter inverse quartic
\newcommand{\ueqMM}{$\uMM$} % millimeter
\newcommand{\ueqCM}{$\uCM$} % centimeter
\newcommand{\ueqKM}{$\uKM$} % kilometer

% liter
\newcommand{\ueqL}{$\uL$} % liter
\newcommand{\ueqML}{$\uML$} % milliliter

% second
\newcommand{\ueqS}{$\uS$} % second
\newcommand{\ueqSinv}{$\uSinv$} % second inverse
\newcommand{\ueqSinvsq}{$\uSinvsq$} % second inverse squared

% hour
\newcommand{\ueqH}{$\uH$} % hour
\newcommand{\ueqHinv}{$\uHinv$} % hour inverse

% kilogram
\newcommand{\ueqKG}{$\uKG$} % kilogram
\newcommand{\ueqKGinv}{$\uKGinv$} % kilogram inverse
\newcommand{\ueqKGinvsq}{$\uKGinvsq$} % kilogram inverse squared
\newcommand{\ueqG}{$\uG$} % gram

% lumen
\newcommand{\ueqLM}{$\uLM$} % lumen

% kelvin
\newcommand{\ueqK}{$\uK$} % kelvin
\newcommand{\ueqKsq}{$\uKsq$} % kelvin squared
\newcommand{\ueqKcu}{$\uKcv$} % kelvin cubed
\newcommand{\ueqKquar}{$\uKquar$} % kelvin quartic
\newcommand{\ueqKinv}{$\uKinv$} % kelvin inverse
\newcommand{\ueqKinvsq}{$\uKinvsq$} % kelvin inverse squared
\newcommand{\ueqKinvcu}{$\uKinvcu$} % kelvin inverse cubed
\newcommand{\ueqKinvquar}{$\uuKinvquar$} % kelvin inverse quartic

% degree celsius
\newcommand{\ueqCELS}{$\uCELS$} % degree celsius

% watt
\newcommand{\ueqW}{$\uW$} % watt
\newcommand{\ueqKW}{$\uKW$} % kilowatt
\newcommand{\ueqMW}{$\uMW$} % megawatt
\newcommand{\ueqGW}{$\uGW$} % gigawatt
\newcommand{\ueqWinv}{$\uWinv$} % watt inverse

% joule
\newcommand{\ueqJ}{$\uJ$} % joule
\newcommand{\ueqKJ}{$\uKJ$} % kilojoule
\newcommand{\ueqMJ}{$\uMJ$} % megajoule
\newcommand{\ueqCAL}{$\uCAL$} % calorie
\newcommand{\ueqKCAL}{$\uKCAL$} % kilocalorie
\newcommand{\ueqWH}{$\uWH$} % watt hour
\newcommand{\ueqKWH}{$\uKWH$} % kilowatt hour
\newcommand{\ueqWandS}{$\uWandS$} % watt second

% ampere
\newcommand{\ueqA}{$\uA$} % ampere

% volt
\newcommand{\ueqV}{$\uV$} % volt

% ohm
\newcommand{\ueqOHM}{$\uOHM$} % ohm
\newcommand{\ueqOHMinv}{$\uOHMinv$} % ohm inverse

% siemens
\newcommand{\ueqSIE}{$\uSIE$} % siemens
\newcommand{\ueqSIEinv}{$\uSIEinv$} % siemens inverse

% newton
\newcommand{\ueqN}{$\uN$} % newton

% currency
\newcommand{\ueqCCY}{$\uCCY$} % currency
\newcommand{\ueqCZK}{$\uCZK$} % czech crown

% --------------------
% Compound units
% --------------------

% velocity-acceleration
\newcommand{\ueqKGandMinvcu}{$\uKGandMinvcu$} % kilogram per meter cubed
\newcommand{\ueqMandSinv}{$\uMandSinv$} % meter per second
\newcommand{\ueqMcuSinv}{$\uMcuSinv$} % meter cubed per second
\newcommand{\ueqMandSinvsq}{$\uMandSinvsq$} % meter per second squared

% power-energy
\newcommand{\ueqWandMinvsq}{$\uWandMinvsq$} % watt per meter squared
\newcommand{\ueqWandMinvcu}{$\uWandMinvcu$} % watt per meter cubed
\newcommand{\ueqJandMinvsq}{$\uJandMinvsq$} % joule per meter squared
\newcommand{\ueqJandMinvcu}{$\uJandMinvcu$} % joule per meter cubed

% heat
\newcommand{\ueqJandKGinvKinv}{$\uJandKGinvKinv$} % joule per kilogram per kelvin
\newcommand{\ueqWandMinvKinv}{$\uWandMinvKinv$} % watt per meter per kelvin
\newcommand{\ueqMsqKandWinv}{$\uMsqKandWinv$} % meter squared kelvin per watt
\newcommand{\ueqKandWinv}{$\uKandWinv$} % kelvin per watt inverse
\newcommand{\ueqWandMinvsqKinv}{$\uWandMinvsqKinv$} % watt per meter squared per kelvin
\newcommand{\ueqWandKinv}{$\uWandKinv$} % watt per kelvin inverse

% electricity
\newcommand{\ueqOHMandMinv}{$\uOHMandMinv$} % ohm per meter
\newcommand{\ueqSIEandMinv}{$\uSIEandMinv$} % siemens per meter
\newcommand{\ueqAandMinvsq}{$\uAandMinvsq$} % ampere per meter squared
\newcommand{\ueqVandMinv}{$\uVandMinv$} % volt per meter


\cvHead{2 - Sdílení tepla}

\begin{document}

\maketitle
\tableofcontents
\newpage



\section*{Příklad 4 - Fourierova-Kirchhoffova rovnice}

\textbf{Zadání}

Fourierova-Kirchhoffova rovnice je definována jako:
\begin{equation}
    \rho \cdot c \cdot \left( \frac{\partial T}{\partial t} + \vec{v} \cdot \vec{\nabla} T \right) = \nabla \cdot \left( \lambda \cdot \vec{\nabla} T \right) + Q_v,
\end{equation}
kde:
\begin{itemize}
    \item $\rho$ - hustota (kg / m3),
    \item $c$ - měrná tepelná kapacita (\ueqJandKGinvKinv),
    \item $T$ - teplota (K),
    \item $t$ - čas (s),
    \item $\vec{v}$ - rychlost (m / s),
    \item $\lambda$ - tepelná vodivost (\ueqJandKGinvKinv),
    \item $Q_v$ - objemový zdroj tepla (W / m3).
\end{itemize}

Řešte zjednodušenou rovnici pro tloušťku stěny $d$, kde:

\begin{itemize}
    \item $\lambda$ = konstanta,
    \item $\vec{v}$ = 0.
    \item $Q_v$ = 0,
    \item T = T(x). ??
\end{itemize}

S okrajovými podmínkami:
\begin{itemize}
    \item $T(0) = T_0$,
    \item $T(d) = T_1$.
\end{itemize}

\textbf{Řešení}

Zjednodušená rovnice je definována jako:
$$
    0 = \nabla \cdot \left( \lambda \cdot \vec{\nabla} T \right) = \lambda \cdot \frac{d^2 T}{d x^2}.
$$

Tedy:
$$
    T^{''}(x) = 0.
$$

Rovnici lze řešit jednodušše řešit jako lineární hommoní rovnici, jejiž charakteristický polynom je:
$$
    \lambda^2 = 0.
$$

Tedy 0 je kořenem násobnosti 2. Poté je obecné řešení:
$$
    T(x) = c_0 \cdot e^{0 \cdot x} + c_1 \cdot x \cdot e^{0 \cdot x} = c_0 + c_1 \cdot x.
$$

S okrajovými podmínkami:
\begin{itemize}
    \item $T(0) = T_0$:
          $$
              T_0 = T(0) = c_0 + c_1 \cdot 0 = c_0 \Rightarrow c_0 = T_0.
          $$
    \item $T(d) = T_1$:
          $$
              T_1 = T(d) = T_0 + c_1 \cdot d \Rightarrow c_1 = \frac{T_1 - T_0}{d}.
          $$
\end{itemize}

Řešení je tedy:
$$
    T(x) = T_0 + \frac{T_1 - T_0}{d} \cdot x.
$$

\textit{Poznámka}

Pro kontrolu lze dosadit do řešení okrajové podmínky, čímž se ověří, že řešení je správné:
$$
    T(0) = T_0 + \frac{T_1 - T_0}{d} \cdot 0 = T_0.
$$

$$
    T(d) = T_0 + \frac{T_1 - T_0}{d} \cdot d = T_1.
$$



\section*{Příklad 5 - Fourieruv zákon}

\textbf{Zadání}

Fourierův zákon je definován jako:
\begin{equation}
    \vec{q} = - \lambda \cdot \vec{\nabla} T,
\end{equation}
kde:
\begin{itemize}
    \item $\vec{q}$ - tepelný tok (\ueqWandMinvsq),
    \item $\lambda$ - tepelná vodivost (\ueqWandMinvsqKinv),
    \item $T$ - teplota (K).
    \item $\vec{\nabla} T$ - gradient teploty (K / m).
\end{itemize}

Dosaďte řešení z příkladu 4 do Fourierova zákona.

\textbf{Řešení}

Dosaďme řešení z příkladu 4 do Fourierova zákona:
$$
    q_x = - \lambda  \frac{dT(x)}{dx} = - \lambda  \frac{d}{dx} \left( T_0 + \frac{T_1 - T_0}{d} \cdot x \right) = - \lambda  \frac{T_1 - T_0}{d} = \lambda  \frac{T_0 - T_1}{d}.
$$

Celkový tepelný tok $Q_x$ (W) je:
$$
    Q_x = q_x \cdot S = \lambda \frac{T_0 - T_1}{d} \cdot S = \lambda \frac{\Delta T}{d} \cdot S = \frac{\Delta T}{\frac{d}{\lambda \cdot S}}.
$$

Tepelný odpor $R_{th}$ je:
$$
    R_{th} = \frac{d}{\lambda \cdot S}.
$$

Poté je celkový tepelný tok:
$$
    Q_x = \frac{\Delta T}{R_{th}}.
$$

\textit{Poznámka}
Tato rovnice je analogická s Ohmovým zákonem, kde:
\begin{itemize}
    \item $Q_x \rightarrow I$ (A) - proud,
    \item $\Delta T \rightarrow U$ (V) - napětí,
    \item $R_{th} \rightarrow R$ ($\Omega$) - odpor.
\end{itemize}

Ohmuv zákon je definován jako:
$$
    I = \frac{U}{R}.
$$

Zde elektrický odpor $R$ je:
$$
    R = \frac{d}{\sigma \cdot S}.
$$

Zde analogie je:
\begin{itemize}
    \item $d$ - tloušťka stěny (m) $\rightarrow$ $d$ - délka vodiče (m),
    \item $\lambda$ - tepelná vodivost (\ueqWandMinvsqKinv) $\rightarrow$ $\sigma$ - elektrická vodivost (s / m = $1 / (\Omega \cdot m)$),
    \item $S$ - plocha průřezu (m2) $\rightarrow$ $S$ - průřez vodiče (m2).
\end{itemize}



\section*{Příklad 6 - Tepelný odpor}

\textbf{Zadání}

\begin{itemize}
    \item $T_1$ = 20 \ueqCELS,
    \item $T_2$ = -10 \ueqCELS.
\end{itemize}

Mějme dva typy zdí:
\begin{enumerate}[a)]
    \item
\end{enumerate}



\section{Zeď \spicy \spicy}


\subsection{Fourierova-Kirchhoffova rovnice \spicy \spicy \spicy}
Pro zeď o tloušťce $d$ budeme uvažovat následující zjednodušející předpoklady:
\begin{itemize}
    \item $\lambda$ = konstanta,
    \item $\vec{v}$ = 0,
    \item $Q_v$ = 0,
    \item $\frac{\partial T}{\partial t}$ = 0,
    \item T = T(x) - teplota závislá pouze na ose x.
\end{itemize}

Poté Fourierova-Kirchhoffova rovnice bude mít tvar:
$$
    0 = \nabla \cdot \left( \lambda \cdot \vec{\nabla} T \right) = \lambda \cdot \frac{d^2 T}{d x^2}.
$$

Rovnici můžeme vydělit $\lambda$ ($\lambda > 0$) a získáme:
$$
    \frac{d^2 T}{d x^2} = 0.
$$

Tuto rovnici můžeme řešit dvojí integrací. První integrace bude vypadat následovně:
$$
    \frac{d T}{d x} = \int 0 \cdot dx = 0 + c_1 = c_1.
$$

Druhá integrace bude vypadat následovně:
$$
    T(x) = \int c_1 \cdot dx = c_1 \cdot x + c_2.
$$

Obecné řešení bude tedy:
$$
    T(x) = c_1 \cdot x + c_2.
$$

S okrajovými podmínkami:
\begin{itemize}
    \item $T(0) = T_1$:
          $$
              T_1 = T(0) = c_1 \cdot 0 + c_2 = c_2 \Rightarrow c_2 = T_1.
          $$
    \item $T(d) = T_2$:
          $$
              T_2 = T(d) = c_1 \cdot d + c_2 \Rightarrow c_1 = \frac{T_2 - c_2}{d} = \frac{T_2 - T_1}{d}.
          $$
\end{itemize}

Řešení je tedy:
$$
    T(x) = \frac{T_2 - T_1}{d} \cdot x + T_1.
$$

Tuto situaci můžeme znázornit následujícím obrázkem:

\begin{center}
    \begin{tikzpicture}
        % Variables
        \def\TOne{4} % T1
        \def\TTwo{1} % T2
        \def\d{5} % width of the wall, d

        % Axes
        \draw[->] (0,0) -- (\d + 1,0) node[anchor=north] {$x$}; % x-axis
        \draw[->] (0,0) -- (0,\TOne + 1) node[anchor=east] {$T(x)$}; % y-axis

        % Line representing T(x)
        \draw[thick, red] (-2,\TOne) -- (0,\TOne);
        \draw[thick, red] (0,\TOne) -- (\d,\TTwo);
        \draw[thick, red] (\d,\TTwo) -- (\d + 2,\TTwo);

        % Dotted line
        \draw[dashed] (0,\TOne) -- (1,\TOne);
        \draw[dashed] (0,\TTwo) -- (\d,\TTwo);

        % Wall lines
        \draw[very thick] (0,0) -- (0,\TOne + 0.5);
        \draw[very thick] (\d,0) -- (\d,\TOne - 1.5);
        \draw[very thick] (\d,\TOne - 0.5) -- (\d,\TOne + 0.5);

        % Labels
        \node[anchor=south] at (4, 2.7) {$T(x) = \frac{T_2 - T_1}{d} \cdot x + T_1$};
        \node[anchor=west] at (1, \TOne) {$T_1$};
        \node[anchor=east] at (0, \TTwo) {$T_2$};
        \node[anchor=north] at (\d, 0) {$d$};
    \end{tikzpicture}
\end{center}



\subsection{Fourieruv zákon \spicy \spicy \spicy}


\subsection{Tepelný odpor}


\subsection{Koeficient prostupu tepla}


\subsection{Číselný příklad}



\section{PENB \spicy \spicy \spicy}



\section{Topná sezóna \spicy \spicy \spicy}



\section{Cihlová pec \spicy \spicy \spicy \spicy}



\section{Sálání (radiace) \spicy \spicy}


\subsection{Sálavá clona}


\subsection{Destička ve vesmíru}



\end{document}
