\documentclass{article}
\usepackage[czech]{babel} % Czech language
\usepackage[shortlabels]{enumitem} % Custom enumeration
\usepackage{graphicx} % Import images
\usepackage{float} % Use [H] to force figure position
\usepackage{indentfirst} % Indent first paragraph
\usepackage{emoji} % Emojis
\usepackage{pgffor} % Loops
\usepackage{tikz} % TikZ
\usepackage{pgfplots} % TikZ plots
\usepackage{circuitikz} % Use circuitikz for circuit diagrams
\usepackage{amsmath} % Math
\usetikzlibrary{arrows.meta}

\makeatletter
\providecommand\add@text{}
\newcommand\tagaddtext[1]{%
    \gdef\add@text{#1\gdef\add@text{}}}% 
\renewcommand\tagform@[1]{%
    \maketag@@@{\llap{\add@text\quad}(\ignorespaces#1\unskip\@@italiccorr)}%
}
\makeatother

\newcommand{\cvHead}[1]{\head{Cvičení #1}}

\newcommand{\head}[1]{
    \title{\textbf{#1}\\Elektroenergetika 3}
    \author{Petr Jílek}
    \date{2024}
}

\newcommand{\spicy}{\emoji{hot-pepper}}

% My font space
\newcommand{\myFS}{\;}

% Text in math mode
\newcommand{\te}[1]{\textrm{#1}}

% --------------------
% Units
% --------------------

\newcommand{\uM}{\textrm{m}} % Meter
\newcommand{\uMsq}{\uM^\textrm{2}} % Meter squared
\newcommand{\uMcu}{\uM^\textrm{3}} % Meter cubed
\newcommand{\uS}{\textrm{s}} % Second
\newcommand{\uKG}{\textrm{kg}} % Kilogram
\newcommand{\uJ}{\textrm{J}} % Joule
\newcommand{\uK}{\textrm{K}} % Kelvin
\newcommand{\uDEGREE}{^\circ} % Degree
\newcommand{\uCELS}{\uDEGREE \textrm{C}} % Celsius
\newcommand{\uW}{\textrm{W}} % Watt
\newcommand{\uMW}{\textrm{MW}} % Mega Watt
\newcommand{\uKW}{\textrm{kW}} % Kilo Watt
\newcommand{\uWH}{\textrm{Wh}} % Watt hour
\newcommand{\uKWH}{\textrm{kWh}} % Kilo Watt hour
\newcommand{\uCCY}{\textrm{CCY}} % Currency
\newcommand{\uCZK}{\textrm{CZK}} % Czech Crown
\newcommand{\uPERCENT}{\textrm{\%}} % Percent
\newcommand{\uNOUNIT}{\textrm{--}} % No unit
\newcommand{\uYEAR}{\textrm{year}} % Year
\newcommand{\uMONTH}{\textrm{month}} % Month
\newcommand{\uHOUR}{\textrm{hour}} % Hour
\newcommand{\uLM}{\textrm{lm}} % Lumen
\newcommand{\uHinv}{\textrm{h}^{-1}} % Hour inverse

\newcommand{\uJperK}{\uJ / \uK} % Joule per Kelvin
\newcommand{\uKperM}{\uK / \uM} % Kelvin per Meter
\newcommand{\uKperS}{\uK / \uS} % Kelvin per Second
\newcommand{\uMsqperS}{\uMsq/\uS} % Meter squared per Second
\newcommand{\uWperMsq}{\uW / \uMsq} % Watt per Meter squared
\newcommand{\uWperMperK}{\uW / \left( \uM \cdot \uK \right)} % Watt per Meter per Kelvin
\newcommand{\uJperKGperK}{\uJ / \left( \uKG \cdot \uK \right)} % Joule per Kilogram per Kelvin
\newcommand{\uKperMsq}{\uK / \uMsq} % Kelvin per Meter squared
\newcommand{\uJperSperMperK}{\uJ / \left( \uS \cdot \uM \cdot \uK \right)} % Joule per Second per Meter per Kelvin
\newcommand{\uKGperMcu}{\uKG / \uMcu} % Kilogram per Meter cubed
\newcommand{\uMsqKperW}{\uMsq \cdot \uK / \uW} % Meter squared Kelvin per Watt
\newcommand{\uWperMsqperK}{\uW / \left( \uMsq \cdot \uK \right)} % Watt per Meter squared per Kelvin
\newcommand{\uKWHperMsqperYEAR}{\uKWH / \left( \uMsq \cdot \uYEAR \right)} % Kilo Watt hour per Meter squared per Year
\newcommand{\uKWHperMsq}{\uKWH / \uMsq} % Kilo Watt hour per Meter squared
\newcommand{\uLMperW}{\uLM / \uW} % Lumen per Watt
\newcommand{\uKWperMsq}{\uKW / \uMsq} % Kilo Watt per Meter squared
\newcommand{\uCCYperYEAR}{\uCCY / \uYEAR} % Currency per Year
\newcommand{\uKWHperYEAR}{\uKWH / \uYEAR} % Kilo Watt hour per Year
\newcommand{\uNOUNITperYEAR}{\uNOUNIT / \uYEAR} % No unit per Year
\newcommand{\uCCYperKWH}{\uCCY / \uKWH} % Currency per Kilo Watt hour
\newcommand{\uCZKperYEAR}{\uCZK / \uYEAR} % Czech Crown per Year
\newcommand{\uCZKperKWH}{\uCZK / \uKWH} % Czech Crown per Kilo Watt hour
\newcommand{\uCZKperMONTH}{\uCZK / \uMONTH} % Czech Crown per Month



% --------------------
% Unit equations
% --------------------

\newcommand{\ueqM}{$\uM$}
\newcommand{\ueqMsq}{$\uMsq$}
\newcommand{\ueqMcu}{$\uMcu$}
\newcommand{\ueqS}{$\uS$}
\newcommand{\ueqJ}{$\uJ$}
\newcommand{\ueqK}{$\uK$}
\newcommand{\ueqDEGREE}{$\uDEGREE$}
\newcommand{\ueqCELS}{$\uCELS$}
\newcommand{\ueqW}{$\uW$}
\newcommand{\ueqMW}{$\uMW$}
\newcommand{\ueqKW}{$\uKW$}
\newcommand{\ueqWH}{$\uWH$}
\newcommand{\ueqKWH}{$\uKWH$}
\newcommand{\ueqCCY}{$\uCCY$}
\newcommand{\ueqCZK}{$\uCZK$}
\newcommand{\ueqPERCENT}{$\uPERCENT$}
\newcommand{\ueqNOUNIT}{$\uNOUNIT$}
\newcommand{\ueqYEAR}{$\uYEAR$}
\newcommand{\ueqMONTH}{$\uMONTH$}
\newcommand{\ueqHOUR}{$\uHOUR$}
\newcommand{\ueqLM}{$\uLM$}
\newcommand{\ueqHinv}{$\uHinv$}

\newcommand{\ueqJperK}{$\uJperK$}
\newcommand{\ueqKperM}{$\uKperM$}
\newcommand{\ueqKperS}{$\uKperS$}
\newcommand{\ueqMsqperS}{$\uMsqperS$}
\newcommand{\ueqWperMsq}{$\uWperMsq$}
\newcommand{\ueqWperMperK}{$\uWperMperK$}
\newcommand{\ueqJperKGperK}{$\uJperKGperK$}
\newcommand{\ueqKperMsq}{$\uKperMsq$}
\newcommand{\ueqJperSperMperK}{$\uJperSperMperK$}
\newcommand{\ueqKGperMcu}{$\uKGperMcu$}
\newcommand{\ueqMsqKperW}{$\uMsqKperW$}
\newcommand{\ueqWperMsqperK}{$\uWperMsqperK$}
\newcommand{\ueqKWHperMsqperYEAR}{$\uKWHperMsqperYEAR$}
\newcommand{\ueqKWHperMsq}{$\uKWHperMsq$}
\newcommand{\ueqLMperW}{$\uLMperW$}
\newcommand{\ueqKWperMsq}{$\uKWperMsq$}
\newcommand{\ueqCCYperYEAR}{$\uCCYperYEAR$}
\newcommand{\ueqKWHperYEAR}{$\uKWHperYEAR$}
\newcommand{\ueqNOUNITperYEAR}{$\uNOUNITperYEAR$}
\newcommand{\ueqCCYperKWH}{$\uCCYperKWH$}
\newcommand{\ueqCZKperYEAR}{$\uCZKperYEAR$}
\newcommand{\ueqCZKperKWH}{$\uCZKperKWH$}
\newcommand{\ueqCZKperMONTH}{$\uCZKperMONTH$}


\cvHead{4 - Symetrizace}

\begin{document}

\maketitle
\tableofcontents
\newpage



\section{\emoji{electric-plug} Symetrizace \spicy \spicy}

\subsection{1 fázová reálná zátěž}
Mějme 1 fázovou reálnou zátěž zdanou reálnou admitancí $G$:
$$
    G = \frac{1}{R},
$$
kde:\\
$G$ - vodivost zátěže $(\uSIE)$,\\
$R$ - odpor zátěže $(\uOHM)$.\\

Zátěž lze jednoduše nakreslit následovně:

\begin{center}
    \begin{circuitikz}[scale=0.8][european voltages]
        \draw
        (0,0) to [fullgeneric, -, l_=$G$] (6,0);
    \end{circuitikz}
\end{center}

Pokud bychom tuto zátěž připojily k 3 fázovému systému, tak by byla nesymetrická. Našim cílem je tuto zátěž symetrizovat. Z této zátěže vytvoříme 3 fázovou symetrickou zátěž následovně:

\begin{center}
    \begin{circuitikz}[scale=0.8][european voltages]
        \draw
        (0,0)
        to [capacitor, *-, l_=$\hat{Y}_{1,3}$] (3,5)
        to [cute inductor, *-, l_=$\hat{Y}_{1,2}$] (6,0)
        to [fullgeneric, *-, l_=$G$] (0,0);

        \node[anchor=west] at (3,5) {1};
        \node[anchor=west] at (6,0) {2};
        \node[anchor=east] at (0,0) {3};
    \end{circuitikz}
\end{center}

Vzorce pro výpočet admitancí jsou poté následující:
$$
    \hat{Y}_{1,2} = -j \frac{G}{\sqrt{3}},
$$
$$
    \hat{Y}_{1,3} = j \frac{G}{\sqrt{3}}.
$$

\textit{Poznámka: Sled fází jde po směru hodinových ručiček, a tento směr je třeba zachovat.}

\subsubsection{Odvození \spicy \spicy \spicy \spicy}
Uvažujme následující zapojení prvku v 3 fázovém systému:

\begin{center}
    \begin{circuitikz}[scale=0.8][european voltages]
        \draw
        (0,0)
        to [fullgeneric, *-, l_=$\hat{Y}_{1,3}$] (3,5)
        to [fullgeneric, *-, l_=$\hat{Y}_{1,2}$] (6,0)
        to [fullgeneric, *-, l_=$G$] (0,0)

        to [short, -] (0,-2)
        to [short, -] (8,-2)
        to [short, -] (8,0)
        to [american voltage source, *-] (11,2.5)

        (6,0) to [short, -] (6,-1)
        to [short, -] (14,-1)
        to [short, -] (14,0)
        to [american voltage source, *-] (11,2.5)

        (3,5) to [short, -] (3,6)
        to [short, -] (11,6)
        to [short, -] (11,5)
        to [american voltage source, *-] (11,2.5)

        (11.5,5) to [open, v^>=$\hat{U}_1$] (11.5,2.5)
        (13.5,0) to [open, v^>=$\hat{U}_2$] (11.2,1.5)
        (8,1) to [open, v^>=$\hat{U}_3$] (10,2.5);

        \draw[-{Triangle}] (7.5,5.7) -- (6.5,5.7) node[midway, below] {$\hat{I}_1$};
        \draw[-{Triangle}] (7.5,-0.7) -- (6.5,-0.7) node[midway, above] {$\hat{I}_2$};
        \draw[-{Triangle}] (7.5,-2.3) -- (6.5,-2.3) node[midway, below] {$\hat{I}_3$};

        \draw[-{Triangle}] (2.5,4.8) -- (2,4) node[midway, left] {$\hat{I}_{1,3}$};
        \draw[-{Triangle}] (3.5,4.8) -- (4,4) node[midway, right] {$\hat{I}_{1,2}$};
        \draw[-{Triangle}] (5.5,-0.3) -- (4.5,-0.3) node[midway, below] {$\hat{I}_{2,3}$};

        \node[anchor=west] at (3,5) {1};
        \node[anchor=west] at (6,0) {2};
        \node[anchor=east] at (0,0) {3};

        \node[anchor=west] at (11,5) {1};
        \node[anchor=west] at (14,0) {2};
        \node[anchor=east] at (8,0) {3};
    \end{circuitikz}
\end{center}

Mezi uzly 2 a 3 je zapojena reálná zátěž, kterou chceme symetrizovat o vodivosti $G \fs (\uSIE) $. Požadujeme, aby po připojení admitancí $\hat{Y}_{1,2} \fs (\uSIE)$ a $\hat{Y_{1.3}} \fs (\uSIE)$ byla zátěž reálná a symetrická. Dalším požadavkem je, aby činný výkon odebíraný zátěží zůstal nezměněn. Matematicky to znamená:

\begin{itemize}
    \item zachování činného výkonu: $\hat{Y}_{1,2}$ a $\hat{Y}_{1,3}$ jsou ryze imaginární,
    \item výsledné zapojení neodebírá jalový výkon: $\hat{Y}_{1,2} = - \hat{Y}_{1,3}$,
    \item symetrie odebíraných proudů: $\hat{I}_1 = k \cdot \hat{U}_1$, $\hat{I}_2 = k \cdot \hat{U_2}$, $\hat{I}_3 = k \cdot \hat{U}_3$.
\end{itemize}

Položme: $\hat{Y}_{1,2} = j \cdot Y$ a $\hat{Y}_{1,3} = -j \cdot Y$.\\

Použijeme operátor pootočení o $120^\circ$ proti směru hodinových ručiček v komplexní rovině:
$$
    \hat{a} = e^{\frac{2 \pi j}{3}} = \te{cos}(\frac{2 \pi}{3}) + j \cdot \te{sin}(\frac{2 \pi}{3}) = -\frac{1}{2} + j \cdot \frac{\sqrt{3}}{2}
$$
$$
    \hat{a}^2 = e^{\frac{4 \pi j}{3}} = \te{cos}(\frac{4 \pi}{3}) + j \cdot \te{sin}(\frac{4 \pi}{3}) = -\frac{1}{2} - j \cdot \frac{\sqrt{3}}{2}
$$
$$
    \hat{a}^3 = e^{\frac{6 \pi j}{3}} = e^{2 \pi j} = 1
$$
$$
    \hat{a}^4 = e^{\frac{8 \pi j}{3}} = e^{\frac{2 \pi j}{3}} \cdot e^{\frac{6 \pi j}{3}} = e^{\frac{2 \pi j}{3}} = \hat{a}
$$

Poté fázory napětí můžeme vyjádřit jako:
$$
    \hat{U}_1 = \hat{U} \cdot \hat{a}
$$
$$
    \hat{U}_2 = \hat{U} \cdot \hat{a}^2
$$
$$
    \hat{U}_3 = \hat{U} \cdot \hat{a}
$$

Můžeme napsat rovnice pro proudy:
$$
    \hat{I}_1 = \hat{I}_{1,2} + \hat{I}_{1,3} = \hat{Y}_{1,2} \cdot \left( \hat{U}_1 - \hat{U}_2 \right) + \hat{Y}_{1,3} \cdot \left( \hat{U}_1 - \hat{U}_3 \right) =
$$
$$
    = j \cdot Y \cdot U \cdot \left( 1 - \hat{a}^2 \right) - j \cdot Y \cdot U \cdot \left( 1 - \hat{a} \right) = k \cdot U
$$

$$
    \hat{I}_2 = \hat{I}_{2,3} - \hat{I}_{1,2} = G \cdot \left( \hat{U}_2 - \hat{U}_3 \right) - \hat{Y}_{1,2} \cdot \left( \hat{U}_1 - \hat{U}_2 \right) =
$$
$$
    = G \cdot U \cdot \left( \hat{a}^2 - \hat{a} \right) - j \cdot Y \cdot U \cdot \left( 1 - \hat{a}^2 \right) = k \cdot U \cdot \hat{a}^2
$$

$$
    \hat{I}_3 = - \hat{I}_{1,3} - \hat{I}_{2,3} = - \hat{Y}_{1,3} \cdot \left( \hat{U}_1 - \hat{U}_3 \right) - G \cdot \left( \hat{U}_2 - \hat{U}_3 \right) =
$$
$$
    = j \cdot Y \cdot U \cdot \left( 1 - \hat{a} \right) - G \cdot U \cdot \left( \hat{a}^2 - \hat{a} \right) = k \cdot U \cdot \hat{a}
$$

Vezmeme konce rovnic, čímž dostaneme:
$$
    j \cdot Y \cdot U \cdot \left( 1 - \hat{a}^2 \right) - j \cdot Y \cdot U \cdot \left( 1 - \hat{a} \right) = k \cdot U
$$
$$
    G \cdot U \cdot \left( \hat{a}^2 - \hat{a} \right) - j \cdot Y \cdot U \cdot \left( 1 - \hat{a}^2 \right) = k \cdot U \cdot \hat{a}^2
$$
$$
    j \cdot Y \cdot U \cdot \left( 1 - \hat{a} \right) - G \cdot U \cdot \left( \hat{a}^2 - \hat{a} \right) = k \cdot U \cdot \hat{a}
$$

Rovnice jsou lineárně závislé, jelikož součet pravých stran je roven nule:
$$
    k \cdot U + k \cdot U \cdot \hat{a} + k \cdot U \cdot \hat{a}^2 = k \cdot U \cdot \left( 1 + \hat{a} + \hat{a}^2 \right) = k \cdot U \cdot 0 = 0
$$

Díky lineární závislosti nám stačí vzít pouze dvě rovnice. Vezmeme první a druhou:
$$
    j \cdot Y \cdot U \cdot (1 - \hat{a}^2) - j \cdot Y \cdot U \cdot (1 - \hat{a}) = k \cdot U
$$
$$
    G \cdot U \cdot (\hat{a}^2 - \hat{a}) - j \cdot Y \cdot U \cdot (1 - \hat{a}^2) = k \cdot U \cdot \hat{a}^2
$$

Můžeme pokrátit $U$:
$$
    j \cdot Y \cdot (1 - \hat{a}^2) - j \cdot Y \cdot (1 - \hat{a}) = k
$$
$$
    G \cdot (\hat{a}^2 - \hat{a}) - j \cdot Y \cdot (1 - \hat{a}^2) = k \cdot \hat{a}^2
$$

Připravíme rovnice na maticový tvar:
$$
    -k + \left( j \cdot (1 - \hat{a}^2) - j \cdot (1 - \hat{a}) \right) \cdot Y = 0
$$
$$
    -k \cdot \hat{a}^2 - j \cdot (1 - \hat{a}^2) \cdot Y = -G \cdot (\hat{a}^2 - \hat{a})
$$

Přepíšeme do maticového tvaru:
$$
    \begin{pmatrix}
        -1         & j (1 - \hat{a}^2) - j (1 - \hat{a}) \\
        -\hat{a}^2 & - j (1 - \hat{a}^2)
    \end{pmatrix}
    \cdot
    \begin{pmatrix}
        k \\
        Y
    \end{pmatrix}
    =
    \begin{pmatrix}
        0 \\
        -G (\hat{a}^2 - \hat{a})
    \end{pmatrix}
$$

$$
    \begin{pmatrix}
        -1         & j - j \hat{a}^2 - j + j \hat{a} \\
        -\hat{a}^2 & - j + j \hat{a}^2
    \end{pmatrix}
    \cdot
    \begin{pmatrix}
        k \\
        Y
    \end{pmatrix}
    =
    \begin{pmatrix}
        0 \\
        - G (\hat{a}^2 - \hat{a})
    \end{pmatrix}
$$

$$
    \begin{pmatrix}
        -1         & - j \hat{a}^2 + j \hat{a} \\
        -\hat{a}^2 & j \hat{a}^2 - j
    \end{pmatrix}
    \cdot
    \begin{pmatrix}
        k \\
        Y
    \end{pmatrix}
    =
    \begin{pmatrix}
        0 \\
        - G (\hat{a}^2 - \hat{a})
    \end{pmatrix}
$$

$$
    \begin{pmatrix}
        -\hat{a}^2 & - j \hat{a}^4 + j \hat{a}^3 \\
        -\hat{a}^2 & j \hat{a}^2 - j
    \end{pmatrix}
    \cdot
    \begin{pmatrix}
        k \\
        Y
    \end{pmatrix}
    =
    \begin{pmatrix}
        0 \\
        - G (\hat{a}^2 - \hat{a})
    \end{pmatrix}
$$

$$
    \begin{pmatrix}
        -\hat{a}^2 & - j \hat{a} + j \\
        -\hat{a}^2 & j \hat{a}^2 - j
    \end{pmatrix}
    \cdot
    \begin{pmatrix}
        k \\
        Y
    \end{pmatrix}
    =
    \begin{pmatrix}
        0 \\
        - G (\hat{a}^2 - \hat{a})
    \end{pmatrix}
$$

$$
    \begin{pmatrix}
        -\hat{a}^2 & - j \hat{a} + j                 \\
        0          & j \hat{a}^2 - j + j \hat{a} - j
    \end{pmatrix}
    \cdot
    \begin{pmatrix}
        k \\
        Y
    \end{pmatrix}
    =
    \begin{pmatrix}
        0 \\
        - G (\hat{a}^2 - \hat{a})
    \end{pmatrix}
$$

$$
    \begin{pmatrix}
        -\hat{a}^2 & - j \hat{a} + j               \\
        0          & j \hat{a}^2 + j \hat{a} - 2 j
    \end{pmatrix}
    \cdot
    \begin{pmatrix}
        k \\
        Y
    \end{pmatrix}
    =
    \begin{pmatrix}
        0 \\
        - G (\hat{a}^2 - \hat{a})
    \end{pmatrix}
$$

Vyřešíme rovnici pro $Y$:
$$
    Y = \frac{- G (\hat{a}^2 - \hat{a})}{j \hat{a}^2 + j \hat{a} - 2 j} = \frac{
        - G \left( -\frac{1}{2} - j \frac{\sqrt{3}}{2} - \left( -\frac{1}{2} + j \frac{\sqrt{3}}{2} \right) \right)
    }{
        j \left( -\frac{1}{2} - j \frac{\sqrt{3}}{2} \right) + j \left( -\frac{1}{2} + j \frac{\sqrt{3}}{2} \right) - 2 j =
    }
$$
$$
    = \frac{
        - G \left( -\frac{1}{2} - j \frac{\sqrt{3}}{2} + \frac{1}{2} - j \frac{\sqrt{3}}{2} \right)
    }{
        -j \frac{1}{2} + \frac{\sqrt{3}}{2} - j \frac{1}{2} - \frac{\sqrt{3}}{2} - 2 j
    } = \frac{
        - G \left( -j \sqrt{3} \right)
    }{
        -3 j
    } = \frac{- G \sqrt{3}}{3} = -\frac{G}{\sqrt{3}}
$$

Vyřešíme rovnici pro $k$:
$$
    -\hat{a}^2 k + (-j \hat{a} + j) Y = 0
$$
$$
    k = \frac{- (-j \hat{a} + j) Y}{-\hat{a}^2} = \frac{(-j \hat{a} + j) Y}{\hat{a}^2} = \frac{(-j \hat{a} + j) \left( -\frac{G}{\sqrt{3}} \right)}{\hat{a}^2} =
$$
$$
    = \frac{
        \left( -j \left( -\frac{1}{2} + j \frac{\sqrt{3}}{2} \right) + j \right) \left( -\frac{G}{\sqrt{3}} \right)
    }{
        -\frac{1}{2} - j \frac{\sqrt{3}}{2}
    } = \frac{
        \left( j \frac{1}{2} + \frac{\sqrt{3}}{2} + j \right) \left( -\frac{G}{\sqrt{3}} \right)
    }{
        -\frac{1}{2} - j \frac{\sqrt{3}}{2}
    } =
$$
$$
    = \frac{
        \left( j \frac{3}{2} + \frac{\sqrt{3}}{2} \right) \left( -\frac{G}{\sqrt{3}} \right)
    }{
        -\frac{1}{2} - j \frac{\sqrt{3}}{2}
    } = \frac{
        \left( - j \frac{3}{2 \sqrt{3}} - \frac{\sqrt{3}}{2 \sqrt{3}} \right) G
    }{
        -\frac{1}{2} - j \frac{\sqrt{3}}{2}
    } =
$$
$$
    = \frac{
        \left( -\frac{1}{2} - j \frac{\sqrt{3}}{2} \right) G
    }{
        -\frac{1}{2} - j \frac{\sqrt{3}}{2}
    } = G
$$

Dostáváme tedy:
$$
    k = G,
$$
$$
    Y = -\frac{G}{\sqrt{3}}.
$$


\subsection{Obecná 3f nesymetrická zátěž}
Mějme obecnou 3 fázovou nesymetrickou zátěž zadanou admitancemi podle obrázku:

\begin{center}
    \begin{circuitikz}[scale=0.8][european voltages]
        \draw
        (0,0)
        to [fullgeneric, *-, l_=$\hat{Y}_{1,3}$] (3,5)
        to [fullgeneric, *-, l_=$\hat{Y}_{1,2}$] (6,0)
        to [fullgeneric, *-, l_=$\hat{Y}_{2,3}$] (0,0);

        \node[anchor=west] at (3,5) {1};
        \node[anchor=west] at (6,0) {2};
        \node[anchor=east] at (0,0) {3};
    \end{circuitikz}
\end{center}

Admitance můžeme zapsat jako:
$$
    \hat{Y}_{1,2} = G_{1,2} + j \cdot B_{1,2},
$$
$$
    \hat{Y}_{1,3} = G_{1,3} + j \cdot B_{1,3},
$$
$$
    \hat{Y}_{2,3} = G_{2,3} + j \cdot B_{2,3}.
$$

První krok je provést kompenzaci jalových částí. Stačí pouze vzít zápornou hodnotu jalové části zátěže:

\begin{center}
    \begin{circuitikz}[scale=0.8][european voltages]
        \draw
        (0,0)
        to [fullgeneric, *-, l_=$-B_{1,2}$] (3,5)
        to [fullgeneric, *-, l_=$-B_{1,3}$] (6,0)
        to [fullgeneric, *-, l_=$-B_{2,3}$] (0,0);

        \node[anchor=west] at (3,5) {1};
        \node[anchor=west] at (6,0) {2};
        \node[anchor=east] at (0,0) {3};
    \end{circuitikz}
\end{center}

Dále je třeba provést symetrizaci pro každou část zvlášť. Nejprve pro větev 1--2, poté pro větev 1--3 a nakonec pro větev 2--3. Tento krok je znázorněn na obrázku:

\begin{figure}[H]
    \centering
    \subfloat{
        \begin{circuitikz}[scale=0.4][european voltages]
            \draw
            (0,0) to [short, *-] (0,0)
            (3,5) to [fullgeneric, *-, l_=$\hat{Y}_{1,2}$]
            (6,0) to [short, *-] (6,0);

            \node[anchor=west] at (3,5) {1};
            \node[anchor=west] at (6,0) {2};
            \node[anchor=east] at (0,0) {3};
        \end{circuitikz}
    }
    \qquad
    \subfloat{
        \begin{circuitikz}[scale=0.4][european voltages]
            \draw
            (0,0) to [fullgeneric, *-, l_=$\hat{Y}_{1,3}$] (3,5)
            (3,5) to [short, *-] (3,5)
            (6,0) to [short, *-] (6,0);

            \node[anchor=west] at (3,5) {1};
            \node[anchor=west] at (6,0) {2};
            \node[anchor=east] at (0,0) {3};
        \end{circuitikz}
    }
    \qquad
    \subfloat{
        \begin{circuitikz}[scale=0.4][european voltages]
            \draw
            (3,5) to [short, *-] (3,5)
            (0,0) to [short, *-] (0,0)
            (6,0) to [fullgeneric, *-, l_=$\hat{Y}_{2,3}$] (0,0);

            \node[anchor=west] at (3,5) {1};
            \node[anchor=west] at (6,0) {2};
            \node[anchor=east] at (0,0) {3};
        \end{circuitikz}
    }
\end{figure}

Výsledná tabulka symetrizace bude vypadat následovně:
\begin{center}
    \begin{tabular}{l c c c}
        \hline
        Větev                      & 1--2                          & 1--3                          & 2--3                          \\
        \hline                                                                                                                     \\
        Kompenzace jalového výkonu & $-j B_{1,2}$                  & $-j B_{1,3}$                  & $-j B_{2,3}$                  \\~\\
        Symetrizace 1--2           & 0                             & $-j \frac{G_{1,2}}{\sqrt{3}}$ & $j \frac{G_{1,2}}{\sqrt{3}}$  \\~\\
        Symetrizace 1--3           & $j \frac{G_{1,3}}{\sqrt{3}}$  & 0                             & $-j \frac{G_{1,3}}{\sqrt{3}}$ \\~\\
        Symetrizace 2--3           & $-j \frac{G_{2,3}}{\sqrt{3}}$ & $j \frac{G_{2,3}}{\sqrt{3}}$  & 0                             \\~\\
        \hline
    \end{tabular}
\end{center}

Symetrizační admitanci pro danou větev dostaneme jako součet všech symetrizačních admitancí (suma ve sloupci):
$$
    \hat{Y}_{s,1,2} = -j B_{1,2} + j \frac{G_{1,3}}{\sqrt{3}} - j \frac{G_{2,3}}{\sqrt{3}},
$$
$$
    \hat{Y}_{s,1,3} = -j B_{1,3} - j \frac{G_{1,2}}{\sqrt{3}} + j \frac{G_{2,3}}{\sqrt{3}},
$$
$$
    \hat{Y}_{s,2,3} = -j B_{2,3} + j \frac{G_{1,2}}{\sqrt{3}} - j \frac{G_{1,3}}{\sqrt{3}}.
$$

Tyto admitance následně připojíme parallelně k odpovídajícím větvím. Výsledný obvod bude vypadat následovně:

\begin{center}
    \begin{circuitikz}[scale=0.8][european voltages]
        \draw
        (0,0)
        to [fullgeneric, *-, l_=$\hat{Y}_{1,3}$] (3,5)
        to [fullgeneric, *-, l_=$\hat{Y}_{1,2}$] (6,0)
        to [fullgeneric, *-, l_=$\hat{Y}_{2,3}$] (0,0)

        (0,0) to [short, -] (-5/3,1)
        (-5/3,1) to [fullgeneric, -, l_=$\hat{Y}_{s,1,3}$] (4/3,6)
        (4/3,6) to [short, -] (3,5)

        (0,0) to [short, -] (0,-2)
        (0,-2) to [fullgeneric, -, l_=$\hat{Y}_{s,2,3}$] (6,-2)
        (6,-2) to [short, -] (6,0)

        (3,5) to [short, -] (14/3,6)
        (14/3,6) to [fullgeneric, -, l_=$\hat{Y}_{s,1,2}$] (23/3,1)
        (23/3,1) to [short, -] (6,0);

        \node[anchor=south] at (3,5) {1};
        \node[anchor=west] at (6,0) {2};
        \node[anchor=east] at (0,0) {3};
    \end{circuitikz}
\end{center}


\subsection{Přepočet výkonů na admitance}
Zátěže jsou často zadány pomocí: činného výkonu, úhlu $\te{cos} (\varphi) \fs$ a informací, zda je zátěž induktivní nebo kapacitní. Tyto informace můžeme převést na admitanci $\hat{Y} \fs (\uSIE)$ následovně:
$$
    \hat{Y} = \frac{P}{U^2} \cdot (1 - j \cdot \te{tg} (\pm \varphi)) = \frac{P}{U^2} \cdot (1 \mp j \cdot \te{tg} (\varphi)),
$$
kde:\\
$P$ - činný výkon (\ueqW),\\
$U$ - efektivní hodnota napětí (\ueqV),\\
$j$ - imaginární jednotka,\\
$\varphi$ - úhel $\te{cos} (\varphi)$,\\
horní znaménko $\pm / \mp$ - induktivní zátěž,\\
dolní znaménko $\pm / \mp$ - kapacitní zátěž.\\

Pro induktivní zátěž:
$$
    \hat{Y} = \frac{P}{U^2} \cdot (1 - j \cdot \te{tg} (\varphi)).
$$

Pro kapacitní zátěž:
$$
    \hat{Y} = \frac{P}{U^2} \cdot (1 + j \cdot \te{tg} (\varphi)).
$$

\subsubsection{Odvození \spicy \spicy \spicy}
Ze vztahu pro napětí vyjádříme produ v závislosti na admitanci:
$$
    \hat{U} = \hat{I} \cdot \hat{Z} = \frac{\hat{I}}{\hat{Y}} \Rightarrow \hat{I} = \hat{U} \cdot \hat{Y}.
$$

Dále použijeme vztah pro zdánlivý výkon:
$$
    \hat{S} = \hat{U} \cdot \hat{I}^* = \hat{U} \cdot \left( \hat{U} \cdot \hat{Y} \right)^* = \hat{U} \cdot \hat{U}^* \cdot \hat{Y}^* = U^2 \cdot \hat{Y}^* \Rightarrow \hat{Y} = \frac{\hat{S}}{U^2}.
$$
$$
    \hat{Y} = \frac{\left( P + j \cdot Q \right)^*}{U^2} = \frac{P - j \cdot Q}{U^2} = \frac{P - j \cdot P \cdot \te{tg} (\varphi)}{U^2}
$$

Úhel $\varphi$ je kladný pro induktivní zátěž a záporný pro kapacitní zátěž. Funkce $\te{tg} (\varphi)$ lichá:
$$
    \te{tg} (-\varphi) = \frac{\te{sin} (-\varphi)}{\te{cos} (-\varphi)} = \frac{-\te{sin} (\varphi)}{\te{cos} (\varphi)} = - \frac{\te{sin} (\varphi)}{\te{cos} (\varphi)} = - \te{tg} (\varphi).
$$

Tudíž můžeme výsledný vztah zapsat jako:
$$
    \hat{Y} = \frac{P}{U^2} \cdot (1 \mp j \cdot \te{tg} (\varphi)).
$$


\subsection{Číselný příklad}
Mějme 3 fázovou nesymetrickou zátěž nazančenou na obrázku:

\begin{center}
    \begin{circuitikz}[scale=0.8][european voltages]
        \draw
        (0,0)
        to [fullgeneric, *-, l_=$P_{1,3}$] (3,5)
        to [fullgeneric, *-, l_=$P_{1,2}$] (6,0)
        to [fullgeneric, *-, l_=$P_{2,3}$] (0,0);

        \node[anchor=south] at (3,5) {1};
        \node[anchor=west] at (6,0) {2};
        \node[anchor=east] at (0,0) {3};
    \end{circuitikz}
\end{center}

Parametry:
\begin{itemize}
    \item $U = 400 \uV$,
    \item $\te{cos} (\varphi) = 0.8$,
    \item $P_{1,2} = 63 \fs \uKW$, induktivní,
    \item $P_{1,3} = 28 \fs \uKW$, induktivní,
    \item $P_{2,3} = 26 \fs \uKW$, kapacitní.
\end{itemize}

Proveďte symetrizaci zátěže.

\subsubsection{Řešení}
Nejprve získáme úhel $\varphi$:
$$
    \varphi = \te{arccos} (0.8) \approx 0.644 \fs \te{rad}.
$$

Dále vypočítáme $\te{tg} (\varphi)$:
$$
    \te{tg} (\varphi) = \te{tg} (0.644) \approx 0.751.
$$

Následně získáme admitance:
$$
    Y_{1,2} = \frac{P_{1,2}}{U^2} \cdot (1 - j \cdot \te{tg} (\varphi)) = \frac{63 \fs 000}{400^2} \cdot (1 - j \cdot 0.751) = (0.394 - j \cdot 0.236) \fs \uSIE,
$$
$$
    Y_{1,3} = \frac{P_{1,3}}{U^2} \cdot (1 - j \cdot \te{tg} (\varphi)) = \frac{28 \fs 000}{400^2} \cdot (1 - j \cdot 0.751) = (0.175 - j \cdot 0.131) \fs \uSIE,
$$
$$
    Y_{2,3} = \frac{P_{2,3}}{U^2} \cdot (1 + j \cdot \te{tg} (\varphi)) = \frac{26 \fs 000}{400^2} \cdot (1 + j \cdot 0.751) = (0.163 + j \cdot 0.122) \fs \uSIE.
$$

Dále vytvoříme tabulku symetrizace:

\begin{center}
    \begin{tabular}{l l l l}
        \hline
        Větev                      & 1--2                        & 1--3                        & 2--3                        \\
        \hline                                                                                                               \\
        Kompenzace jalového výkonu & $j 0.236$                   & $j 0.131$                   & $-j 0.122$                  \\~\\
        Symetrizace 1--2           & 0                           & $-j \frac{0.394}{\sqrt{3}}$ & $j \frac{0.394}{\sqrt{3}}$  \\~\\
        Symetrizace 1--3           & $j \frac{0.175}{\sqrt{3}}$  & 0                           & $-j \frac{0.175}{\sqrt{3}}$ \\~\\
        Symetrizace 2--3           & $-j \frac{0.163}{\sqrt{3}}$ & $j \frac{0.163}{\sqrt{3}}$  & 0                           \\~\\
        \hline
    \end{tabular}
\end{center}

Symetrizační admitance:
$$
    Y_{s,1,2} = j 0.236 + j \frac{0.175}{\sqrt{3}} - j \frac{0.163}{\sqrt{3}} = j 0.236 + j 0.101 - j 0.094 = j 0.243 \fs \uSIE,
$$
$$
    Y_{s,1,3} = j 0.131 - j \frac{0.394}{\sqrt{3}} + j \frac{0.163}{\sqrt{3}} = j 0.131 - j 0.227 + j 0.094 = -j 0.002 \fs \uSIE,
$$
$$
    Y_{s,2,3} = -j 0.122 + j \frac{0.394}{\sqrt{3}} - j \frac{0.175}{\sqrt{3}} = -j 0.122 + j 0.227 - j 0.101 = j 0.004 \fs \uSIE.
$$

Výsledné zapojení bude vypadat následovně:

\begin{center}
    \begin{circuitikz}[scale=0.8][european voltages]
        \draw
        (0,0)
        to [fullgeneric, *-, l_=$\hat{Y}_{1,3}$] (3,5)
        to [fullgeneric, *-, l_=$\hat{Y}_{1,2}$] (6,0)
        to [fullgeneric, *-, l_=$\hat{Y}_{2,3}$] (0,0)

        (0,0) to [short, -] (-5/3,1)
        (-5/3,1) to [cute inductor, -, l_=$\hat{Y}_{s,1,3}$] (4/3,6)
        (4/3,6) to [short, -] (3,5)

        (0,0) to [short, -] (0,-2)
        (0,-2) to [capacitor, -, l_=$\hat{Y}_{s,2,3}$] (6,-2)
        (6,-2) to [short, -] (6,0)

        (3,5) to [short, -] (14/3,6)
        (14/3,6) to [capacitor, -, l_=$\hat{Y}_{s,1,2}$] (23/3,1)
        (23/3,1) to [short, -] (6,0);

        \node[anchor=south] at (3,5) {1};
        \node[anchor=west] at (6,0) {2};
        \node[anchor=east] at (0,0) {3};
    \end{circuitikz}
\end{center}

\end{document}
