\documentclass{article}
\usepackage[czech]{babel} % Czech language
\usepackage[shortlabels]{enumitem} % Custom enumeration
\usepackage{graphicx} % Import images
\usepackage{float} % Use [H] to force figure position
\usepackage{indentfirst} % Indent first paragraph
\usepackage{emoji} % Emojis
\usepackage{pgffor} % Loops
\usepackage{tikz} % TikZ
\usepackage{pgfplots} % TikZ plots
\usepackage{amsmath} % Math
\usetikzlibrary{arrows.meta}

\makeatletter
\providecommand\add@text{}
\newcommand\tagaddtext[1]{%
    \gdef\add@text{#1\gdef\add@text{}}}% 
\renewcommand\tagform@[1]{%
    \maketag@@@{\llap{\add@text\quad}(\ignorespaces#1\unskip\@@italiccorr)}%
}
\makeatother

\newcommand{\cvHead}[1]{\head{Cvičení #1}}

\newcommand{\head}[1]{
    \title{\textbf{#1}\\Elektroenergetika 3}
    \author{Petr Jílek}
    \date{2024}
}

\newcommand{\spicy}{\emoji{hot-pepper}}

% --------------------
% --------------------
% Units
% --------------------
% --------------------

% --------------------
% Basic units
% --------------------  

% no unit
\newcommand{\uNOUNIT}{\te{--}} % no unit
\newcommand{\uPERCENT}{\te{\%}} % percent
\newcommand{\uDEGREE}{^\circ} % degree
\newcommand{\uRAD}{\te{rad}} % radian

% meter
\newcommand{\uM}{\te{m}} % meter
\newcommand{\uMsq}{\uM^\te{2}} % meter squared
\newcommand{\uMcu}{\uM^\te{3}} % meter cubed
\newcommand{\uMquar}{\uM^\te{4}} % meter quartic
\newcommand{\uMinv}{\uM^{-1}} % meter inverse
\newcommand{\uMinvsq}{\uM^{-2}} % meter inverse squared
\newcommand{\uMinvcu}{\uM^{-3}} % meter inverse cubed
\newcommand{\uMinvquar}{\uM^{-4}} % meter inverse quartic
\newcommand{\uMM}{\te{mm}} % millimeter
\newcommand{\uCM}{\te{cm}} % centimeter
\newcommand{\uKM}{\te{km}} % kilometer

% liter
\newcommand{\uL}{\te{l}} % liter
\newcommand{\uML}{\te{ml}} % milliliter

% second
\newcommand{\uS}{\te{s}} % second
\newcommand{\uSinv}{\uS^{-1}} % second inverse
\newcommand{\uSinvsq}{\uS^{-2}} % second inverse squared

% hour
\newcommand{\uH}{\te{h}} % hour
\newcommand{\uHinv}{\te{h}^{-1}} % hour inverse

% kilogram
\newcommand{\uKG}{\te{kg}} % kilogram
\newcommand{\uKGinv}{\uKG^{-1}} % kilogram inverse
\newcommand{\uKGinvsq}{\uKG^{-2}} % kilogram inverse squared
\newcommand{\uG}{\te{g}} % gram

% lumen
\newcommand{\uLM}{\te{lm}} % lumen

% kelvin
\newcommand{\uK}{\te{K}} % kelvin
\newcommand{\uKsq}{\uK^\te{2}} % kelvin squared
\newcommand{\uKcu}{\uK^\te{3}} % kelvin cubed
\newcommand{\uKquar}{\uK^\te{4}} % kelvin quartic
\newcommand{\uKinv}{\uK^{-1}} % kelvin inverse
\newcommand{\uKinvsq}{\uK^{-2}} % kelvin inverse squared
\newcommand{\uKinvcu}{\uK^{-3}} % kelvin inverse cubed
\newcommand{\uKinvquar}{\uK^{-4}} % kelvin inverse quartic

% degree celsius
\newcommand{\uCELS}{\uDEGREE \te{C}} % degree celsius

% watt
\newcommand{\uW}{\te{W}} % watt
\newcommand{\uKW}{\te{kW}} % kilowatt
\newcommand{\uMW}{\te{MW}} % megawatt
\newcommand{\uGW}{\te{GW}} % gigawatt
\newcommand{\uWinv}{\uW^{-1}} % watt inverse

% joule
\newcommand{\uJ}{\te{J}} % joule
\newcommand{\uKJ}{\te{kJ}} % kilojoule
\newcommand{\uMJ}{\te{MJ}} % megajoule
\newcommand{\uCAL}{\te{cal}} % calorie
\newcommand{\uKCAL}{\te{kcal}} % kilocalorie
\newcommand{\uWH}{\te{Wh}} % watt hour
\newcommand{\uKWH}{\te{kWh}} % kilowatt hour
\newcommand{\uWandS}{\te{Ws}} % watt second

% ampere
\newcommand{\uA}{\te{A}} % ampere

% volt
\newcommand{\uV}{\te{V}} % volt

% ohm
\newcommand{\uOHM}{\te{\Omega}} % ohm
\newcommand{\uOHMinv}{\uOHM^{-1}} % ohm inverse

% siemens
\newcommand{\uSIE}{\te{S}} % siemens
\newcommand{\uSIEinv}{\uSIE^{-1}} % siemens inverse

% newton
\newcommand{\uN}{\te{N}} % newton

% currency
\newcommand{\uCCY}{\te{CCY}} % currency
\newcommand{\uCZK}{\te{CZK}} % czech crown

% --------------------
% Compound units
% --------------------

% velocity-acceleration
\newcommand{\uKGandMinvcu}{\uKG \cdot \uMinvcu} % kilogram per meter cubed
\newcommand{\uMandSinv}{\uM \cdot \uSinv} % meter per second
\newcommand{\uMcuSinv}{\uMcu \cdot \uSinv} % meter cubed per second
\newcommand{\uMandSinvsq}{\uM \cdot \uSinvsq} % meter per second squared

% power-energy
\newcommand{\uWandMinvsq}{\uW \cdot \uMinvsq} % watt per meter squared
\newcommand{\uWandMinvcu}{\uW \cdot \uMinvcu} % watt per meter cubed
\newcommand{\uJandMinvsq}{\uJ \cdot \uMinvsq} % joule per meter squared
\newcommand{\uJandMinvcu}{\uJ \cdot \uMinvcu} % joule per meter cubed

% heat
\newcommand{\uJandKGinvKinv}{\uJ \cdot \uKGinv \cdot \uKinv} % joule per kilogram per kelvin
\newcommand{\uWandMinvKinv}{\uW \cdot \uMinv \cdot \uKinv} % watt per meter per kelvin
\newcommand{\uMsqKandWinv}{\uMsq \cdot \uK \cdot \uWinv} % meter squared kelvin per watt
\newcommand{\uKandWinv}{\uK \cdot \uWinv} % kelvin per watt inverse
\newcommand{\uWandMinvsqKinv}{\uW \cdot \uMinvsq \cdot \uKinv} % watt per meter squared per kelvin
\newcommand{\uWandKinv}{\uW \cdot \uKinv} % watt per kelvin inverse

% electricity
\newcommand{\uOHMandMinv}{\uOHM \cdot \uMinv} % ohm per meter
\newcommand{\uSIEandMinv}{\uSIE \cdot \uMinv} % siemens per meter
\newcommand{\uAandMinvsq}{\uA \cdot \uMinvsq} % ampere per meter squared
\newcommand{\uVandMinv}{\uV \cdot \uMinv} % volt per meter


% --------------------
% --------------------
% Units in equation
% --------------------
% --------------------

% --------------------
% Basic units
% --------------------

% no unit
\newcommand{\ueqNOUNIT}{$\uNOUNIT$} % no unit
\newcommand{\ueqPERCENT}{$\uPERCENT$} % percent
\newcommand{\ueqDEGREE}{$\uDEGREE$} % degree
\newcommand{\ueqRAD}{$\uRAD$} % radian

% meter
\newcommand{\ueqM}{$\uM$} % meter
\newcommand{\ueqMsq}{$\uMsq$} % meter squared
\newcommand{\ueqMcu}{$\uMcu$} % meter cubed
\newcommand{\ueqMquar}{$\uMquar$} % meter quartic
\newcommand{\ueqMinv}{$\uMinv$} % meter inverse
\newcommand{\ueqMinvsq}{$\uMinvsq$} % meter inverse squared
\newcommand{\ueqMinvcu}{$\uMinvcu$} % meter inverse cubed
\newcommand{\ueqMinvquar}{$\uMinvquar$} % meter inverse quartic
\newcommand{\ueqMM}{$\uMM$} % millimeter
\newcommand{\ueqCM}{$\uCM$} % centimeter
\newcommand{\ueqKM}{$\uKM$} % kilometer

% liter
\newcommand{\ueqL}{$\uL$} % liter
\newcommand{\ueqML}{$\uML$} % milliliter

% second
\newcommand{\ueqS}{$\uS$} % second
\newcommand{\ueqSinv}{$\uSinv$} % second inverse
\newcommand{\ueqSinvsq}{$\uSinvsq$} % second inverse squared

% hour
\newcommand{\ueqH}{$\uH$} % hour
\newcommand{\ueqHinv}{$\uHinv$} % hour inverse

% kilogram
\newcommand{\ueqKG}{$\uKG$} % kilogram
\newcommand{\ueqKGinv}{$\uKGinv$} % kilogram inverse
\newcommand{\ueqKGinvsq}{$\uKGinvsq$} % kilogram inverse squared
\newcommand{\ueqG}{$\uG$} % gram

% lumen
\newcommand{\ueqLM}{$\uLM$} % lumen

% kelvin
\newcommand{\ueqK}{$\uK$} % kelvin
\newcommand{\ueqKsq}{$\uKsq$} % kelvin squared
\newcommand{\ueqKcu}{$\uKcv$} % kelvin cubed
\newcommand{\ueqKquar}{$\uKquar$} % kelvin quartic
\newcommand{\ueqKinv}{$\uKinv$} % kelvin inverse
\newcommand{\ueqKinvsq}{$\uKinvsq$} % kelvin inverse squared
\newcommand{\ueqKinvcu}{$\uKinvcu$} % kelvin inverse cubed
\newcommand{\ueqKinvquar}{$\uuKinvquar$} % kelvin inverse quartic

% degree celsius
\newcommand{\ueqCELS}{$\uCELS$} % degree celsius

% watt
\newcommand{\ueqW}{$\uW$} % watt
\newcommand{\ueqKW}{$\uKW$} % kilowatt
\newcommand{\ueqMW}{$\uMW$} % megawatt
\newcommand{\ueqGW}{$\uGW$} % gigawatt
\newcommand{\ueqWinv}{$\uWinv$} % watt inverse

% joule
\newcommand{\ueqJ}{$\uJ$} % joule
\newcommand{\ueqKJ}{$\uKJ$} % kilojoule
\newcommand{\ueqMJ}{$\uMJ$} % megajoule
\newcommand{\ueqCAL}{$\uCAL$} % calorie
\newcommand{\ueqKCAL}{$\uKCAL$} % kilocalorie
\newcommand{\ueqWH}{$\uWH$} % watt hour
\newcommand{\ueqKWH}{$\uKWH$} % kilowatt hour
\newcommand{\ueqWandS}{$\uWandS$} % watt second

% ampere
\newcommand{\ueqA}{$\uA$} % ampere

% volt
\newcommand{\ueqV}{$\uV$} % volt

% ohm
\newcommand{\ueqOHM}{$\uOHM$} % ohm
\newcommand{\ueqOHMinv}{$\uOHMinv$} % ohm inverse

% siemens
\newcommand{\ueqSIE}{$\uSIE$} % siemens
\newcommand{\ueqSIEinv}{$\uSIEinv$} % siemens inverse

% newton
\newcommand{\ueqN}{$\uN$} % newton

% currency
\newcommand{\ueqCCY}{$\uCCY$} % currency
\newcommand{\ueqCZK}{$\uCZK$} % czech crown

% --------------------
% Compound units
% --------------------

% velocity-acceleration
\newcommand{\ueqKGandMinvcu}{$\uKGandMinvcu$} % kilogram per meter cubed
\newcommand{\ueqMandSinv}{$\uMandSinv$} % meter per second
\newcommand{\ueqMcuSinv}{$\uMcuSinv$} % meter cubed per second
\newcommand{\ueqMandSinvsq}{$\uMandSinvsq$} % meter per second squared

% power-energy
\newcommand{\ueqWandMinvsq}{$\uWandMinvsq$} % watt per meter squared
\newcommand{\ueqWandMinvcu}{$\uWandMinvcu$} % watt per meter cubed
\newcommand{\ueqJandMinvsq}{$\uJandMinvsq$} % joule per meter squared
\newcommand{\ueqJandMinvcu}{$\uJandMinvcu$} % joule per meter cubed

% heat
\newcommand{\ueqJandKGinvKinv}{$\uJandKGinvKinv$} % joule per kilogram per kelvin
\newcommand{\ueqWandMinvKinv}{$\uWandMinvKinv$} % watt per meter per kelvin
\newcommand{\ueqMsqKandWinv}{$\uMsqKandWinv$} % meter squared kelvin per watt
\newcommand{\ueqKandWinv}{$\uKandWinv$} % kelvin per watt inverse
\newcommand{\ueqWandMinvsqKinv}{$\uWandMinvsqKinv$} % watt per meter squared per kelvin
\newcommand{\ueqWandKinv}{$\uWandKinv$} % watt per kelvin inverse

% electricity
\newcommand{\ueqOHMandMinv}{$\uOHMandMinv$} % ohm per meter
\newcommand{\ueqSIEandMinv}{$\uSIEandMinv$} % siemens per meter
\newcommand{\ueqAandMinvsq}{$\uAandMinvsq$} % ampere per meter squared
\newcommand{\ueqVandMinv}{$\uVandMinv$} % volt per meter


\cvHead{4 - Symetrizace}

\begin{document}

\maketitle
\tableofcontents
\newpage



\section{Symetrizace \spicy \spicy}

\subsection{1 fázová reálná zátěž}
Mějme 1 fázovou reálnou zátěž zdanou reálnou admitancí $G \fs (\uOHMinv)$:
$$
    G = \frac{1}{R},
$$
kde:\\
$R$ - odpor zátěže $(\uOHM)$.\\

\begin{center}
    \begin{circuitikz}[scale=0.8][european voltages]
        \draw
        (0,0) to [fullgeneric, -, l_=$G$] (6,0);
    \end{circuitikz}
\end{center}

Pokud bychom tuto zátěž připojily k 3 fázovému systému, tak by byla nesymetrická. Našim cílem je tuto zátěž symetrizovat. Z této zátěže vytvoříme 3 fázovou symetrickou zátěž následovně:

\begin{center}
    \begin{circuitikz}[scale=0.8][european voltages]
        \draw
        (0,0)
        to [capacitor, *-, l_=$\hat{Y}_{1,3}$] (3,5)
        to [cute inductor, *-, l_=$\hat{Y}_{1,2}$] (6,0)
        to [fullgeneric, *-, l_=$G$] (0,0);
    \end{circuitikz}
\end{center}

$$
    Y_{1,2} = -j \frac{G}{\sqrt{3}}
$$
$$
    Y_{1,3} = j \frac{G}{\sqrt{3}}
$$

\subsubsection{Odvození \spicy \spicy \spicy \spicy}
Uvažujme následující zapojení prvku v 3 fázovém systému:

\begin{center}
    \begin{circuitikz}[scale=0.8][european voltages]
        \draw
        (0,0)
        to [fullgeneric, *-, l_=$\hat{Y}_{1,3}$] (3,5)
        to [fullgeneric, *-, l_=$\hat{Y}_{1,2}$] (6,0)
        to [fullgeneric, *-, l_=$G$] (0,0)

        to [short, -] (0,-2)
        to [short, -] (8,-2)
        to [short, -] (8,0)
        to [american voltage source, *-] (11,2.5)

        (6,0) to [short, -] (6,-1)
        to [short, -] (14,-1)
        to [short, -] (14,0)
        to [american voltage source, *-] (11,2.5)

        (3,5) to [short, -] (3,6)
        to [short, -] (11,6)
        to [short, -] (11,5)
        to [american voltage source, *-] (11,2.5)

        (11.5,5) to [open, v^>=$\hat{U}_1$] (11.5,2.5)
        (13.5,0) to [open, v^>=$\hat{U}_2$] (11.2,1.5)
        (8,1) to [open, v^>=$\hat{U}_3$] (10,2.5);

        \draw[-{Triangle}] (7.5,5.7) -- (6.5,5.7) node[midway, below] {$\hat{I}_1$};
        \draw[-{Triangle}] (7.5,-0.7) -- (6.5,-0.7) node[midway, above] {$\hat{I}_2$};
        \draw[-{Triangle}] (7.5,-2.3) -- (6.5,-2.3) node[midway, below] {$\hat{I}_3$};

        \draw[-{Triangle}] (2.5,4.8) -- (2,4) node[midway, left] {$\hat{I}_{1,3}$};
        \draw[-{Triangle}] (3.5,4.8) -- (4,4) node[midway, right] {$\hat{I}_{1,2}$};
        \draw[-{Triangle}] (5.5,-0.3) -- (4.5,-0.3) node[midway, below] {$\hat{I}_{2,3}$};

        \node[anchor=west] at (3,5) {1};
        \node[anchor=west] at (6,0) {2};
        \node[anchor=east] at (0,0) {3};

        \node[anchor=west] at (11,5) {1};
        \node[anchor=west] at (14,0) {2};
        \node[anchor=east] at (8,0) {3};
    \end{circuitikz}
\end{center}

Mezi uzly 2 a 3 je zapojena reálná zátěž, kterou chceme symetrizovat o vodivosti $G \fs (\uOHMinv) $. Požadujeme, aby po připojení admitancí $\hat{Y}_{1,2} \fs (\uOHMinv)$ a $\hat{Y_{1.3}} \fs (\uOHMinv)$ byla zátěž reálná a symetrická. Dalším požadavkem je, aby činný výkon odebíraný zátěží zůstal nezměněn. Matematicky to znamená:

\begin{itemize}
    \item zachování činného výkonu: $\hat{Y}_{1,2}$ a $\hat{Y}_{1,3}$ jsou ryze imaginární,
    \item výsledné zapojení neodebírá jalový výkon: $\hat{Y}_{1,2} = - \hat{Y}_{1,3}$
    \item symetrie odebíraných proudů: $\hat{I}_1 = k \cdot \hat{U}_1$, $\hat{I}_2 = k \cdot \hat{U_2}$, $\hat{I}_3 = k \cdot \hat{U}_3$.
\end{itemize}

Položme: $\hat{Y}_{1,2} = j \cdot Y$ a $\hat{Y}_{1,3} = -j \cdot Y$.\\

Použijeme operátor pootočení o $120^\circ$ proti směru hodinových ručiček v komplexní rovině:
$$
    \hat{a} = e^{\frac{2 \pi j}{3}} = \te{cos}(\frac{2 \pi}{3}) + j \cdot \te{sin}(\frac{2 \pi}{3}) = -\frac{1}{2} + j \cdot \frac{\sqrt{3}}{2}
$$
$$
    \hat{a}^2 = e^{\frac{4 \pi j}{3}} = \te{cos}(\frac{4 \pi}{3}) + j \cdot \te{sin}(\frac{4 \pi}{3}) = -\frac{1}{2} - j \cdot \frac{\sqrt{3}}{2}
$$
$$
    \hat{a}^3 = e^{\frac{6 \pi j}{3}} = e^{2 \pi j} = 1
$$
$$
    \hat{a}^4 = e^{\frac{8 \pi j}{3}} = e^{\frac{2 \pi j}{3}} \cdot e^{\frac{6 \pi j}{3}} = e^{\frac{2 \pi j}{3}} = \hat{a}
$$

Poté fázory napětí můžeme vyjádřit jako:
$$
    \hat{U}_1 = \hat{U} \cdot \hat{a}
$$
$$
    \hat{U}_2 = \hat{U} \cdot \hat{a}^2
$$
$$
    \hat{U}_3 = \hat{U} \cdot \hat{a}
$$

Můžeme napsat rovnice pro proudy:
$$
    \hat{I}_1 = \hat{I}_{1,2} + \hat{I}_{1,3} = \hat{Y}_{1,2} \cdot \left( \hat{U}_1 - \hat{U}_2 \right) + \hat{Y}_{1,3} \cdot \left( \hat{U}_1 - \hat{U}_3 \right) =
$$
$$
    = j \cdot Y \cdot U \cdot \left( 1 - \hat{a}^2 \right) - j \cdot Y \cdot U \cdot \left( 1 - \hat{a} \right) = k \cdot U
$$

$$
    \hat{I}_2 = \hat{I}_{2,3} - \hat{I}_{1,2} = G \cdot \left( \hat{U}_2 - \hat{U}_3 \right) - \hat{Y}_{1,2} \cdot \left( \hat{U}_1 - \hat{U}_2 \right) =
$$
$$
    = G \cdot U \cdot \left( \hat{a}^2 - \hat{a} \right) - j \cdot Y \cdot U \cdot \left( 1 - \hat{a}^2 \right) = k \cdot U \cdot \hat{a}^2
$$

$$
    \hat{I}_3 = - \hat{I}_{1,3} - \hat{I}_{2,3} = - \hat{Y}_{1,3} \cdot \left( \hat{U}_1 - \hat{U}_3 \right) - G \cdot \left( \hat{U}_2 - \hat{U}_3 \right) =
$$
$$
    = j \cdot Y \cdot U \cdot \left( 1 - \hat{a} \right) - G \cdot U \cdot \left( \hat{a}^2 - \hat{a} \right) = k \cdot U \cdot \hat{a}
$$

Vezmeme konce rovnic, čímž dostaneme:
$$
    j \cdot Y \cdot U \cdot \left( 1 - \hat{a}^2 \right) - j \cdot Y \cdot U \cdot \left( 1 - \hat{a} \right) = k \cdot U
$$
$$
    G \cdot U \cdot \left( \hat{a}^2 - \hat{a} \right) - j \cdot Y \cdot U \cdot \left( 1 - \hat{a}^2 \right) = k \cdot U \cdot \hat{a}^2
$$
$$
    j \cdot Y \cdot U \cdot \left( 1 - \hat{a} \right) - G \cdot U \cdot \left( \hat{a}^2 - \hat{a} \right) = k \cdot U \cdot \hat{a}
$$

Rovnice jsou lineárně závislé, jelikož součet pravých stran je roven nule:
$$
    k \cdot U + k \cdot U \cdot \hat{a} + k \cdot U \cdot \hat{a}^2 = k \cdot U \cdot \left( 1 + \hat{a} + \hat{a}^2 \right) = k \cdot U \cdot 0 = 0
$$

Díky lineární závislosti nám stačí vzít pouze dvě rovnice. Vezmeme první a druhou:
$$
    j \cdot Y \cdot U \cdot (1 - \hat{a}^2) - j \cdot Y \cdot U \cdot (1 - \hat{a}) = k \cdot U
$$
$$
    G \cdot U \cdot (\hat{a}^2 - \hat{a}) - j \cdot Y \cdot U \cdot (1 - \hat{a}^2) = k \cdot U \cdot \hat{a}^2
$$

Můžeme pokrátit $U$:
$$
    j \cdot Y \cdot (1 - \hat{a}^2) - j \cdot Y \cdot (1 - \hat{a}) = k
$$
$$
    G \cdot (\hat{a}^2 - \hat{a}) - j \cdot Y \cdot (1 - \hat{a}^2) = k \cdot \hat{a}^2
$$

Připravíme rovnice na maticový tvar:
$$
    -k + \left( j \cdot (1 - \hat{a}^2) - j \cdot (1 - \hat{a}) \right) \cdot Y = 0
$$
$$
    -k \cdot \hat{a}^2 - j \cdot (1 - \hat{a}^2) \cdot Y = -G \cdot (\hat{a}^2 - \hat{a})
$$

Přepíšeme do maticového tvaru:
$$
    \begin{pmatrix}
        -1         & j (1 - \hat{a}^2) - j (1 - \hat{a}) \\
        -\hat{a}^2 & - j (1 - \hat{a}^2)
    \end{pmatrix}
    \cdot
    \begin{pmatrix}
        k \\
        Y
    \end{pmatrix}
    =
    \begin{pmatrix}
        0 \\
        -G (\hat{a}^2 - \hat{a})
    \end{pmatrix}
$$

$$
    \begin{pmatrix}
        -1         & j - j \hat{a}^2 - j + j \hat{a} \\
        -\hat{a}^2 & - j + j \hat{a}^2
    \end{pmatrix}
    \cdot
    \begin{pmatrix}
        k \\
        Y
    \end{pmatrix}
    =
    \begin{pmatrix}
        0 \\
        - G (\hat{a}^2 - \hat{a})
    \end{pmatrix}
$$

$$
    \begin{pmatrix}
        -1         & - j \hat{a}^2 + j \hat{a} \\
        -\hat{a}^2 & j \hat{a}^2 - j
    \end{pmatrix}
    \cdot
    \begin{pmatrix}
        k \\
        Y
    \end{pmatrix}
    =
    \begin{pmatrix}
        0 \\
        - G (\hat{a}^2 - \hat{a})
    \end{pmatrix}
$$

$$
    \begin{pmatrix}
        -\hat{a}^2 & - j \hat{a}^4 + j \hat{a}^3 \\
        -\hat{a}^2 & j \hat{a}^2 - j
    \end{pmatrix}
    \cdot
    \begin{pmatrix}
        k \\
        Y
    \end{pmatrix}
    =
    \begin{pmatrix}
        0 \\
        - G (\hat{a}^2 - \hat{a})
    \end{pmatrix}
$$

$$
    \begin{pmatrix}
        -\hat{a}^2 & - j \hat{a} + j \\
        -\hat{a}^2 & j \hat{a}^2 - j
    \end{pmatrix}
    \cdot
    \begin{pmatrix}
        k \\
        Y
    \end{pmatrix}
    =
    \begin{pmatrix}
        0 \\
        - G (\hat{a}^2 - \hat{a})
    \end{pmatrix}
$$

$$
    \begin{pmatrix}
        -\hat{a}^2 & - j \hat{a} + j                 \\
        0          & j \hat{a}^2 - j + j \hat{a} - j
    \end{pmatrix}
    \cdot
    \begin{pmatrix}
        k \\
        Y
    \end{pmatrix}
    =
    \begin{pmatrix}
        0 \\
        - G (\hat{a}^2 - \hat{a})
    \end{pmatrix}
$$

$$
    \begin{pmatrix}
        -\hat{a}^2 & - j \hat{a} + j               \\
        0          & j \hat{a}^2 + j \hat{a} - 2 j
    \end{pmatrix}
    \cdot
    \begin{pmatrix}
        k \\
        Y
    \end{pmatrix}
    =
    \begin{pmatrix}
        0 \\
        - G (\hat{a}^2 - \hat{a})
    \end{pmatrix}
$$

Vyřešíme rovnici pro $Y$:
$$
    Y = \frac{- G (\hat{a}^2 - \hat{a})}{j \hat{a}^2 + j \hat{a} - 2 j} = \frac{
        - G \left( -\frac{1}{2} - j \frac{\sqrt{3}}{2} - \left( -\frac{1}{2} + j \frac{\sqrt{3}}{2} \right) \right)
    }{
        j \left( -\frac{1}{2} - j \frac{\sqrt{3}}{2} \right) + j \left( -\frac{1}{2} + j \frac{\sqrt{3}}{2} \right) - 2 j =
    }
$$
$$
    = \frac{
        - G \left( -\frac{1}{2} - j \frac{\sqrt{3}}{2} + \frac{1}{2} - j \frac{\sqrt{3}}{2} \right)
    }{
        -j \frac{1}{2} + \frac{\sqrt{3}}{2} - j \frac{1}{2} - \frac{\sqrt{3}}{2} - 2 j
    } = \frac{
        - G \left( -j \sqrt{3} \right)
    }{
        -3 j
    } = \frac{- G \sqrt{3}}{3} = -\frac{G}{\sqrt{3}}
$$

Vyřešíme rovnici pro $k$:
$$
    -\hat{a}^2 k + (-j \hat{a} + j) Y = 0
$$
$$
    k = \frac{- (-j \hat{a} + j) Y}{-\hat{a}^2} = \frac{(-j \hat{a} + j) Y}{\hat{a}^2} = \frac{(-j \hat{a} + j) \left( -\frac{G}{\sqrt{3}} \right)}{\hat{a}^2} =
$$
$$
    = \frac{
        \left( -j \left( -\frac{1}{2} + j \frac{\sqrt{3}}{2} \right) + j \right) \left( -\frac{G}{\sqrt{3}} \right)
    }{
        -\frac{1}{2} - j \frac{\sqrt{3}}{2}
    } = \frac{
        \left( j \frac{1}{2} + \frac{\sqrt{3}}{2} + j \right) \left( -\frac{G}{\sqrt{3}} \right)
    }{
        -\frac{1}{2} - j \frac{\sqrt{3}}{2}
    } =
$$
$$
    = \frac{
        \left( j \frac{3}{2} + \frac{\sqrt{3}}{2} \right) \left( -\frac{G}{\sqrt{3}} \right)
    }{
        -\frac{1}{2} - j \frac{\sqrt{3}}{2}
    } = \frac{
        \left( - j \frac{3}{2 \sqrt{3}} - \frac{\sqrt{3}}{2 \sqrt{3}} \right) G
    }{
        -\frac{1}{2} - j \frac{\sqrt{3}}{2}
    } =
$$
$$
    = \frac{
        \left( -\frac{1}{2} - j \frac{\sqrt{3}}{2} \right) G
    }{
        -\frac{1}{2} - j \frac{\sqrt{3}}{2}
    } = G
$$

$$
    k = G
$$
$$
    Y = -\frac{G}{\sqrt{3}}
$$


\subsection{Obecná 3f nesymetrická zátěž}
Mějme obecnou 3 fázovou nesymetrickou zátěž zadanou admitancemi podle obrázku:

\begin{center}
    \begin{circuitikz}[scale=0.8][european voltages]
        \draw
        (0,0)
        to [fullgeneric, *-, l_=$\hat{Y}_{1,3}$] (3,5)
        to [fullgeneric, *-, l_=$\hat{Y}_{1,2}$] (6,0)
        to [fullgeneric, *-, l_=$\hat{Y}_{2,3}$] (0,0);

        \node[anchor=west] at (3,5) {1};
        \node[anchor=west] at (6,0) {2};
        \node[anchor=east] at (0,0) {3};
    \end{circuitikz}
\end{center}

Admitance můžeme zapsat jako:
$$
    \hat{Y}_{1,2} = G_{1,2} + j \cdot B_{1,2}
$$
$$
    \hat{Y}_{1,3} = G_{1,3} + j \cdot B_{1,3}
$$
$$
    \hat{Y}_{2,3} = G_{2,3} + j \cdot B_{2,3}.
$$

První krok je provést kompenzaci jalových částí. Stačí pouze vzít zápornou hodnotu jalové části zátěže:

\begin{center}
    \begin{circuitikz}[scale=0.8][european voltages]
        \draw
        (0,0)
        to [fullgeneric, *-, l_=$-B_{1,2}$] (3,5)
        to [fullgeneric, *-, l_=$-B_{1,3}$] (6,0)
        to [fullgeneric, *-, l_=$-B_{2,3}$] (0,0);

        \node[anchor=west] at (3,5) {1};
        \node[anchor=west] at (6,0) {2};
        \node[anchor=east] at (0,0) {3};
    \end{circuitikz}
\end{center}

Dále je třeba provést symetrizaci pro každou část zvlášť. Nejprve pro větev 1--2, poté pro větev 1--3 a nakonec pro větev 2--3. Tento krok je znázorněn na obrázku:

\begin{figure}[H]
    \centering
    \subfloat{
        \begin{circuitikz}[scale=0.4][european voltages]
            \draw
            (0,0) to [short, *-] (0,0)
            (3,5) to [fullgeneric, *-, l_=$\hat{Y}_{1,2}$]
            (6,0) to [short, *-] (6,0);

            \node[anchor=west] at (3,5) {1};
            \node[anchor=west] at (6,0) {2};
            \node[anchor=east] at (0,0) {3};
        \end{circuitikz}
    }
    \qquad
    \subfloat{
        \begin{circuitikz}[scale=0.4][european voltages]
            \draw
            (0,0) to [fullgeneric, *-, l_=$\hat{Y}_{1,3}$] (3,5)
            (3,5) to [short, *-] (3,5)
            (6,0) to [short, *-] (6,0);

            \node[anchor=west] at (3,5) {1};
            \node[anchor=west] at (6,0) {2};
            \node[anchor=east] at (0,0) {3};
        \end{circuitikz}
    }
    \qquad
    \subfloat{
        \begin{circuitikz}[scale=0.4][european voltages]
            \draw
            (3,5) to [short, *-] (3,5)
            (0,0) to [short, *-] (0,0)
            (6,0) to [fullgeneric, *-, l_=$\hat{Y}_{2,3}$] (0,0);

            \node[anchor=west] at (3,5) {1};
            \node[anchor=west] at (6,0) {2};
            \node[anchor=east] at (0,0) {3};
        \end{circuitikz}
    }
\end{figure}

Výsledná tabulka symetrizace bude vypadat následovně:
\begin{center}
    \begin{tabular}{l c c c}
        \hline
        Větev                      & 1--2                          & 1--3                          & 2--3                          \\
        \hline                                                                                                                     \\
        Kompenzace jalového výkonu & $-j B_{1,2}$                  & $-j B_{1,3}$                  & $-j B_{2,3}$                  \\~\\
        Symetrizace 1--2           & 0                             & $-j \frac{G_{1,2}}{\sqrt{3}}$ & $j \frac{G_{1,2}}{\sqrt{3}}$  \\~\\
        Symetrizace 1--3           & $j \frac{G_{1,3}}{\sqrt{3}}$  & 0                             & $-j \frac{G_{1,3}}{\sqrt{3}}$ \\~\\
        Symetrizace 2--3           & $-j \frac{G_{2,3}}{\sqrt{3}}$ & $j \frac{G_{2,3}}{\sqrt{3}}$  & 0                             \\~\\
        \hline
    \end{tabular}
\end{center}

Symetrizační admitanci pro danou větev dostaneme jako součet všech symetrizačních admitancí (suma ve sloupci):
$$
    \hat{Y}_{s,1,2} = -j B_{1,2} + j \frac{G_{1,3}}{\sqrt{3}} - j \frac{G_{2,3}}{\sqrt{3}}
$$
$$
    \hat{Y}_{s,1,3} = -j B_{1,3} - j \frac{G_{1,2}}{\sqrt{3}} + j \frac{G_{2,3}}{\sqrt{3}}
$$
$$
    \hat{Y}_{s,2,3} = -j B_{2,3} + j \frac{G_{1,2}}{\sqrt{3}} - j \frac{G_{1,3}}{\sqrt{3}}
$$

Tyto admitance následně připojíme parallelně k odpovídajícím větvím. Výsledný obvod bude vypadat následovně:

\begin{center}
    \begin{circuitikz}[scale=0.8][european voltages]
        \draw
        (0,0)
        to [fullgeneric, *-, l_=$\hat{Y}_{1,3}$] (3,5)
        to [fullgeneric, *-, l_=$\hat{Y}_{1,2}$] (6,0)
        to [fullgeneric, *-, l_=$\hat{Y}_{2,3}$] (0,0)

        (0,0) to [short, -] (-5/3,1)
        (-5/3,1) to [fullgeneric, -, l_=$\hat{Y}_{s,1,3}$] (4/3,6)
        (4/3,6) to [short, -] (3,5)

        (0,0) to [short, -] (0,-2)
        (0,-2) to [fullgeneric, -, l_=$\hat{Y}_{s,2,3}$] (6,-2)
        (6,-2) to [short, -] (6,0)

        (3,5) to [short, -] (14/3,6)
        (14/3,6) to [fullgeneric, -, l_=$\hat{Y}_{s,1,2}$] (23/3,1)
        (23/3,1) to [short, -] (6,0);

        \node[anchor=south] at (3,5) {1};
        \node[anchor=west] at (6,0) {2};
        \node[anchor=east] at (0,0) {3};
    \end{circuitikz}
\end{center}


\subsection{Přepočet výkonů na admitance}
Zátěže jsou často zadány pomocí: činného výkonu, úhlu $\te{cos} (\varphi) \fs$ a informací, zda je zátěž induktivní nebo kapacitní. Tyto informace můžeme převést na admitanci $\hat{Y} \fs (\uOHMinv)$ následovně:
$$
    \hat{Y} = \frac{P}{U^2} \cdot (1 - j \cdot \te{tg} (\pm \varphi)) = \frac{P}{U^2} \cdot (1 \pm j \cdot \te{tg} (\varphi)),
$$
kde:
$P$ - činný výkon (\ueqW),\\
$U$ - efektivní hodnota napětí (\ueqV),\\
$j$ - imaginární jednotka,\\
$\varphi$ - úhel $\te{cos} (\varphi)$,\\
$+$ - induktivní zátěž,\\
$-$ - kapacitní zátěž.

\subsubsection{Odvození \spicy \spicy \spicy}



\subsection{Číselný příklad}
Mějme 3 fázovou nesymetrickou zátěž nazančenou na obrázku:

\begin{center}
    \begin{circuitikz}[scale=0.8][european voltages]
        \draw
        (0,0)
        to [fullgeneric, *-, l_=$P_{1,3}$] (3,5)
        to [fullgeneric, *-, l_=$P_{1,2}$] (6,0)
        to [fullgeneric, *-, l_=$P_{2,3}$] (0,0);
    \end{circuitikz}
\end{center}

Parametry:
\begin{itemize}
    \item $U = 400 \uV$,
    \item $\te{cos} (\varphi) = 0.8$,
    \item $P_{1,2} = 63 \fs \uKW$, induktivní,
    \item $P_{1,3} = 28 \fs \uKW$, induktivní,
    \item $P_{2,3} = 26 \fs \uKW$, kapacitní.
\end{itemize}

Provete symetrizaci zátěže.

\subsubsection{Řešení}
Nejprve získáme úhel $\varphi$:
$$
    \varphi = \te{arccos} (0.8) \approx 0.644 \fs \te{rad}.
$$

Dále vypočítáme $\te{tg} (\varphi)$:
$$
    \te{tg} (\varphi) = \te{tg} (0.644) \approx 0.751.
$$

Následně získáme admitance:
$$
    Y_{1,2} = \frac{P_{1,2}}{U^2} \cdot (1 + j \cdot \te{tg} (\varphi)) = \frac{63 \fs 000}{400^2} \cdot (1 + j \cdot 0.751) = (0.394 + j \cdot 0.236) \fs \uOHMinv
$$
$$
    Y_{1,3} = \frac{P_{1,3}}{U^2} \cdot (1 + j \cdot \te{tg} (\varphi)) = \frac{28 \fs 000}{400^2} \cdot (1 + j \cdot 0.751) = (0.175 + j \cdot 0.131) \fs \uOHMinv
$$
$$
    Y_{2,3} = \frac{P_{2,3}}{U^2} \cdot (1 - j \cdot \te{tg} (\varphi)) = \frac{26 \fs 000}{400^2} \cdot (1 - j \cdot 0.751) = (0.163 - j \cdot 0.122) \fs \uOHMinv.
$$

Dále vytvoříme tabulku symetrizace:

\begin{center}
    \begin{tabular}{l l l l}
        \hline
        Větev                      & 1--2                        & 1--3                        & 2--3                        \\
        \hline                                                                                                               \\
        Kompenzace jalového výkonu & $-j 0.236$                  & $-j 0.131$                  & $j 0.122$                   \\~\\
        Symetrizace 1--2           & 0                           & $-j \frac{0.394}{\sqrt{3}}$ & $j \frac{0.394}{\sqrt{3}}$  \\~\\
        Symetrizace 1--3           & $j \frac{0.175}{\sqrt{3}}$  & 0                           & $-j \frac{0.175}{\sqrt{3}}$ \\~\\
        Symetrizace 2--3           & $-j \frac{0.163}{\sqrt{3}}$ & $j \frac{0.163}{\sqrt{3}}$  & 0                           \\~\\
        \hline
    \end{tabular}
\end{center}

Symetrizační admitance:
$$
    Y_{s,1,2} = -j 0.236 + j \frac{0.175}{\sqrt{3}} - j \frac{0.163}{\sqrt{3}} = -j 0.236 + j 0.101 - j 0.094 = -j 0.229 \fs \uOHMinv
$$
$$
    Y_{s,1,3} = -j 0.131 - j \frac{0.394}{\sqrt{3}} + j \frac{0.163}{\sqrt{3}} = -j 0.131 - j 0.227 + j 0.094 = -j 0.264 \fs \uOHMinv
$$
$$
    Y_{s,2,3} = j 0.122 + j \frac{0.394}{\sqrt{3}} - j \frac{0.175}{\sqrt{3}} = j 0.122 + j 0.227 - j 0.101 = j 0.248 \fs \uOHMinv.
$$

Výsledné zapojení bude vypadat následovně:

\begin{center}
    \begin{circuitikz}[scale=0.8][european voltages]
        \draw
        (0,0)
        to [fullgeneric, *-, l_=$\hat{Y}_{1,3}$] (3,5)
        to [fullgeneric, *-, l_=$\hat{Y}_{1,2}$] (6,0)
        to [fullgeneric, *-, l_=$\hat{Y}_{2,3}$] (0,0)

        (0,0) to [short, -] (-5/3,1)
        (-5/3,1) to [cute inductor, -, l_=$\hat{Y}_{s,1,3}$] (4/3,6)
        (4/3,6) to [short, -] (3,5)

        (0,0) to [short, -] (0,-2)
        (0,-2) to [capacitor, -, l_=$\hat{Y}_{s,2,3}$] (6,-2)
        (6,-2) to [short, -] (6,0)

        (3,5) to [short, -] (14/3,6)
        (14/3,6) to [cute inductor, -, l_=$\hat{Y}_{s,1,2}$] (23/3,1)
        (23/3,1) to [short, -] (6,0);

        \node[anchor=south] at (3,5) {1};
        \node[anchor=west] at (6,0) {2};
        \node[anchor=east] at (0,0) {3};
    \end{circuitikz}
\end{center}

\end{document}
