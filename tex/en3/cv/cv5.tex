\documentclass{article}
\usepackage[czech]{babel} % Czech language
\usepackage[shortlabels]{enumitem} % Custom enumeration
\usepackage{graphicx} % Import images
\usepackage{float} % Use [H] to force figure position
\usepackage{indentfirst} % Indent first paragraph
\usepackage{emoji} % Emojis
\usepackage{pgffor} % Loops
\usepackage{tikz} % TikZ
\usepackage{pgfplots} % TikZ plots
\usepackage{circuitikz} % Use circuitikz for circuit diagrams
\usepackage{amsmath} % Math
\usetikzlibrary{arrows.meta}

\makeatletter
\providecommand\add@text{}
\newcommand\tagaddtext[1]{%
    \gdef\add@text{#1\gdef\add@text{}}}% 
\renewcommand\tagform@[1]{%
    \maketag@@@{\llap{\add@text\quad}(\ignorespaces#1\unskip\@@italiccorr)}%
}
\makeatother

\newcommand{\cvHead}[1]{\head{Cvičení #1}}

\newcommand{\head}[1]{
    \title{\textbf{#1}\\Elektroenergetika 3}
    \author{Petr Jílek}
    \date{2024}
}

\newcommand{\spicy}{\emoji{hot-pepper}}

% My font space
\newcommand{\myFS}{\;}

% Text in math mode
\newcommand{\te}[1]{\textrm{#1}}

% --------------------
% Units
% --------------------

\newcommand{\uM}{\textrm{m}} % Meter
\newcommand{\uMsq}{\uM^\textrm{2}} % Meter squared
\newcommand{\uMcu}{\uM^\textrm{3}} % Meter cubed
\newcommand{\uS}{\textrm{s}} % Second
\newcommand{\uKG}{\textrm{kg}} % Kilogram
\newcommand{\uJ}{\textrm{J}} % Joule
\newcommand{\uK}{\textrm{K}} % Kelvin
\newcommand{\uDEGREE}{^\circ} % Degree
\newcommand{\uCELS}{\uDEGREE \textrm{C}} % Celsius
\newcommand{\uW}{\textrm{W}} % Watt
\newcommand{\uMW}{\textrm{MW}} % Mega Watt
\newcommand{\uKW}{\textrm{kW}} % Kilo Watt
\newcommand{\uWH}{\textrm{Wh}} % Watt hour
\newcommand{\uKWH}{\textrm{kWh}} % Kilo Watt hour
\newcommand{\uCCY}{\textrm{CCY}} % Currency
\newcommand{\uCZK}{\textrm{CZK}} % Czech Crown
\newcommand{\uPERCENT}{\textrm{\%}} % Percent
\newcommand{\uNOUNIT}{\textrm{--}} % No unit
\newcommand{\uYEAR}{\textrm{year}} % Year
\newcommand{\uMONTH}{\textrm{month}} % Month
\newcommand{\uHOUR}{\textrm{hour}} % Hour
\newcommand{\uLM}{\textrm{lm}} % Lumen
\newcommand{\uHinv}{\textrm{h}^{-1}} % Hour inverse

\newcommand{\uJperK}{\uJ / \uK} % Joule per Kelvin
\newcommand{\uKperM}{\uK / \uM} % Kelvin per Meter
\newcommand{\uKperS}{\uK / \uS} % Kelvin per Second
\newcommand{\uMsqperS}{\uMsq/\uS} % Meter squared per Second
\newcommand{\uWperMsq}{\uW / \uMsq} % Watt per Meter squared
\newcommand{\uWperMperK}{\uW / \left( \uM \cdot \uK \right)} % Watt per Meter per Kelvin
\newcommand{\uJperKGperK}{\uJ / \left( \uKG \cdot \uK \right)} % Joule per Kilogram per Kelvin
\newcommand{\uKperMsq}{\uK / \uMsq} % Kelvin per Meter squared
\newcommand{\uJperSperMperK}{\uJ / \left( \uS \cdot \uM \cdot \uK \right)} % Joule per Second per Meter per Kelvin
\newcommand{\uKGperMcu}{\uKG / \uMcu} % Kilogram per Meter cubed
\newcommand{\uMsqKperW}{\uMsq \cdot \uK / \uW} % Meter squared Kelvin per Watt
\newcommand{\uWperMsqperK}{\uW / \left( \uMsq \cdot \uK \right)} % Watt per Meter squared per Kelvin
\newcommand{\uKWHperMsqperYEAR}{\uKWH / \left( \uMsq \cdot \uYEAR \right)} % Kilo Watt hour per Meter squared per Year
\newcommand{\uKWHperMsq}{\uKWH / \uMsq} % Kilo Watt hour per Meter squared
\newcommand{\uLMperW}{\uLM / \uW} % Lumen per Watt
\newcommand{\uKWperMsq}{\uKW / \uMsq} % Kilo Watt per Meter squared
\newcommand{\uCCYperYEAR}{\uCCY / \uYEAR} % Currency per Year
\newcommand{\uKWHperYEAR}{\uKWH / \uYEAR} % Kilo Watt hour per Year
\newcommand{\uNOUNITperYEAR}{\uNOUNIT / \uYEAR} % No unit per Year
\newcommand{\uCCYperKWH}{\uCCY / \uKWH} % Currency per Kilo Watt hour
\newcommand{\uCZKperYEAR}{\uCZK / \uYEAR} % Czech Crown per Year
\newcommand{\uCZKperKWH}{\uCZK / \uKWH} % Czech Crown per Kilo Watt hour
\newcommand{\uCZKperMONTH}{\uCZK / \uMONTH} % Czech Crown per Month



% --------------------
% Unit equations
% --------------------

\newcommand{\ueqM}{$\uM$}
\newcommand{\ueqMsq}{$\uMsq$}
\newcommand{\ueqMcu}{$\uMcu$}
\newcommand{\ueqS}{$\uS$}
\newcommand{\ueqJ}{$\uJ$}
\newcommand{\ueqK}{$\uK$}
\newcommand{\ueqDEGREE}{$\uDEGREE$}
\newcommand{\ueqCELS}{$\uCELS$}
\newcommand{\ueqW}{$\uW$}
\newcommand{\ueqMW}{$\uMW$}
\newcommand{\ueqKW}{$\uKW$}
\newcommand{\ueqWH}{$\uWH$}
\newcommand{\ueqKWH}{$\uKWH$}
\newcommand{\ueqCCY}{$\uCCY$}
\newcommand{\ueqCZK}{$\uCZK$}
\newcommand{\ueqPERCENT}{$\uPERCENT$}
\newcommand{\ueqNOUNIT}{$\uNOUNIT$}
\newcommand{\ueqYEAR}{$\uYEAR$}
\newcommand{\ueqMONTH}{$\uMONTH$}
\newcommand{\ueqHOUR}{$\uHOUR$}
\newcommand{\ueqLM}{$\uLM$}
\newcommand{\ueqHinv}{$\uHinv$}

\newcommand{\ueqJperK}{$\uJperK$}
\newcommand{\ueqKperM}{$\uKperM$}
\newcommand{\ueqKperS}{$\uKperS$}
\newcommand{\ueqMsqperS}{$\uMsqperS$}
\newcommand{\ueqWperMsq}{$\uWperMsq$}
\newcommand{\ueqWperMperK}{$\uWperMperK$}
\newcommand{\ueqJperKGperK}{$\uJperKGperK$}
\newcommand{\ueqKperMsq}{$\uKperMsq$}
\newcommand{\ueqJperSperMperK}{$\uJperSperMperK$}
\newcommand{\ueqKGperMcu}{$\uKGperMcu$}
\newcommand{\ueqMsqKperW}{$\uMsqKperW$}
\newcommand{\ueqWperMsqperK}{$\uWperMsqperK$}
\newcommand{\ueqKWHperMsqperYEAR}{$\uKWHperMsqperYEAR$}
\newcommand{\ueqKWHperMsq}{$\uKWHperMsq$}
\newcommand{\ueqLMperW}{$\uLMperW$}
\newcommand{\ueqKWperMsq}{$\uKWperMsq$}
\newcommand{\ueqCCYperYEAR}{$\uCCYperYEAR$}
\newcommand{\ueqKWHperYEAR}{$\uKWHperYEAR$}
\newcommand{\ueqNOUNITperYEAR}{$\uNOUNITperYEAR$}
\newcommand{\ueqCCYperKWH}{$\uCCYperKWH$}
\newcommand{\ueqCZKperYEAR}{$\uCZKperYEAR$}
\newcommand{\ueqCZKperKWH}{$\uCZKperKWH$}
\newcommand{\ueqCZKperMONTH}{$\uCZKperMONTH$}


\cvHead{5 - Aplikace}

\begin{document}

\maketitle
\tableofcontents
\newpage




\section{\emoji{snowflake} Topná sezóna \spicy \spicy \spicy}
Průměrná venkonví teplota v topné sezóně je $\overline{T}_{out} = 5 \fs \uCELS$. Vnitřní teplota je $T_{in} = 20 \fs \uCELS$. Doba topné sezóny je 200 dní. Celková plocha je $S = 300 \fs \uMsq$. Součinitel prostupu tepla je $U_\vartheta = 0,5 \fs \uWandMinvsqKinv$. Jaký množství tepla projde stěnami za celou topnou sezónu?



\subsection{Řešení}
Celkové množství tepla, které projde stěnami za celou topnou sezónu $Q$ vypočteme jako:
$$
    Q = \int_{t_1}^{t_2} \dot{Q} \cdot dt,
$$
kde:\\
$\dot{Q}$ je tepelný tok (\ueqW),\\
$t_1$ je začátek topné sezóny (\uH),\\
$t_2$ je konec topné sezóny (\uH).\\

Integrál můžeme rozepsat jako:
$$
    Q = \int_{t_1}^{t_2} U_\vartheta \cdot S \cdot (T_{in} - T_{out} (t)) \cdot dt = U_\vartheta \cdot S \cdot \int_{t_1}^{t_2} (T_{in} - T_{out} (t)) \cdot dt =
$$
$$
    = U_\vartheta \cdot S \cdot T_{in} \cdot \int_{t_1}^{t_2} dt - U_\vartheta \cdot S \cdot \int_{t_1}^{t_2} T_{out} (t) \cdot dt =
$$
$$
    = U_\vartheta \cdot S \cdot T_{in} \cdot (t_2 - t_1) - U_\vartheta \cdot S \cdot \int_{t_1}^{t_2} T_{out} (t) \cdot dt.
$$

Nyní uděláme odbočku, kde vyjádříme průměrnou venkovní teplotu $\overline{T}_{out}$ jako:
$$
    \overline{T}_{out} = \frac{1}{t_2 - t_1} \cdot \int_{t_1}^{t_2} T_{out} (t) \cdot dt.
$$

Vyjádříme daný integrál:
$$
    \int_{t_1}^{t_2} T_{out} (t) \cdot dt = \overline{T}_{out} \cdot (t_2 - t_1).
$$

Dosadíme zpět do původní rovnice:
$$
    Q = U_\vartheta \cdot S \cdot T_{in} \cdot (t_2 - t_1) - U_\vartheta \cdot S \cdot \overline{T}_{out} \cdot (t_2 - t_1) =
$$
$$
    = U_\vartheta \cdot S \cdot (T_{in} - \overline{T}_{out}) \cdot (t_2 - t_1).
$$

Nyní můžeme dosadit a vypočítat:
$$
    Q = 0,5 \cdot 300 \cdot (20 - 5) \cdot (200 \cdot 24 \cdot 3 \fs 600 - 0) =
$$
$$
    = 0,5 \cdot 300 \cdot 15 \cdot 200 \cdot 24 \cdot 3 \fs 600 = 38,9 \fs \uGJ.
$$

\newpage




\section{\emoji{factory} Cihlová pec \spicy \spicy \spicy}
U cihlové pece je vysoký rozdíl teplot, tudíž tepelnou vodivost je třeba uvažovat jako proměnnou. Tepelnou vodivost $\lambda$ můžeme zapsat jako:
$$
    \lambda (x) = \lambda_0 + \lambda_1 \cdot T,
$$

Pro zeď o tloušťce $d$ budeme uvažovat následující zjednodušející předpoklady:
\begin{itemize}
    \item $\frac{\partial T}{\partial t}$ = 0,
    \item $\vec{v}$ = 0,
    \item $\dot{Q}_V$ = 0,
    \item $\lambda (x) = \lambda_0 + \lambda_1 \cdot T$,
    \item $T = T(x)$ - teplota závislá pouze na ose x.
\end{itemize}

Poté Fourierova-Kirchhoffova rovnice bude mít tvar:
$$
    0 = \frac{d}{dx} \left ( \lambda (T) \cdot \frac{dT}{dx} \right ).
$$

Fourieruv zákon pro tepelný tok bude mít tvar:
$$
    \dot{q} = - \lambda (T) \cdot \frac{dT}{dx}.
$$

Vynásobme rovnici mínus jedna:
$$
    - \dot{q} = \lambda (T) \cdot \frac{dT}{dx}.
$$

Pokud dosadíme do Fourierova-Kirchhoffovy rovnice, dostaneme:
$$
    0 = \frac{d}{dx} \left ( - \dot{q} \right ).
$$

Z toho vidíme, že měrný tepelný tok $\dot{q}$ je konstantní a nezávisí na poloze x. Můžeme tedy rozepsat Fourierův zákon jako:
$$
    - \dot{q} = \left( \lambda_0 + \lambda_1 \cdot T \right) \cdot \frac{dT}{dx}.
$$

Můžeme řešit tuto diferenciální rovnici separací proměnných:
$$
    - \dot{q} \cdot dx = \left( \lambda_0 + \lambda_1 \cdot T \right) \cdot dT.
$$

Nyní meze budou jak pro x, tak pro T:
\begin{itemize}
    \item $x = 0 \Rightarrow T = T_1$,
    \item $x = d \Rightarrow T = T_2$.
\end{itemize}

Rovnici můžeme integrovat:
$$
    - \int_{0}^{d} \dot{q} \cdot dx = \int_{T_1}^{T_2} \left( \lambda_0 + \lambda_1 \cdot T \right) \cdot dT
$$
$$
    - \dot{q} \cdot \left[ x \right]_{0}^{d} = \lambda_0 \cdot \left[ T \right]_{T_1}^{T_2} + \frac{\lambda_1}{2} \cdot \left[ T^2 \right]_{T_1}^{T_2}
$$
$$
    - \dot{q} \cdot d = \lambda_0 \cdot (T_2 - T_1) + \frac{\lambda_1}{2} \cdot (T_2^2 - T_1^2)
$$
$$
    - \dot{q} \cdot d = \lambda_0 \cdot (T_2 - T_1) + \frac{\lambda_1}{2} \cdot (T_2 - T_1) \cdot (T_2 + T_1)
$$
$$
    - \dot{q} \cdot d = (T_2 - T_1) \cdot \left( \lambda_0 + \frac{\lambda_1}{2} \cdot (T_2 + T_1) \right).
$$

Nyní odvoďme střední hodnotu tepelné vodivosti $\overline{\lambda}$:
$$
    \overline{\lambda} = \frac{1}{T_2 - T_1} \cdot \int_{T_1}^{T_2} \lambda (T) \cdot dT = \frac{1}{T_2 - T_1} \cdot \int_{T_1}^{T_2} \left( \lambda_0 + \lambda_1 \cdot T \right) \cdot dT
$$
$$
    = \frac{1}{T_2 - T_1} \cdot \left[ \lambda_0 \cdot T + \frac{\lambda_1}{2} \cdot T^2 \right]_{T_1}^{T_2} = \frac{1}{T_2 - T_1} \cdot \left( \lambda_0 \cdot (T_2 - T_1) + \frac{\lambda_1}{2} \cdot (T_2^2 - T_1^2) \right) =
$$
$$
    = \lambda_0 + \frac{\lambda_1}{2} \cdot (T_2 + T_1).
$$

Nyní můžeme videt, že člen v závorce u vyřešené diferenciální rovnice je roven střední hodnotě tepelné vodivosti $\overline{\lambda}$:
$$
    - \dot{q} \cdot d = (T_2 - T_1) \cdot \overline{\lambda}.
$$

Nyní můžeme odvodit vztah pro měrný tepelný tok $\dot{q}$:
$$
    \dot{q} = \frac{T_1 - T_2}{d} \cdot \overline{\lambda}.
$$

\newpage




\section{\emoji{shower} Průtokový ohřívač \spicy \spicy \spicy \spicy}
Máme průtokový ohřívač tvořený dvěma obdelníkovými elektrodovými deskami o délce $l \fs (\uM)$ a šířce $b \fs (\uM)$. Vzdálenost mezi deskami je $d \fs (\uM)$. Mezi deskami je voda produdící o rychlostí $v$ pouze ve směru $x$. Voda má měrnou tepelnou kapacitu $c \fs (\uJandKGinvKinv)$ a hustotu $\rho \fs (\uKGandMinvcu)$. Elektrické napětí mezi deskami je $U \fs (\uV)$. Odvoďte změnu teploty vody $T_2 \fs (\uK)$ na výstupu průtokového ohřívače, pokud na vstupu je voda o teplotě $T_1 \fs (\uK)$.\\

Budou platit následující předpoklady:
\begin{itemize}
    \item $\frac{\partial T}{\partial t}$ = 0,
    \item $\nabla \cdot (\lambda \cdot \vec{\nabla} T)$ = 0.
\end{itemize}



\subsection{Řešení}
Fourier-Kirchhoffova rovnice:
\begin{equation}
    \rho \cdot c \cdot \vec{v} \cdot \vec{\nabla} T = \dot{Q}_V,
    \unit{(\ueqWandMinvcu)}
\end{equation}
kde:\\
$\rho$ -- hustota (\ueqKGandMinvcu),\\
$c$ -- měrná tepelná kapacita (\ueqJandKGinvKinv),\\
$\vec{v}$ -- rychlost (\ueqMandSinv),\\
$\vec{\nabla} T$ -- gradient teploty (\ueqKandMinv),\\
$\dot{Q}_V$ -- objemový zdroj tepla (\ueqWandMinvcu).\\

Vektor rychlosti $\vec{v}$ je:
$$
    \vec{v} = \begin{pmatrix} v \\ 0 \\ 0 \end{pmatrix}.
$$

Vektor rychlosti $\vec{v}$ má pouze složku $v$, protože voda teče pouze ve směru osy $x$. Můžeme tedy rovnici zjednodušit na:
$$
    \rho \cdot c \cdot v \cdot \frac{dT}{dx} = \dot{Q}_V.
$$

Dále je třeba vyjádřit rychlost $v$ a objemový zdroj tepla $\dot{Q}_V$. Rychlost $v$ je:
$$
    v = \frac{\dot{V}}{S} = \frac{\dot{V}}{d \cdot b},
$$
kde:\\
$\dot{V}$ -- objemový průtok (\ueqMcuSinv).\\

Objemový zdroj tepla $\dot{Q}_V$ můžeme vyjádřit pomocí intenzity elektrického pole $E$ a proudové hustoty $J$:
$$
    \dot{Q}_V = E \cdot J,
$$
kde:\\
$E$ -- intenzita elektrického pole (\ueqVandMinv),\\
$J$ -- proudová hustota (\ueqAandMinvsq).\\

Proudovou hustotu $J$ můžeme vyjádřit pomocí Ohmova zákona v diferenciálním tvaru:
$$
    J = \sigma_e \cdot E,
$$
kde:\\
$\sigma_e$ -- měrná elektrická vodivost (\ueqOHMandMinv).\\

Dostaneme tedy:
$$
    \dot{Q}_V = E \cdot \sigma_e \cdot E = \sigma_e \cdot E^2.
$$

Intenzitu elektrického pole $E$ můžeme vyjádřit pomocí napětí $U$ a vzdálenosti mezi deskami $d$:
$$
    E = \frac{U}{d}.
$$

Dostaneme tedy:
$$
    \dot{Q}_V = \sigma_e \cdot \left( \frac{U}{d} \right)^2.
$$

Zde je třeba uvažovat měrnou elektrickou vodivost $\sigma_e$ jako proměrnou závislou na teplotě. U kovů se vzrůstající teplotou se měrná elektrická vodivost snižuje, jelikož volné nostiče náboje mají problém se v rozpohybované krystalové mřížce kvůli teplotě pohybovat. Naopak u iontových roztoku jako třeba u vody se měrná elektrická vodivost zvyšuje s teplotou. Nahradíme tedy $\sigma_e$ lineární funkcí teploty $\sigma_e(T)$:
$$
    \sigma_e = \sigma_{e,0} + \sigma_{e,1} \cdot T.
$$

Nyní můžeme dosadit vše do rovnice:
$$
    \rho \cdot c \cdot \frac{\dot{V}}{d \cdot b} \cdot \frac{dT}{dx} = \left( \sigma_{e,0} + \sigma_{e,1} \cdot T \right) \cdot \left( \frac{U}{d} \right)^2.
$$

Dostáváme diferenciální rovnici, kterou můžeme řešit separací proměnných:
$$
    \frac{dT}{\sigma_{e,0} + \sigma_{e,1} \cdot T} = \frac{U^2 \cdot b}{d \cdot \rho \cdot c \cdot \dot{V}} \cdot dx.
$$

Pravou staranu můžeme pro lepší čitelnost narhadit:
$$
    \xi = \frac{U^2 \cdot b}{d \cdot \rho \cdot c \cdot \dot{V}}.
$$

Dostáváme:
$$
    \frac{dT}{\sigma_{e,0} + \sigma_{e,1} \cdot T} = \xi \cdot dx.
$$

Integrujeme obě strany:
$$
    \int_{T_1}^{T_2} \frac{dT}{\sigma_{e,0} + \sigma_{e,1} \cdot T} = \int_{0}^{l} \xi \cdot dx,
$$
kde:\\
$T_1$ -- počáteční teplota (\ueqK),\\
$T_2$ -- konečná teplota (\ueqK),\\
$l$ -- délka průtokového ohřívače (\ueqM).\\

Dostáváme:
$$
    \left[ \frac{1}{\sigma_{e,1}} \cdot \ln \left( \sigma_{e,0} + \sigma_{e,1} \cdot T \right) \right]_{T_1}^{T_2} = \left[ \xi \cdot x \right]_0^l
$$
$$
    \frac{1}{\sigma_{e,1}} \cdot \ln \left( \sigma_{e,0} + \sigma_{e,1} \cdot T_2 \right) - \frac{1}{\sigma_{e,1}} \cdot \ln \left( \sigma_{e,0} + \sigma_{e,1} \cdot T_1 \right) = \xi \cdot l
$$
$$
    \ln \left( \frac{\sigma_{e,0} + \sigma_{e,1} \cdot T_2}{\sigma_{e,0} + \sigma_{e,1} \cdot T_1} \right) = \sigma_{e,1} \cdot \xi \cdot l
$$
$$
    \frac{\sigma_{e,0} + \sigma_{e,1} \cdot T_2}{\sigma_{e,0} + \sigma_{e,1} \cdot T_1} = e^{\sigma_{e,1} \cdot \xi \cdot l}
$$
$$
    \sigma_{e,0} + \sigma_{e,1} \cdot T_2 = \left( \sigma_{e,0} + \sigma_{e,1} \cdot T_1 \right) \cdot e^{\sigma_{e,1} \cdot \xi \cdot l}
$$
$$
    T_2 = \frac{\left( \sigma_{e,0} + \sigma_{e,1} \cdot T_1 \right) \cdot e^{\sigma_{e,1} \cdot \xi \cdot l} - \sigma_{e,0}}{\sigma_{e,1}} = \frac{\sigma_{e,0} \cdot e^{\sigma_{e,1} \cdot \xi \cdot l} - \sigma_{e,0} + \sigma_{e,1} \cdot T_1 \cdot e^{\sigma_{e,1} \cdot \xi \cdot l}}{\sigma_{e,1}} =
$$
$$
    = \frac{\sigma_{e,0} \cdot e^{\sigma_{e,1} \cdot \xi \cdot l} - \sigma_{e,0}}{\sigma_{e,1}} + T_1 \cdot e^{\sigma_{e,1} \cdot \xi \cdot l}.
$$

Pokud za $\xi$ dosadíme původní výraz, dostaneme:
$$
    T_2 = \frac{\sigma_{e,0} \cdot e^{\sigma_{e,1} \cdot \frac{U^2 \cdot b}{d \cdot \rho \cdot c \cdot \dot{V}} \cdot l} - \sigma_{e,0}}{\sigma_{e,1}} + T_1 \cdot e^{\sigma_{e,1} \cdot \frac{U^2 \cdot b}{d \cdot \rho \cdot c \cdot \dot{V}} \cdot l}.
$$

\end{document}
