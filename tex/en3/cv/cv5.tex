\documentclass{article}
\usepackage[czech]{babel} % Czech language
\usepackage[shortlabels]{enumitem} % Custom enumeration
\usepackage{graphicx} % Import images
\usepackage{float} % Use [H] to force figure position
\usepackage{indentfirst} % Indent first paragraph
\usepackage{emoji} % Emojis
\usepackage{pgffor} % Loops
\usepackage{tikz} % TikZ
\usepackage{pgfplots} % TikZ plots
\usepackage{amsmath} % Math
\usetikzlibrary{arrows.meta}

\makeatletter
\providecommand\add@text{}
\newcommand\tagaddtext[1]{%
    \gdef\add@text{#1\gdef\add@text{}}}% 
\renewcommand\tagform@[1]{%
    \maketag@@@{\llap{\add@text\quad}(\ignorespaces#1\unskip\@@italiccorr)}%
}
\makeatother

\newcommand{\cvHead}[1]{\head{Cvičení #1}}

\newcommand{\head}[1]{
    \title{\textbf{#1}\\Elektroenergetika 3}
    \author{Petr Jílek}
    \date{2024}
}

\newcommand{\spicy}{\emoji{hot-pepper}}

% --------------------
% --------------------
% Units
% --------------------
% --------------------

% --------------------
% Basic units
% --------------------  

% no unit
\newcommand{\uNOUNIT}{\te{--}} % no unit
\newcommand{\uPERCENT}{\te{\%}} % percent
\newcommand{\uDEGREE}{^\circ} % degree
\newcommand{\uRAD}{\te{rad}} % radian

% meter
\newcommand{\uM}{\te{m}} % meter
\newcommand{\uMsq}{\uM^\te{2}} % meter squared
\newcommand{\uMcu}{\uM^\te{3}} % meter cubed
\newcommand{\uMquar}{\uM^\te{4}} % meter quartic
\newcommand{\uMinv}{\uM^{-1}} % meter inverse
\newcommand{\uMinvsq}{\uM^{-2}} % meter inverse squared
\newcommand{\uMinvcu}{\uM^{-3}} % meter inverse cubed
\newcommand{\uMinvquar}{\uM^{-4}} % meter inverse quartic
\newcommand{\uMM}{\te{mm}} % millimeter
\newcommand{\uCM}{\te{cm}} % centimeter
\newcommand{\uKM}{\te{km}} % kilometer

% liter
\newcommand{\uL}{\te{l}} % liter
\newcommand{\uML}{\te{ml}} % milliliter

% second
\newcommand{\uS}{\te{s}} % second
\newcommand{\uSinv}{\uS^{-1}} % second inverse
\newcommand{\uSinvsq}{\uS^{-2}} % second inverse squared

% hour
\newcommand{\uH}{\te{h}} % hour
\newcommand{\uHinv}{\te{h}^{-1}} % hour inverse

% kilogram
\newcommand{\uKG}{\te{kg}} % kilogram
\newcommand{\uKGinv}{\uKG^{-1}} % kilogram inverse
\newcommand{\uKGinvsq}{\uKG^{-2}} % kilogram inverse squared
\newcommand{\uG}{\te{g}} % gram

% lumen
\newcommand{\uLM}{\te{lm}} % lumen

% kelvin
\newcommand{\uK}{\te{K}} % kelvin
\newcommand{\uKsq}{\uK^\te{2}} % kelvin squared
\newcommand{\uKcu}{\uK^\te{3}} % kelvin cubed
\newcommand{\uKquar}{\uK^\te{4}} % kelvin quartic
\newcommand{\uKinv}{\uK^{-1}} % kelvin inverse
\newcommand{\uKinvsq}{\uK^{-2}} % kelvin inverse squared
\newcommand{\uKinvcu}{\uK^{-3}} % kelvin inverse cubed
\newcommand{\uKinvquar}{\uK^{-4}} % kelvin inverse quartic

% degree celsius
\newcommand{\uCELS}{\uDEGREE \te{C}} % degree celsius

% watt
\newcommand{\uW}{\te{W}} % watt
\newcommand{\uKW}{\te{kW}} % kilowatt
\newcommand{\uMW}{\te{MW}} % megawatt
\newcommand{\uGW}{\te{GW}} % gigawatt
\newcommand{\uWinv}{\uW^{-1}} % watt inverse

% joule
\newcommand{\uJ}{\te{J}} % joule
\newcommand{\uKJ}{\te{kJ}} % kilojoule
\newcommand{\uMJ}{\te{MJ}} % megajoule
\newcommand{\uCAL}{\te{cal}} % calorie
\newcommand{\uKCAL}{\te{kcal}} % kilocalorie
\newcommand{\uWH}{\te{Wh}} % watt hour
\newcommand{\uKWH}{\te{kWh}} % kilowatt hour
\newcommand{\uWandS}{\te{Ws}} % watt second

% ampere
\newcommand{\uA}{\te{A}} % ampere

% volt
\newcommand{\uV}{\te{V}} % volt

% ohm
\newcommand{\uOHM}{\te{\Omega}} % ohm
\newcommand{\uOHMinv}{\uOHM^{-1}} % ohm inverse

% siemens
\newcommand{\uSIE}{\te{S}} % siemens
\newcommand{\uSIEinv}{\uSIE^{-1}} % siemens inverse

% newton
\newcommand{\uN}{\te{N}} % newton

% currency
\newcommand{\uCCY}{\te{CCY}} % currency
\newcommand{\uCZK}{\te{CZK}} % czech crown

% --------------------
% Compound units
% --------------------

% velocity-acceleration
\newcommand{\uKGandMinvcu}{\uKG \cdot \uMinvcu} % kilogram per meter cubed
\newcommand{\uMandSinv}{\uM \cdot \uSinv} % meter per second
\newcommand{\uMcuSinv}{\uMcu \cdot \uSinv} % meter cubed per second
\newcommand{\uMandSinvsq}{\uM \cdot \uSinvsq} % meter per second squared

% power-energy
\newcommand{\uWandMinvsq}{\uW \cdot \uMinvsq} % watt per meter squared
\newcommand{\uWandMinvcu}{\uW \cdot \uMinvcu} % watt per meter cubed
\newcommand{\uJandMinvsq}{\uJ \cdot \uMinvsq} % joule per meter squared
\newcommand{\uJandMinvcu}{\uJ \cdot \uMinvcu} % joule per meter cubed

% heat
\newcommand{\uJandKGinvKinv}{\uJ \cdot \uKGinv \cdot \uKinv} % joule per kilogram per kelvin
\newcommand{\uWandMinvKinv}{\uW \cdot \uMinv \cdot \uKinv} % watt per meter per kelvin
\newcommand{\uMsqKandWinv}{\uMsq \cdot \uK \cdot \uWinv} % meter squared kelvin per watt
\newcommand{\uKandWinv}{\uK \cdot \uWinv} % kelvin per watt inverse
\newcommand{\uWandMinvsqKinv}{\uW \cdot \uMinvsq \cdot \uKinv} % watt per meter squared per kelvin
\newcommand{\uWandKinv}{\uW \cdot \uKinv} % watt per kelvin inverse

% electricity
\newcommand{\uOHMandMinv}{\uOHM \cdot \uMinv} % ohm per meter
\newcommand{\uSIEandMinv}{\uSIE \cdot \uMinv} % siemens per meter
\newcommand{\uAandMinvsq}{\uA \cdot \uMinvsq} % ampere per meter squared
\newcommand{\uVandMinv}{\uV \cdot \uMinv} % volt per meter


% --------------------
% --------------------
% Units in equation
% --------------------
% --------------------

% --------------------
% Basic units
% --------------------

% no unit
\newcommand{\ueqNOUNIT}{$\uNOUNIT$} % no unit
\newcommand{\ueqPERCENT}{$\uPERCENT$} % percent
\newcommand{\ueqDEGREE}{$\uDEGREE$} % degree
\newcommand{\ueqRAD}{$\uRAD$} % radian

% meter
\newcommand{\ueqM}{$\uM$} % meter
\newcommand{\ueqMsq}{$\uMsq$} % meter squared
\newcommand{\ueqMcu}{$\uMcu$} % meter cubed
\newcommand{\ueqMquar}{$\uMquar$} % meter quartic
\newcommand{\ueqMinv}{$\uMinv$} % meter inverse
\newcommand{\ueqMinvsq}{$\uMinvsq$} % meter inverse squared
\newcommand{\ueqMinvcu}{$\uMinvcu$} % meter inverse cubed
\newcommand{\ueqMinvquar}{$\uMinvquar$} % meter inverse quartic
\newcommand{\ueqMM}{$\uMM$} % millimeter
\newcommand{\ueqCM}{$\uCM$} % centimeter
\newcommand{\ueqKM}{$\uKM$} % kilometer

% liter
\newcommand{\ueqL}{$\uL$} % liter
\newcommand{\ueqML}{$\uML$} % milliliter

% second
\newcommand{\ueqS}{$\uS$} % second
\newcommand{\ueqSinv}{$\uSinv$} % second inverse
\newcommand{\ueqSinvsq}{$\uSinvsq$} % second inverse squared

% hour
\newcommand{\ueqH}{$\uH$} % hour
\newcommand{\ueqHinv}{$\uHinv$} % hour inverse

% kilogram
\newcommand{\ueqKG}{$\uKG$} % kilogram
\newcommand{\ueqKGinv}{$\uKGinv$} % kilogram inverse
\newcommand{\ueqKGinvsq}{$\uKGinvsq$} % kilogram inverse squared
\newcommand{\ueqG}{$\uG$} % gram

% lumen
\newcommand{\ueqLM}{$\uLM$} % lumen

% kelvin
\newcommand{\ueqK}{$\uK$} % kelvin
\newcommand{\ueqKsq}{$\uKsq$} % kelvin squared
\newcommand{\ueqKcu}{$\uKcv$} % kelvin cubed
\newcommand{\ueqKquar}{$\uKquar$} % kelvin quartic
\newcommand{\ueqKinv}{$\uKinv$} % kelvin inverse
\newcommand{\ueqKinvsq}{$\uKinvsq$} % kelvin inverse squared
\newcommand{\ueqKinvcu}{$\uKinvcu$} % kelvin inverse cubed
\newcommand{\ueqKinvquar}{$\uuKinvquar$} % kelvin inverse quartic

% degree celsius
\newcommand{\ueqCELS}{$\uCELS$} % degree celsius

% watt
\newcommand{\ueqW}{$\uW$} % watt
\newcommand{\ueqKW}{$\uKW$} % kilowatt
\newcommand{\ueqMW}{$\uMW$} % megawatt
\newcommand{\ueqGW}{$\uGW$} % gigawatt
\newcommand{\ueqWinv}{$\uWinv$} % watt inverse

% joule
\newcommand{\ueqJ}{$\uJ$} % joule
\newcommand{\ueqKJ}{$\uKJ$} % kilojoule
\newcommand{\ueqMJ}{$\uMJ$} % megajoule
\newcommand{\ueqCAL}{$\uCAL$} % calorie
\newcommand{\ueqKCAL}{$\uKCAL$} % kilocalorie
\newcommand{\ueqWH}{$\uWH$} % watt hour
\newcommand{\ueqKWH}{$\uKWH$} % kilowatt hour
\newcommand{\ueqWandS}{$\uWandS$} % watt second

% ampere
\newcommand{\ueqA}{$\uA$} % ampere

% volt
\newcommand{\ueqV}{$\uV$} % volt

% ohm
\newcommand{\ueqOHM}{$\uOHM$} % ohm
\newcommand{\ueqOHMinv}{$\uOHMinv$} % ohm inverse

% siemens
\newcommand{\ueqSIE}{$\uSIE$} % siemens
\newcommand{\ueqSIEinv}{$\uSIEinv$} % siemens inverse

% newton
\newcommand{\ueqN}{$\uN$} % newton

% currency
\newcommand{\ueqCCY}{$\uCCY$} % currency
\newcommand{\ueqCZK}{$\uCZK$} % czech crown

% --------------------
% Compound units
% --------------------

% velocity-acceleration
\newcommand{\ueqKGandMinvcu}{$\uKGandMinvcu$} % kilogram per meter cubed
\newcommand{\ueqMandSinv}{$\uMandSinv$} % meter per second
\newcommand{\ueqMcuSinv}{$\uMcuSinv$} % meter cubed per second
\newcommand{\ueqMandSinvsq}{$\uMandSinvsq$} % meter per second squared

% power-energy
\newcommand{\ueqWandMinvsq}{$\uWandMinvsq$} % watt per meter squared
\newcommand{\ueqWandMinvcu}{$\uWandMinvcu$} % watt per meter cubed
\newcommand{\ueqJandMinvsq}{$\uJandMinvsq$} % joule per meter squared
\newcommand{\ueqJandMinvcu}{$\uJandMinvcu$} % joule per meter cubed

% heat
\newcommand{\ueqJandKGinvKinv}{$\uJandKGinvKinv$} % joule per kilogram per kelvin
\newcommand{\ueqWandMinvKinv}{$\uWandMinvKinv$} % watt per meter per kelvin
\newcommand{\ueqMsqKandWinv}{$\uMsqKandWinv$} % meter squared kelvin per watt
\newcommand{\ueqKandWinv}{$\uKandWinv$} % kelvin per watt inverse
\newcommand{\ueqWandMinvsqKinv}{$\uWandMinvsqKinv$} % watt per meter squared per kelvin
\newcommand{\ueqWandKinv}{$\uWandKinv$} % watt per kelvin inverse

% electricity
\newcommand{\ueqOHMandMinv}{$\uOHMandMinv$} % ohm per meter
\newcommand{\ueqSIEandMinv}{$\uSIEandMinv$} % siemens per meter
\newcommand{\ueqAandMinvsq}{$\uAandMinvsq$} % ampere per meter squared
\newcommand{\ueqVandMinv}{$\uVandMinv$} % volt per meter


\cvHead{5 - Aplikace}

\begin{document}

\maketitle
\tableofcontents
\newpage




\section{\emoji{snowflake} Topná sezóna \spicy \spicy \spicy}
Průměrná venkonví teplota v topné sezóně je $\overline{T}_{out} = 5 \fs \uCELS$. Vnitřní teplota je $T_{in} = 20 \fs \uCELS$. Doba topné sezóny je 200 dní. Celková plocha je $S = 300 \fs \uMsq$. Součinitel prostupu tepla je $U_\vartheta = 0,5 \fs \uWandMinvsqKinv$. Jaký množství tepla projde stěnami za celou topnou sezónu?



\subsection{Řešení}
Celkové množství tepla, které projde stěnami za celou topnou sezónu $Q$ vypočteme jako:
$$
    Q = \int_{t_1}^{t_2} \dot{Q} \cdot dt,
$$
kde:\\
$\dot{Q}$ je tepelný tok (\ueqW),\\
$t_1$ je začátek topné sezóny (\uH),\\
$t_2$ je konec topné sezóny (\uH).\\

Integrál můžeme rozepsat jako:
$$
    Q = \int_{t_1}^{t_2} U_\vartheta \cdot S \cdot (T_{in} - T_{out} (t)) \cdot dt = U_\vartheta \cdot S \cdot \int_{t_1}^{t_2} (T_{in} - T_{out} (t)) \cdot dt =
$$
$$
    = U_\vartheta \cdot S \cdot T_{in} \cdot \int_{t_1}^{t_2} dt - U_\vartheta \cdot S \cdot \int_{t_1}^{t_2} T_{out} (t) \cdot dt =
$$
$$
    = U_\vartheta \cdot S \cdot T_{in} \cdot (t_2 - t_1) - U_\vartheta \cdot S \cdot \int_{t_1}^{t_2} T_{out} (t) \cdot dt.
$$

Nyní uděláme odbočku, kde vyjádříme průměrnou venkovní teplotu $\overline{T}_{out}$ jako:
$$
    \overline{T}_{out} = \frac{1}{t_2 - t_1} \cdot \int_{t_1}^{t_2} T_{out} (t) \cdot dt.
$$

Vyjádříme daný integrál:
$$
    \int_{t_1}^{t_2} T_{out} (t) \cdot dt = \overline{T}_{out} \cdot (t_2 - t_1).
$$

Dosadíme zpět do původní rovnice:
$$
    Q = U_\vartheta \cdot S \cdot T_{in} \cdot (t_2 - t_1) - U_\vartheta \cdot S \cdot \overline{T}_{out} \cdot (t_2 - t_1) =
$$
$$
    = U_\vartheta \cdot S \cdot (T_{in} - \overline{T}_{out}) \cdot (t_2 - t_1).
$$

Nyní můžeme dosadit a vypočítat:
$$
    Q = 0,5 \cdot 300 \cdot (20 - 5) \cdot (200 \cdot 24 \cdot 3 \fs 600 - 0) =
$$
$$
    = 0,5 \cdot 300 \cdot 15 \cdot 200 \cdot 24 \cdot 3 \fs 600 = 38,9 \fs \uGJ.
$$

\newpage




\section{\emoji{factory} Cihlová pec \spicy \spicy \spicy}
U cihlové pece je vysoký rozdíl teplot, tudíž tepelnou vodivost je třeba uvažovat jako proměnnou. Tepelnou vodivost $\lambda$ můžeme zapsat jako:
$$
    \lambda (x) = \lambda_0 + \lambda_1 \cdot T,
$$

Pro zeď o tloušťce $d$ budeme uvažovat následující zjednodušející předpoklady:
\begin{itemize}
    \item $\frac{\partial T}{\partial t}$ = 0,
    \item $\vec{v}$ = 0,
    \item $\dot{Q}_V$ = 0,
    \item $\lambda (x) = \lambda_0 + \lambda_1 \cdot T$,
    \item $T = T(x)$ - teplota závislá pouze na ose x.
\end{itemize}

Poté Fourierova-Kirchhoffova rovnice bude mít tvar:
$$
    0 = \frac{d}{dx} \left ( \lambda (T) \cdot \frac{dT}{dx} \right ).
$$

Fourieruv zákon pro tepelný tok bude mít tvar:
$$
    \dot{q} = - \lambda (T) \cdot \frac{dT}{dx}.
$$

Vynásobme rovnici mínus jedna:
$$
    - \dot{q} = \lambda (T) \cdot \frac{dT}{dx}.
$$

Pokud dosadíme do Fourierova-Kirchhoffovy rovnice, dostaneme:
$$
    0 = \frac{d}{dx} \left ( - \dot{q} \right ).
$$

Z toho vidíme, že měrný tepelný tok $\dot{q}$ je konstantní a nezávisí na poloze x. Můžeme tedy rozepsat Fourierův zákon jako:
$$
    - \dot{q} = \left( \lambda_0 + \lambda_1 \cdot T \right) \cdot \frac{dT}{dx}.
$$

Můžeme řešit tuto diferenciální rovnici separací proměnných:
$$
    - \dot{q} \cdot dx = \left( \lambda_0 + \lambda_1 \cdot T \right) \cdot dT.
$$

Nyní meze budou jak pro x, tak pro T:
\begin{itemize}
    \item $x = 0 \Rightarrow T = T_1$,
    \item $x = d \Rightarrow T = T_2$.
\end{itemize}

Rovnici můžeme integrovat:
$$
    - \int_{0}^{d} \dot{q} \cdot dx = \int_{T_1}^{T_2} \left( \lambda_0 + \lambda_1 \cdot T \right) \cdot dT
$$
$$
    - \dot{q} \cdot \left[ x \right]_{0}^{d} = \lambda_0 \cdot \left[ T \right]_{T_1}^{T_2} + \frac{\lambda_1}{2} \cdot \left[ T^2 \right]_{T_1}^{T_2}
$$
$$
    - \dot{q} \cdot d = \lambda_0 \cdot (T_2 - T_1) + \frac{\lambda_1}{2} \cdot (T_2^2 - T_1^2)
$$
$$
    - \dot{q} \cdot d = \lambda_0 \cdot (T_2 - T_1) + \frac{\lambda_1}{2} \cdot (T_2 - T_1) \cdot (T_2 + T_1)
$$
$$
    - \dot{q} \cdot d = (T_2 - T_1) \cdot \left( \lambda_0 + \frac{\lambda_1}{2} \cdot (T_2 + T_1) \right).
$$

Nyní odvoďme střední hodnotu tepelné vodivosti $\overline{\lambda}$:
$$
    \overline{\lambda} = \frac{1}{T_2 - T_1} \cdot \int_{T_1}^{T_2} \lambda (T) \cdot dT = \frac{1}{T_2 - T_1} \cdot \int_{T_1}^{T_2} \left( \lambda_0 + \lambda_1 \cdot T \right) \cdot dT
$$
$$
    = \frac{1}{T_2 - T_1} \cdot \left[ \lambda_0 \cdot T + \frac{\lambda_1}{2} \cdot T^2 \right]_{T_1}^{T_2} = \frac{1}{T_2 - T_1} \cdot \left( \lambda_0 \cdot (T_2 - T_1) + \frac{\lambda_1}{2} \cdot (T_2^2 - T_1^2) \right) =
$$
$$
    = \lambda_0 + \frac{\lambda_1}{2} \cdot (T_2 + T_1).
$$

Nyní můžeme videt, že člen v závorce u vyřešené diferenciální rovnice je roven střední hodnotě tepelné vodivosti $\overline{\lambda}$:
$$
    - \dot{q} \cdot d = (T_2 - T_1) \cdot \overline{\lambda}.
$$

Nyní můžeme odvodit vztah pro měrný tepelný tok $\dot{q}$:
$$
    \dot{q} = \frac{T_1 - T_2}{d} \cdot \overline{\lambda}.
$$

\newpage




\section{\emoji{shower} Průtokový ohřívač \spicy \spicy \spicy \spicy}
Máme průtokový ohřívač tvořený dvěma obdelníkovými elektrodovými deskami o délce $l \fs (\uM)$ a šířce $b \fs (\uM)$. Vzdálenost mezi deskami je $d \fs (\uM)$. Mezi deskami je voda produdící o rychlostí $v$ pouze ve směru $x$. Voda má měrnou tepelnou kapacitu $c \fs (\uJandKGinvKinv)$ a hustotu $\rho \fs (\uKGandMinvcu)$. Elektrické napětí mezi deskami je $U \fs (\uV)$. Odvoďte změnu teploty vody $T_2 \fs (\uK)$ na výstupu průtokového ohřívače, pokud na vstupu je voda o teplotě $T_1 \fs (\uK)$.\\

Budou platit následující předpoklady:
\begin{itemize}
    \item $\frac{\partial T}{\partial t}$ = 0,
    \item $\nabla \cdot (\lambda \cdot \vec{\nabla} T)$ = 0.
\end{itemize}



\subsection{Řešení}
Fourier-Kirchhoffova rovnice:
\begin{equation}
    \rho \cdot c \cdot \vec{v} \cdot \vec{\nabla} T = \dot{Q}_V,
    \unit{(\ueqWandMinvcu)}
\end{equation}
kde:\\
$\rho$ -- hustota (\ueqKGandMinvcu),\\
$c$ -- měrná tepelná kapacita (\ueqJandKGinvKinv),\\
$\vec{v}$ -- rychlost (\ueqMandSinv),\\
$\vec{\nabla} T$ -- gradient teploty (\ueqKandMinv),\\
$\dot{Q}_V$ -- objemový zdroj tepla (\ueqWandMinvcu).\\

Vektor rychlosti $\vec{v}$ je:
$$
    \vec{v} = \begin{pmatrix} v \\ 0 \\ 0 \end{pmatrix}.
$$

Vektor rychlosti $\vec{v}$ má pouze složku $v$, protože voda teče pouze ve směru osy $x$. Můžeme tedy rovnici zjednodušit na:
$$
    \rho \cdot c \cdot v \cdot \frac{dT}{dx} = \dot{Q}_V.
$$

Dále je třeba vyjádřit rychlost $v$ a objemový zdroj tepla $\dot{Q}_V$. Rychlost $v$ je:
$$
    v = \frac{\dot{V}}{S} = \frac{\dot{V}}{d \cdot b},
$$
kde:\\
$\dot{V}$ -- objemový průtok (\ueqMcuSinv).\\

Objemový zdroj tepla $\dot{Q}_V$ můžeme vyjádřit pomocí intenzity elektrického pole $E$ a proudové hustoty $J$:
$$
    \dot{Q}_V = E \cdot J,
$$
kde:\\
$E$ -- intenzita elektrického pole (\ueqVandMinv),\\
$J$ -- proudová hustota (\ueqAandMinvsq).\\

Proudovou hustotu $J$ můžeme vyjádřit pomocí Ohmova zákona v diferenciálním tvaru:
$$
    J = \sigma_e \cdot E,
$$
kde:\\
$\sigma_e$ -- měrná elektrická vodivost (\ueqOHMandMinv).\\

Dostaneme tedy:
$$
    \dot{Q}_V = E \cdot \sigma_e \cdot E = \sigma_e \cdot E^2.
$$

Intenzitu elektrického pole $E$ můžeme vyjádřit pomocí napětí $U$ a vzdálenosti mezi deskami $d$:
$$
    E = \frac{U}{d}.
$$

Dostaneme tedy:
$$
    \dot{Q}_V = \sigma_e \cdot \left( \frac{U}{d} \right)^2.
$$

Zde je třeba uvažovat měrnou elektrickou vodivost $\sigma_e$ jako proměrnou závislou na teplotě. U kovů se vzrůstající teplotou se měrná elektrická vodivost snižuje, jelikož volné nostiče náboje mají problém se v rozpohybované krystalové mřížce kvůli teplotě pohybovat. Naopak u iontových roztoku jako třeba u vody se měrná elektrická vodivost zvyšuje s teplotou. Nahradíme tedy $\sigma_e$ lineární funkcí teploty $\sigma_e(T)$:
$$
    \sigma_e = \sigma_{e,0} + \sigma_{e,1} \cdot T.
$$

Nyní můžeme dosadit vše do rovnice:
$$
    \rho \cdot c \cdot \frac{\dot{V}}{d \cdot b} \cdot \frac{dT}{dx} = \left( \sigma_{e,0} + \sigma_{e,1} \cdot T \right) \cdot \left( \frac{U}{d} \right)^2.
$$

Dostáváme diferenciální rovnici, kterou můžeme řešit separací proměnných:
$$
    \frac{dT}{\sigma_{e,0} + \sigma_{e,1} \cdot T} = \frac{U^2 \cdot b}{d \cdot \rho \cdot c \cdot \dot{V}} \cdot dx.
$$

Pravou staranu můžeme pro lepší čitelnost narhadit:
$$
    \xi = \frac{U^2 \cdot b}{d \cdot \rho \cdot c \cdot \dot{V}}.
$$

Dostáváme:
$$
    \frac{dT}{\sigma_{e,0} + \sigma_{e,1} \cdot T} = \xi \cdot dx.
$$

Integrujeme obě strany:
$$
    \int_{T_1}^{T_2} \frac{dT}{\sigma_{e,0} + \sigma_{e,1} \cdot T} = \int_{0}^{l} \xi \cdot dx,
$$
kde:\\
$T_1$ -- počáteční teplota (\ueqK),\\
$T_2$ -- konečná teplota (\ueqK),\\
$l$ -- délka průtokového ohřívače (\ueqM).\\

Dostáváme:
$$
    \left[ \frac{1}{\sigma_{e,1}} \cdot \ln \left( \sigma_{e,0} + \sigma_{e,1} \cdot T \right) \right]_{T_1}^{T_2} = \left[ \xi \cdot x \right]_0^l
$$
$$
    \frac{1}{\sigma_{e,1}} \cdot \ln \left( \sigma_{e,0} + \sigma_{e,1} \cdot T_2 \right) - \frac{1}{\sigma_{e,1}} \cdot \ln \left( \sigma_{e,0} + \sigma_{e,1} \cdot T_1 \right) = \xi \cdot l
$$
$$
    \ln \left( \frac{\sigma_{e,0} + \sigma_{e,1} \cdot T_2}{\sigma_{e,0} + \sigma_{e,1} \cdot T_1} \right) = \sigma_{e,1} \cdot \xi \cdot l
$$
$$
    \frac{\sigma_{e,0} + \sigma_{e,1} \cdot T_2}{\sigma_{e,0} + \sigma_{e,1} \cdot T_1} = e^{\sigma_{e,1} \cdot \xi \cdot l}
$$
$$
    \sigma_{e,0} + \sigma_{e,1} \cdot T_2 = \left( \sigma_{e,0} + \sigma_{e,1} \cdot T_1 \right) \cdot e^{\sigma_{e,1} \cdot \xi \cdot l}
$$
$$
    T_2 = \frac{\left( \sigma_{e,0} + \sigma_{e,1} \cdot T_1 \right) \cdot e^{\sigma_{e,1} \cdot \xi \cdot l} - \sigma_{e,0}}{\sigma_{e,1}} = \frac{\sigma_{e,0} \cdot e^{\sigma_{e,1} \cdot \xi \cdot l} - \sigma_{e,0} + \sigma_{e,1} \cdot T_1 \cdot e^{\sigma_{e,1} \cdot \xi \cdot l}}{\sigma_{e,1}} =
$$
$$
    = \frac{\sigma_{e,0} \cdot e^{\sigma_{e,1} \cdot \xi \cdot l} - \sigma_{e,0}}{\sigma_{e,1}} + T_1 \cdot e^{\sigma_{e,1} \cdot \xi \cdot l}.
$$

Pokud za $\xi$ dosadíme původní výraz, dostaneme:
$$
    T_2 = \frac{\sigma_{e,0} \cdot e^{\sigma_{e,1} \cdot \frac{U^2 \cdot b}{d \cdot \rho \cdot c \cdot \dot{V}} \cdot l} - \sigma_{e,0}}{\sigma_{e,1}} + T_1 \cdot e^{\sigma_{e,1} \cdot \frac{U^2 \cdot b}{d \cdot \rho \cdot c \cdot \dot{V}} \cdot l}.
$$

\end{document}
