\documentclass{article}
\usepackage[czech]{babel} % Czech language
\usepackage[shortlabels]{enumitem} % Custom enumeration
\usepackage{graphicx} % Import images
\usepackage{float} % Use [H] to force figure position
\usepackage{indentfirst} % Indent first paragraph
\usepackage{emoji} % Emojis
\usepackage{pgffor} % Loops
\usepackage{tikz} % TikZ
\usepackage{pgfplots} % TikZ plots
\usepackage{amsmath} % Math
\usetikzlibrary{arrows.meta}

\makeatletter
\providecommand\add@text{}
\newcommand\tagaddtext[1]{%
    \gdef\add@text{#1\gdef\add@text{}}}% 
\renewcommand\tagform@[1]{%
    \maketag@@@{\llap{\add@text\quad}(\ignorespaces#1\unskip\@@italiccorr)}%
}
\makeatother

\newcommand{\cvHead}[1]{\head{Cvičení #1}}

\newcommand{\head}[1]{
    \title{\textbf{#1}\\Elektroenergetika 3}
    \author{Petr Jílek}
    \date{2024}
}

\newcommand{\spicy}{\emoji{hot-pepper}}

% --------------------
% --------------------
% Units
% --------------------
% --------------------

% --------------------
% Basic units
% --------------------  

% no unit
\newcommand{\uNOUNIT}{\te{--}} % no unit
\newcommand{\uPERCENT}{\te{\%}} % percent
\newcommand{\uDEGREE}{^\circ} % degree
\newcommand{\uRAD}{\te{rad}} % radian

% meter
\newcommand{\uM}{\te{m}} % meter
\newcommand{\uMsq}{\uM^\te{2}} % meter squared
\newcommand{\uMcu}{\uM^\te{3}} % meter cubed
\newcommand{\uMquar}{\uM^\te{4}} % meter quartic
\newcommand{\uMinv}{\uM^{-1}} % meter inverse
\newcommand{\uMinvsq}{\uM^{-2}} % meter inverse squared
\newcommand{\uMinvcu}{\uM^{-3}} % meter inverse cubed
\newcommand{\uMinvquar}{\uM^{-4}} % meter inverse quartic
\newcommand{\uMM}{\te{mm}} % millimeter
\newcommand{\uCM}{\te{cm}} % centimeter
\newcommand{\uKM}{\te{km}} % kilometer

% liter
\newcommand{\uL}{\te{l}} % liter
\newcommand{\uML}{\te{ml}} % milliliter

% second
\newcommand{\uS}{\te{s}} % second
\newcommand{\uSinv}{\uS^{-1}} % second inverse
\newcommand{\uSinvsq}{\uS^{-2}} % second inverse squared

% hour
\newcommand{\uH}{\te{h}} % hour
\newcommand{\uHinv}{\te{h}^{-1}} % hour inverse

% kilogram
\newcommand{\uKG}{\te{kg}} % kilogram
\newcommand{\uKGinv}{\uKG^{-1}} % kilogram inverse
\newcommand{\uKGinvsq}{\uKG^{-2}} % kilogram inverse squared
\newcommand{\uG}{\te{g}} % gram

% lumen
\newcommand{\uLM}{\te{lm}} % lumen

% kelvin
\newcommand{\uK}{\te{K}} % kelvin
\newcommand{\uKsq}{\uK^\te{2}} % kelvin squared
\newcommand{\uKcu}{\uK^\te{3}} % kelvin cubed
\newcommand{\uKquar}{\uK^\te{4}} % kelvin quartic
\newcommand{\uKinv}{\uK^{-1}} % kelvin inverse
\newcommand{\uKinvsq}{\uK^{-2}} % kelvin inverse squared
\newcommand{\uKinvcu}{\uK^{-3}} % kelvin inverse cubed
\newcommand{\uKinvquar}{\uK^{-4}} % kelvin inverse quartic

% degree celsius
\newcommand{\uCELS}{\uDEGREE \te{C}} % degree celsius

% watt
\newcommand{\uW}{\te{W}} % watt
\newcommand{\uKW}{\te{kW}} % kilowatt
\newcommand{\uMW}{\te{MW}} % megawatt
\newcommand{\uGW}{\te{GW}} % gigawatt
\newcommand{\uWinv}{\uW^{-1}} % watt inverse

% joule
\newcommand{\uJ}{\te{J}} % joule
\newcommand{\uKJ}{\te{kJ}} % kilojoule
\newcommand{\uMJ}{\te{MJ}} % megajoule
\newcommand{\uCAL}{\te{cal}} % calorie
\newcommand{\uKCAL}{\te{kcal}} % kilocalorie
\newcommand{\uWH}{\te{Wh}} % watt hour
\newcommand{\uKWH}{\te{kWh}} % kilowatt hour
\newcommand{\uWandS}{\te{Ws}} % watt second

% ampere
\newcommand{\uA}{\te{A}} % ampere

% volt
\newcommand{\uV}{\te{V}} % volt

% ohm
\newcommand{\uOHM}{\te{\Omega}} % ohm
\newcommand{\uOHMinv}{\uOHM^{-1}} % ohm inverse

% siemens
\newcommand{\uSIE}{\te{S}} % siemens
\newcommand{\uSIEinv}{\uSIE^{-1}} % siemens inverse

% newton
\newcommand{\uN}{\te{N}} % newton

% currency
\newcommand{\uCCY}{\te{CCY}} % currency
\newcommand{\uCZK}{\te{CZK}} % czech crown

% --------------------
% Compound units
% --------------------

% velocity-acceleration
\newcommand{\uKGandMinvcu}{\uKG \cdot \uMinvcu} % kilogram per meter cubed
\newcommand{\uMandSinv}{\uM \cdot \uSinv} % meter per second
\newcommand{\uMcuSinv}{\uMcu \cdot \uSinv} % meter cubed per second
\newcommand{\uMandSinvsq}{\uM \cdot \uSinvsq} % meter per second squared

% power-energy
\newcommand{\uWandMinvsq}{\uW \cdot \uMinvsq} % watt per meter squared
\newcommand{\uWandMinvcu}{\uW \cdot \uMinvcu} % watt per meter cubed
\newcommand{\uJandMinvsq}{\uJ \cdot \uMinvsq} % joule per meter squared
\newcommand{\uJandMinvcu}{\uJ \cdot \uMinvcu} % joule per meter cubed

% heat
\newcommand{\uJandKGinvKinv}{\uJ \cdot \uKGinv \cdot \uKinv} % joule per kilogram per kelvin
\newcommand{\uWandMinvKinv}{\uW \cdot \uMinv \cdot \uKinv} % watt per meter per kelvin
\newcommand{\uMsqKandWinv}{\uMsq \cdot \uK \cdot \uWinv} % meter squared kelvin per watt
\newcommand{\uKandWinv}{\uK \cdot \uWinv} % kelvin per watt inverse
\newcommand{\uWandMinvsqKinv}{\uW \cdot \uMinvsq \cdot \uKinv} % watt per meter squared per kelvin
\newcommand{\uWandKinv}{\uW \cdot \uKinv} % watt per kelvin inverse

% electricity
\newcommand{\uOHMandMinv}{\uOHM \cdot \uMinv} % ohm per meter
\newcommand{\uSIEandMinv}{\uSIE \cdot \uMinv} % siemens per meter
\newcommand{\uAandMinvsq}{\uA \cdot \uMinvsq} % ampere per meter squared
\newcommand{\uVandMinv}{\uV \cdot \uMinv} % volt per meter


% --------------------
% --------------------
% Units in equation
% --------------------
% --------------------

% --------------------
% Basic units
% --------------------

% no unit
\newcommand{\ueqNOUNIT}{$\uNOUNIT$} % no unit
\newcommand{\ueqPERCENT}{$\uPERCENT$} % percent
\newcommand{\ueqDEGREE}{$\uDEGREE$} % degree
\newcommand{\ueqRAD}{$\uRAD$} % radian

% meter
\newcommand{\ueqM}{$\uM$} % meter
\newcommand{\ueqMsq}{$\uMsq$} % meter squared
\newcommand{\ueqMcu}{$\uMcu$} % meter cubed
\newcommand{\ueqMquar}{$\uMquar$} % meter quartic
\newcommand{\ueqMinv}{$\uMinv$} % meter inverse
\newcommand{\ueqMinvsq}{$\uMinvsq$} % meter inverse squared
\newcommand{\ueqMinvcu}{$\uMinvcu$} % meter inverse cubed
\newcommand{\ueqMinvquar}{$\uMinvquar$} % meter inverse quartic
\newcommand{\ueqMM}{$\uMM$} % millimeter
\newcommand{\ueqCM}{$\uCM$} % centimeter
\newcommand{\ueqKM}{$\uKM$} % kilometer

% liter
\newcommand{\ueqL}{$\uL$} % liter
\newcommand{\ueqML}{$\uML$} % milliliter

% second
\newcommand{\ueqS}{$\uS$} % second
\newcommand{\ueqSinv}{$\uSinv$} % second inverse
\newcommand{\ueqSinvsq}{$\uSinvsq$} % second inverse squared

% hour
\newcommand{\ueqH}{$\uH$} % hour
\newcommand{\ueqHinv}{$\uHinv$} % hour inverse

% kilogram
\newcommand{\ueqKG}{$\uKG$} % kilogram
\newcommand{\ueqKGinv}{$\uKGinv$} % kilogram inverse
\newcommand{\ueqKGinvsq}{$\uKGinvsq$} % kilogram inverse squared
\newcommand{\ueqG}{$\uG$} % gram

% lumen
\newcommand{\ueqLM}{$\uLM$} % lumen

% kelvin
\newcommand{\ueqK}{$\uK$} % kelvin
\newcommand{\ueqKsq}{$\uKsq$} % kelvin squared
\newcommand{\ueqKcu}{$\uKcv$} % kelvin cubed
\newcommand{\ueqKquar}{$\uKquar$} % kelvin quartic
\newcommand{\ueqKinv}{$\uKinv$} % kelvin inverse
\newcommand{\ueqKinvsq}{$\uKinvsq$} % kelvin inverse squared
\newcommand{\ueqKinvcu}{$\uKinvcu$} % kelvin inverse cubed
\newcommand{\ueqKinvquar}{$\uuKinvquar$} % kelvin inverse quartic

% degree celsius
\newcommand{\ueqCELS}{$\uCELS$} % degree celsius

% watt
\newcommand{\ueqW}{$\uW$} % watt
\newcommand{\ueqKW}{$\uKW$} % kilowatt
\newcommand{\ueqMW}{$\uMW$} % megawatt
\newcommand{\ueqGW}{$\uGW$} % gigawatt
\newcommand{\ueqWinv}{$\uWinv$} % watt inverse

% joule
\newcommand{\ueqJ}{$\uJ$} % joule
\newcommand{\ueqKJ}{$\uKJ$} % kilojoule
\newcommand{\ueqMJ}{$\uMJ$} % megajoule
\newcommand{\ueqCAL}{$\uCAL$} % calorie
\newcommand{\ueqKCAL}{$\uKCAL$} % kilocalorie
\newcommand{\ueqWH}{$\uWH$} % watt hour
\newcommand{\ueqKWH}{$\uKWH$} % kilowatt hour
\newcommand{\ueqWandS}{$\uWandS$} % watt second

% ampere
\newcommand{\ueqA}{$\uA$} % ampere

% volt
\newcommand{\ueqV}{$\uV$} % volt

% ohm
\newcommand{\ueqOHM}{$\uOHM$} % ohm
\newcommand{\ueqOHMinv}{$\uOHMinv$} % ohm inverse

% siemens
\newcommand{\ueqSIE}{$\uSIE$} % siemens
\newcommand{\ueqSIEinv}{$\uSIEinv$} % siemens inverse

% newton
\newcommand{\ueqN}{$\uN$} % newton

% currency
\newcommand{\ueqCCY}{$\uCCY$} % currency
\newcommand{\ueqCZK}{$\uCZK$} % czech crown

% --------------------
% Compound units
% --------------------

% velocity-acceleration
\newcommand{\ueqKGandMinvcu}{$\uKGandMinvcu$} % kilogram per meter cubed
\newcommand{\ueqMandSinv}{$\uMandSinv$} % meter per second
\newcommand{\ueqMcuSinv}{$\uMcuSinv$} % meter cubed per second
\newcommand{\ueqMandSinvsq}{$\uMandSinvsq$} % meter per second squared

% power-energy
\newcommand{\ueqWandMinvsq}{$\uWandMinvsq$} % watt per meter squared
\newcommand{\ueqWandMinvcu}{$\uWandMinvcu$} % watt per meter cubed
\newcommand{\ueqJandMinvsq}{$\uJandMinvsq$} % joule per meter squared
\newcommand{\ueqJandMinvcu}{$\uJandMinvcu$} % joule per meter cubed

% heat
\newcommand{\ueqJandKGinvKinv}{$\uJandKGinvKinv$} % joule per kilogram per kelvin
\newcommand{\ueqWandMinvKinv}{$\uWandMinvKinv$} % watt per meter per kelvin
\newcommand{\ueqMsqKandWinv}{$\uMsqKandWinv$} % meter squared kelvin per watt
\newcommand{\ueqKandWinv}{$\uKandWinv$} % kelvin per watt inverse
\newcommand{\ueqWandMinvsqKinv}{$\uWandMinvsqKinv$} % watt per meter squared per kelvin
\newcommand{\ueqWandKinv}{$\uWandKinv$} % watt per kelvin inverse

% electricity
\newcommand{\ueqOHMandMinv}{$\uOHMandMinv$} % ohm per meter
\newcommand{\ueqSIEandMinv}{$\uSIEandMinv$} % siemens per meter
\newcommand{\ueqAandMinvsq}{$\uAandMinvsq$} % ampere per meter squared
\newcommand{\ueqVandMinv}{$\uVandMinv$} % volt per meter


\cvHead{1 - Energie}

\begin{document}

\maketitle
\tableofcontents
\newpage



\section{\emoji{snow-capped-mountain} Potenciální energie \spicy}
Do jaké výšky vynese energie jedné Fidorky (30 g $\rightarrow$ 162 \ueqKCAL) 80 \ueqKG \fs člověka?

\begin{enumerate}[a)]
    \item Gravitační zrychlení je konstantní s hodnotou 9,81 \ueqMandSinvsq.
    \item Počáteční výška je 0 m nad mořem a gravitační zrychlení je proměnné. Známe:
          \begin{itemize}
              \item gravitační konstanta: $G = 6,67430 \cdot 10^{-11}$ \ueqNandMsqKGinvsq,
              \item poloměr Země: $R = 6 \fs 371$ \ueqKM,
              \item hmotnost Země: $M = 5,97219 \cdot 10^{24}$ \ueqKG.
          \end{itemize}
\end{enumerate}


\subsection{a}
Energie jedné Fidorky:
$$
    162 \fs \uKCAL = 162 \cdot 4 \fs 184 \fs \frac{\uJ}{\uKCAL} = 677 \fs 810 \fs \uJ.
$$

Ze vzorce pro potenciální energii vyjádříme výšku:
$$
    E_p = m \cdot g \cdot \Delta h \Rightarrow \Delta h = \frac{E_p}{m \cdot g} = \frac{677 \fs 810 \fs \uJ}{80 \fs \uKG \cdot 9,81 \fs \uMandSinvsq} \approx 863,7 \fs m.
$$


\subsection{b \spicy \spicy \spicy}
Nejdříve si odvodíme vzorec pro gravitační zrychlení z Newtonova gravitačního zákona:
$$
    F = G \cdot \frac{m_1 \cdot m_2}{r^2},
$$
kde:\\
$F$ -- gravitační síla (\ueqN),\\
$G$ -- gravitační konstanta (\ueqNandMsqKGinvsq),\\
$m_1$ -- hmotnost prvního tělesa (\ueqKG),\\
$m_2$ -- hmotnost druhého tělesa (\ueqKG),\\
$r$ -- vzdálenost mezi tělesy (\ueqM).\\

Gravitační síla je definována jako:
$$
    F = m \cdot g,
$$
kde:\\
$m$ -- hmotnost (\ueqKG),\\
$g$ -- gravitační zrychlení (\ueqMandSinvsq).\\

Dále budeme uvažovat, že hmotnost $m_1$ je hmotnost Země $M$ a hmotnost $m_2$ je hmotnost člověka $m$ a dosadíme do Newtonova gravitačního zákona:
$$
    m \cdot g = G \cdot \frac{M \cdot m}{r^2}.
$$

Poté gravitační zrychlení $g$ je:
$$
    g = G \cdot \frac{M}{r^2}.
$$

Takto můžeme dosadit do vzorce pro potenciální energii:
$$
    E_p = m \cdot g \cdot h = m \cdot G \cdot \frac{M}{r^2} \cdot h.
$$

Nyní je třeba si uvědomit, že gravitační zrychlení $g$ je proměnné a závisí na výšce $h$. Je možné analyzovat kolik potřebujeme energie $\Delta E_p$ pro zvýšení výšky $h$ o $\Delta h$:
$$
    \Delta E_p = m \cdot G \cdot \frac{M}{r^2} \cdot \Delta h
$$

Zde se dopouštíme jisté nepřesnosti, jelikož pokud například $\Delta h$ bude 2 m, tak poloměr od středu země se změní taky, čímž se změní gravitační zrychlení. Pro získání přesného výsledku je třeba nahradit $\Delta h$ infinitesimálně malým $dh$ a $\Delta E_p$ bude infinitesimálně malé $dE_p$:
$$
    dE_p = m \cdot G \cdot \frac{M}{r^2} \cdot dh.
$$

Toto očividně vede na diferenciální rovnici, která lze snadno řešit separací proměnných, ale pozor. Nyní bychom brali poloměr $r$ jako konstantu. To je chyba, jelikož poloměr $r$ se mění s výškou $h$. Jeden z přístupů je nahradit výšku $h$ poloměrem $r$ a místo posouvání se o výšku $dh$ se posuneme o poloměr $dr$. Ale musíme si poté zapamatovat, že počáteční mezí je poloměr země $R$ a konečná mez bude tedy $R + h$. Poté je možné řešit rovnici:
$$
    E_p = \int_{R}^{R + h} m \cdot G \cdot \frac{M}{r^2} \cdot dr.
$$

Hmotnost $m$ a $M$ a gravitační konstanta $G$ je konstantní, takže je můžeme vytknout před integrál:
$$
    E_p = m \cdot G \cdot M \int_{R}^{R + h} \frac{1}{r^2} \cdot dr = m \cdot G \cdot M \left[ - \frac{1}{r} \right]_{R}^{R + h} = m \cdot G \cdot M \left( - \frac{1}{R + h} + \frac{1}{R} \right).
$$

Po úpravě dostaneme:
$$
    E_p = m \cdot G \cdot M \left( \frac{1}{R} - \frac{1}{R + h} \right).
$$

Nyní je třeba osamostatnit $h$:
$$
    E_p = m \cdot G \cdot M \left( \frac{R + h - R}{R \cdot (R + h)} \right) = m \cdot G \cdot M \left( \frac{h}{R \cdot (R + h)} \right)
$$

$$
    E_p \cdot R \cdot (R + h) = m \cdot G \cdot M \cdot h
$$

$$
    E_p \cdot R^2 + E_p \cdot R \cdot h = m \cdot G \cdot M \cdot h
$$

$$
    E_p \cdot R \cdot h - m \cdot G \cdot M \cdot h = - E_p \cdot R^2
$$

$$
    h \cdot (E_p \cdot R - m \cdot G \cdot M) = - E_p \cdot R^2
$$

$$
    h = \frac{- E_p \cdot R^2}{E_p \cdot R - m \cdot G \cdot M}
$$

$$
    h = \frac{E_p \cdot R^2}{m \cdot G \cdot M - E_p \cdot R}.
$$

Poté dosadíme hodnoty:
$$
    h = \frac{677 \fs 810 \cdot 6 \fs 371 \fs 000^2}{80 \cdot 6,67430 \cdot 10^{-11} \cdot 5,97219 \cdot 10^{24} - 677 \fs 810 \cdot 6 \fs 371 \fs 000}.
$$

$$
    h \approx 862,9 \fs m.
$$

Jako dodatek je možné si vytvořit graf závislosti gravitačního zrychlení na výšce $h$ nad mořem:

\begin{center}
    \begin{tikzpicture}
        \begin{axis}[
                title = {Závislost gravitačního zrychlení na výšce $h$ nad mořem.},
                axis lines = middle,
                xlabel = $h (\uM)$,
                ylabel = {$g (\uMandSinvsq)$},
                samples=400,
                domain=0:10000,
                xmin=0,
                xmax=10000,
                ymin=9.76,
                ymax=9.84,
                legend pos=outer north east,
                grid=both
            ]
            \addplot[color=blue, thick] {6.67430 * 10^(-11) * 5.97219 * 10^(24) / (6371000 + x)^2};
        \end{axis}
    \end{tikzpicture}
\end{center}

\newpage



\section{\emoji{fire} Tepelná kapacita \spicy}
Na jakou teplotu by energie potřebná k ohřátí vody z 10 \ueqCELS \fs na 100 \ueqCELS \fs ohřála ocel a zlato o stejné:

\begin{enumerate}[a)]
    \item hmotnosti,
    \item objemu.
\end{enumerate}

\begin{table}[H]
    \centering
    \begin{tabular}{l|ll}
        \hline
        Mateiál    & $\rho$ (\ueqKGandMinvcu) & $c$ (\ueqJandKGinvKinv) \\
        \hline
        Voda (H2O) & 1 000                    & 4 186                   \\
        Ocel       & 7 750                    & 450                     \\
        Zlato      & 19 320                   & 129                     \\
        \hline
    \end{tabular}
    \caption {Hustota a měrná tepelná kapacita materiálů}
\end{table}


\subsection{a}
Rozdíl teplot pro vodu je:
$$
    \Delta T_{H2O} = 100 \fs \uCELS - 10 \fs \uCELS = 90 \fs \uK.
$$

Podle rovnice pro měrnou tepelnou kapacitu můžeme napsat:
$$
    m_{H2O} \cdot c_{H2O} \cdot \Delta T_{H2O} = m_{ocel} \cdot c_{ocel} \cdot \Delta T_{ocel} = m_{zlato} \cdot c_{zlato} \cdot \Delta T_{zlato}.
$$

A zároveň pro tento příklad hmotnosti jsou stejné:
$$
    m_{H2O} = m_{ocel} = m_{zlato} = m.
$$

Tedy:
$$
    m \cdot c_{H2O} \cdot \Delta T_{H2O} = m \cdot c_{ocel} \cdot \Delta T_{ocel} = m \cdot c_{zlato} \cdot \Delta T_{zlato}.
$$

Rovnici lze vydělit hmotností $m$ ($m >$ 0) a dostaneme:
$$
    c_{H2O} \cdot \Delta T_{H2O} = c_{ocel} \cdot \Delta T_{ocel} = c_{zlato} \cdot \Delta T_{zlato}.
$$

Pro ocel platí:
$$
    \Delta T_{ocel} = \Delta T_{H2O} \cdot \frac{c_{H2O}}{c_{ocel}} = 90 \cdot \frac{4 186}{450} \approx 837,2 \fs \uK.
$$

Pro zlato platí:
$$
    \Delta T_{zlato} = \Delta T_{H2O} \cdot \frac{c_{H2O}}{c_{zlato}} = 90 \cdot \frac{4 186}{129} \approx 2 920,5 \fs \uK.
$$


\subsection{b}
Podle rovnice pro měrnou tepelnou kapacitu můžeme napsat:
$$
    V_{H2O} \cdot \rho_{H2O} \cdot c_{H2O} \cdot \Delta T_{H2O} = V_{ocel} \cdot \rho_{ocel} \cdot c_{ocel} \cdot \Delta T_{ocel} =
$$
$$
    = V_{zlato} \cdot \rho_{zlato} \cdot c_{zlato} \cdot \Delta T_{zlato}
$$

A zároveň pro tento příklad objemy jsou stejné:
$$
    V_{H2O} = V_{ocel} = V_{zlato} = V.
$$

Tedy:
$$
    V \cdot \rho_{H2O} \cdot c_{H2O} \cdot \Delta T_{H2O} = V \cdot \rho_{ocel} \cdot c_{ocel} \cdot \Delta T_{ocel} =
$$
$$
    =  V \cdot \rho_{zlato} \cdot c_{zlato} \cdot \Delta T_{zlato}.
$$

Rovnici lze vydělit objemem $V$ ($V >$ 0) a dostaneme:
$$
    \rho_{H2O} \cdot c_{H2O} \cdot \Delta T_{H2O} = \rho_{ocel} \cdot c_{ocel} \cdot \Delta T_{ocel} = \rho_{zlato} \cdot c_{zlato} \cdot \Delta T_{zlato}.
$$

Pro ocel platí:
$$
    \Delta T_{ocel} = \Delta T_{H2O} \cdot \frac{\rho_{H2O} \cdot c_{H2O}}{\rho_{ocel} \cdot c_{ocel}} = 90 \cdot \frac{1 000 \cdot 4 186}{7 750 \cdot 450} \approx 108,0 \fs \uK.
$$

Pro zlato platí:
$$
    \Delta T_{zlato} = \Delta T_{H2O} \cdot \frac{\rho_{H2O} \cdot c_{H2O}}{\rho_{zlato} \cdot c_{zlato}} = 90 \cdot \frac{1 000 \cdot 4 186}{19 320 \cdot 129} \approx 1511,6 \fs \uK.
$$

\newpage



\section{\emoji{shower} Průtokový ohřívač \spicy}
Mějme průtokový ohřívač vody, který ohřívá studenou vodu o teplotě $10 \fs \uCELS$ \fs na teplotu $40 \fs \uCELS$. Při sprchování je spotřeba vody $10 \fs \uL$ za minutu.

\begin{enumerate}[a)]
    \item Jaký je výkon ohřívače?
    \item Uvažujme, že ohřívač je na jednu fázi $230 \fs \uV$. Bude nám stačit jistič na $16 \fs \uA$?
\end{enumerate}


\subsection{a}

Objem protečené vody za 1 hodinu je:
$$
    V = 10 \fs \frac{\uL}{\te{min}} = 10 \fs \frac{\frac{\uMcu}{1000}}{\frac{\uH}{60}} = 10 \fs \frac{\uMcu}{1000} \cdot \frac{60}{\uH} = 0,6 \fs \frac{\uMcu}{\uH}.
$$

Hmotnost protečené vody za 1 hodinu je:
$$
    m = V \cdot \rho = 0,6 \fs \uMcu \cdot 1 000 \fs \frac{\uKG}{\uMcu} = 600 \fs \uKG.
$$

Množství energie za 1 hodinu potřebné k ohřátí vody je:
$$
    Q = m \cdot c \cdot \Delta T.
$$
$$
    Q = 600 \fs \uKG \cdot 4 \fs 186 \fs \frac{\uJ}{\uKG \cdot \uK} \cdot 30 \fs \uK = 75 \fs 348 \fs 000 \fs \uJ
$$

Výkon ohřívače je:
$$
    P = \frac{Q}{t} = \frac{75 \fs 348 \fs 000 \fs \uJ}{1 \fs \uH} = \frac{75 \fs 348 \fs 000 \fs \te{Ws}}{3 \fs 600 \fs \uS} = 20 \fs 930 \fs \uW = 20,93 \fs \uKW.
$$


\subsection{b}

Výkon ohřívače je:
$$
    P = U \cdot I,
$$
kde:\\
$U$ -- napětí (\ueqV),\\
$I$ -- proud (\ueqA).\\

Z rovnice pro výkon ohřívače můžeme vyjádřit proud:
$$
    I = \frac{P}{U} = \frac{20,93 \fs \uKW}{230 \fs \uV} = \frac{20 \fs 930 \fs \uW}{230 \fs \uV} = 91 \fs \uA.
$$

Jistič na $16 \fs \uA$ tedy nestačí, jelikož proud je $91 \fs \uA$.

\newpage



\section{\emoji{battery} Monočlánek \spicy}
Mějme monočlánek s kapacitou $2 \fs 500 \fs \te{mAh}$ a napětím $1,2 \fs \uV$.

\begin{enumerate}[a)]
    \item Kolik litrů vody ohřeje z $10 \fs \uCELS$ na $100 \fs \uCELS$?
    \item Jak vysoko vynese $80 \fs \uKG$ člověka?
\end{enumerate}


\subsection{a}
Energie monočlánku je:
$$
    E = 1,2 \fs \uV \cdot 2 \fs 500 \fs \te{mAh} = 1,5 \fs \uV \cdot 2,5 \fs \uA = 3 \fs \uWH = 3 \cdot 3 \fs 600 \fs \uJ = 10 \fs 800 \fs \uJ.
$$

Množství energie potřebné k ohřátí vody je:
$$
    Q = V \cdot \rho \cdot c \cdot \Delta T
$$

Z rovnice pro měrnou tepelnou kapacitu můžeme vyjádřit objem:
$$
    m = \frac{Q}{\rho \cdot c \cdot \Delta T} = \frac{10 \fs 800 \fs \uJ}{1 000 \fs \uKGandMinvcu \cdot 4 \fs 186 \fs \uJandKGinvKinv \cdot 90 \fs \uK} =
$$
$$
    = 2,87 \cdot 10^{-5} \fs \uMcu = 2,87 \cdot 10^{-2} \fs \uL = 28,7 \fs \uML.
$$


\subsection{b}
Ze vzorce pro potenciální energii vyjádříme výšku:
$$
    E_p = m \cdot g \cdot \Delta h \Rightarrow \Delta h = \frac{E_p}{m \cdot g} = \frac{10 \fs 800 \fs \uJ}{80 \fs \uKG \cdot 9,81 \fs \uMandSinvsq} \approx 13,76 \fs \uM.
$$

\end{document}
