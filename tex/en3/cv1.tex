\documentclass{article}
\usepackage[czech]{babel}
% \usepackage{graphicx}
\usepackage{float}
\usepackage{indentfirst}

\title{\textbf{Cvičení 1}\\Elektroenergetika 3}
\author{Petr Jílek}
\date{Září 2024}

\begin{document}

\newcommand{\cvHead}[1]{\head{Cvičení #1}}

\newcommand{\head}[1]{
    \title{\textbf{#1}\\Elektroenergetika 3}
    \author{Petr Jílek}
    \date{2024}
}

\newcommand{\spicy}{\emoji{hot-pepper}}

% My font space
\newcommand{\myFS}{\;}

% Text in math mode
\newcommand{\te}[1]{\textrm{#1}}

% --------------------
% Units
% --------------------

\newcommand{\uM}{\textrm{m}} % Meter
\newcommand{\uMsq}{\uM^\textrm{2}} % Meter squared
\newcommand{\uMcu}{\uM^\textrm{3}} % Meter cubed
\newcommand{\uS}{\textrm{s}} % Second
\newcommand{\uKG}{\textrm{kg}} % Kilogram
\newcommand{\uJ}{\textrm{J}} % Joule
\newcommand{\uK}{\textrm{K}} % Kelvin
\newcommand{\uDEGREE}{^\circ} % Degree
\newcommand{\uCELS}{\uDEGREE \textrm{C}} % Celsius
\newcommand{\uW}{\textrm{W}} % Watt
\newcommand{\uMW}{\textrm{MW}} % Mega Watt
\newcommand{\uKW}{\textrm{kW}} % Kilo Watt
\newcommand{\uWH}{\textrm{Wh}} % Watt hour
\newcommand{\uKWH}{\textrm{kWh}} % Kilo Watt hour
\newcommand{\uCCY}{\textrm{CCY}} % Currency
\newcommand{\uCZK}{\textrm{CZK}} % Czech Crown
\newcommand{\uPERCENT}{\textrm{\%}} % Percent
\newcommand{\uNOUNIT}{\textrm{--}} % No unit
\newcommand{\uYEAR}{\textrm{year}} % Year
\newcommand{\uMONTH}{\textrm{month}} % Month
\newcommand{\uHOUR}{\textrm{hour}} % Hour
\newcommand{\uLM}{\textrm{lm}} % Lumen
\newcommand{\uHinv}{\textrm{h}^{-1}} % Hour inverse

\newcommand{\uJperK}{\uJ / \uK} % Joule per Kelvin
\newcommand{\uKperM}{\uK / \uM} % Kelvin per Meter
\newcommand{\uKperS}{\uK / \uS} % Kelvin per Second
\newcommand{\uMsqperS}{\uMsq/\uS} % Meter squared per Second
\newcommand{\uWperMsq}{\uW / \uMsq} % Watt per Meter squared
\newcommand{\uWperMperK}{\uW / \left( \uM \cdot \uK \right)} % Watt per Meter per Kelvin
\newcommand{\uJperKGperK}{\uJ / \left( \uKG \cdot \uK \right)} % Joule per Kilogram per Kelvin
\newcommand{\uKperMsq}{\uK / \uMsq} % Kelvin per Meter squared
\newcommand{\uJperSperMperK}{\uJ / \left( \uS \cdot \uM \cdot \uK \right)} % Joule per Second per Meter per Kelvin
\newcommand{\uKGperMcu}{\uKG / \uMcu} % Kilogram per Meter cubed
\newcommand{\uMsqKperW}{\uMsq \cdot \uK / \uW} % Meter squared Kelvin per Watt
\newcommand{\uWperMsqperK}{\uW / \left( \uMsq \cdot \uK \right)} % Watt per Meter squared per Kelvin
\newcommand{\uKWHperMsqperYEAR}{\uKWH / \left( \uMsq \cdot \uYEAR \right)} % Kilo Watt hour per Meter squared per Year
\newcommand{\uKWHperMsq}{\uKWH / \uMsq} % Kilo Watt hour per Meter squared
\newcommand{\uLMperW}{\uLM / \uW} % Lumen per Watt
\newcommand{\uKWperMsq}{\uKW / \uMsq} % Kilo Watt per Meter squared
\newcommand{\uCCYperYEAR}{\uCCY / \uYEAR} % Currency per Year
\newcommand{\uKWHperYEAR}{\uKWH / \uYEAR} % Kilo Watt hour per Year
\newcommand{\uNOUNITperYEAR}{\uNOUNIT / \uYEAR} % No unit per Year
\newcommand{\uCCYperKWH}{\uCCY / \uKWH} % Currency per Kilo Watt hour
\newcommand{\uCZKperYEAR}{\uCZK / \uYEAR} % Czech Crown per Year
\newcommand{\uCZKperKWH}{\uCZK / \uKWH} % Czech Crown per Kilo Watt hour
\newcommand{\uCZKperMONTH}{\uCZK / \uMONTH} % Czech Crown per Month



% --------------------
% Unit equations
% --------------------

\newcommand{\ueqM}{$\uM$}
\newcommand{\ueqMsq}{$\uMsq$}
\newcommand{\ueqMcu}{$\uMcu$}
\newcommand{\ueqS}{$\uS$}
\newcommand{\ueqJ}{$\uJ$}
\newcommand{\ueqK}{$\uK$}
\newcommand{\ueqDEGREE}{$\uDEGREE$}
\newcommand{\ueqCELS}{$\uCELS$}
\newcommand{\ueqW}{$\uW$}
\newcommand{\ueqMW}{$\uMW$}
\newcommand{\ueqKW}{$\uKW$}
\newcommand{\ueqWH}{$\uWH$}
\newcommand{\ueqKWH}{$\uKWH$}
\newcommand{\ueqCCY}{$\uCCY$}
\newcommand{\ueqCZK}{$\uCZK$}
\newcommand{\ueqPERCENT}{$\uPERCENT$}
\newcommand{\ueqNOUNIT}{$\uNOUNIT$}
\newcommand{\ueqYEAR}{$\uYEAR$}
\newcommand{\ueqMONTH}{$\uMONTH$}
\newcommand{\ueqHOUR}{$\uHOUR$}
\newcommand{\ueqLM}{$\uLM$}
\newcommand{\ueqHinv}{$\uHinv$}

\newcommand{\ueqJperK}{$\uJperK$}
\newcommand{\ueqKperM}{$\uKperM$}
\newcommand{\ueqKperS}{$\uKperS$}
\newcommand{\ueqMsqperS}{$\uMsqperS$}
\newcommand{\ueqWperMsq}{$\uWperMsq$}
\newcommand{\ueqWperMperK}{$\uWperMperK$}
\newcommand{\ueqJperKGperK}{$\uJperKGperK$}
\newcommand{\ueqKperMsq}{$\uKperMsq$}
\newcommand{\ueqJperSperMperK}{$\uJperSperMperK$}
\newcommand{\ueqKGperMcu}{$\uKGperMcu$}
\newcommand{\ueqMsqKperW}{$\uMsqKperW$}
\newcommand{\ueqWperMsqperK}{$\uWperMsqperK$}
\newcommand{\ueqKWHperMsqperYEAR}{$\uKWHperMsqperYEAR$}
\newcommand{\ueqKWHperMsq}{$\uKWHperMsq$}
\newcommand{\ueqLMperW}{$\uLMperW$}
\newcommand{\ueqKWperMsq}{$\uKWperMsq$}
\newcommand{\ueqCCYperYEAR}{$\uCCYperYEAR$}
\newcommand{\ueqKWHperYEAR}{$\uKWHperYEAR$}
\newcommand{\ueqNOUNITperYEAR}{$\uNOUNITperYEAR$}
\newcommand{\ueqCCYperKWH}{$\uCCYperKWH$}
\newcommand{\ueqCZKperYEAR}{$\uCZKperYEAR$}
\newcommand{\ueqCZKperKWH}{$\uCZKperKWH$}
\newcommand{\ueqCZKperMONTH}{$\uCZKperMONTH$}


\maketitle

\tableofcontents



\section*{Znacky}

\begin{itemize}
    \item $c$ - Měrná tepelná kapacita (\ueqJperKGperK)
    \item $\rho$ - Hustota (kg / m3)
\end{itemize}

\newpage



\section*{Příklad 1}

test

\begin{itemize}
    \item test
    \item test
\end{itemize}



\section*{Příklad 2 - Tepelná kapacita materiálů}

\textbf{Zadání}

Na jakou teplotu by energie potřebná k vaření vody ohřála stejnou hmotnost / objem oceli a zlata? Voda je ohřívána z 10 \ueqCELS \myFS na 100 \ueqCELS.

\begin{table}[H]
    \centering
    \begin{tabular}{l|ll}
        \hline
        Mateiál    & $\rho$ (kg / m3) & $c$ (\ueqJperKGperK) \\
        \hline
        Voda (H2O) & 1 000            & 4 186                \\
        Ocel       & 7 750 ??         & 450 ??               \\
        Zlato      & 19 320           & 129                  \\
        \hline
    \end{tabular}
    \caption {Hustota a měrná tepelná kapacita materiálů}
\end{table}

\textbf{Řešení}

Měrná tepelná kapacita je definována jako množství tepla, které je potřeba k ohřátí jednoho kilogramu látky o jeden stupeň Kelvina:
\begin{equation}
    Q = m \cdot c \cdot \Delta T,
\end{equation}
kde:
\begin{itemize}
    \item $Q$ - množství tepla (J),
    \item $m$ - hmotnost (kg),
    \item $c$ - měrná tepelná kapacita (\ueqJperKGperK),
    \item $\Delta T$ - změna teploty (K).
\end{itemize}

Rozdíl teplot pro vodu je:
$$
    \Delta T_{H2O} = 100 \myFS \uCELS - 10 \myFS \uCELS = 90 \myFS \uK.
$$

\textbf{Řeěení pro stejnou hmotnost}
$$
    m_{H2O} \cdot c_{H2O} \cdot \Delta T_{H2O} = m_{ocel} \cdot c_{ocel} \cdot \Delta T_{ocel} = m_{zlato} \cdot c_{zlato} \cdot \Delta T_{zlato}.
$$

A zároveň pro tento příklad hmotnosti jsou stejné:
$$
    m_{H2O} = m_{ocel} = m_{zlato} = m.
$$

Tedy:
$$
    m \cdot c_{H2O} \cdot \Delta T_{H2O} = m \cdot c_{ocel} \cdot \Delta T_{ocel} = m \cdot c_{zlato} \cdot \Delta T_{zlato}.
$$

Rovnici lze vydělit hmotností $m$ ($m >$ 0) a dostaneme:
$$
    c_{H2O} \cdot \Delta T_{H2O} = c_{ocel} \cdot \Delta T_{ocel} = c_{zlato} \cdot \Delta T_{zlato}.
$$

Pro ocel platí:
$$
    \Delta T_{ocel} = \Delta T_{H2O} \cdot \frac{c_{H2O}}{c_{ocel}} = 90 \cdot \frac{4 186}{450} \approx 837,2 \myFS \uK.
$$

Pro zlato platí:
$$
    \Delta T_{zlato} = \Delta T_{H2O} \cdot \frac{c_{H2O}}{c_{zlato}} = 90 \cdot \frac{4 186}{129} \approx 2 920,5 \myFS \uK.
$$

\textbf{Řeěení pro stejný objem}

Hmotnost je definována jako:
$$
    m = V \cdot \rho,
$$
kde:
\begin{itemize}
    \item $m$ - hmotnost (kg),
    \item $V$ - objem (m3),
    \item $\rho$ - hustota (kg / m3).
\end{itemize}

Pro stejný objem platí:
$$
    V_{H2O} \cdot \rho_{H2O} \cdot c_{H2O} \cdot \Delta T_{H2O} = V_{ocel} \cdot \rho_{ocel} \cdot c_{ocel} \cdot \Delta T_{ocel} = V_{zlato} \cdot \rho_{zlato} \cdot c_{zlato} \cdot \Delta T_{zlato}.
$$

A zároveň pro tento příklad objemy jsou stejné:
$$
    V_{H2O} = V_{ocel} = V_{zlato} = V.
$$

Tedy:
$$
    V \cdot \rho_{H2O} \cdot c_{H2O} \cdot \Delta T_{H2O} = V \cdot \rho_{ocel} \cdot c_{ocel} \cdot \Delta T_{ocel} = V \cdot \rho_{zlato} \cdot c_{zlato} \cdot \Delta T_{zlato}.
$$

Rovnici lze vydělit objemem $V$ ($V >$ 0) a dostaneme:
$$
    \rho_{H2O} \cdot c_{H2O} \cdot \Delta T_{H2O} = \rho_{ocel} \cdot c_{ocel} \cdot \Delta T_{ocel} = \rho_{zlato} \cdot c_{zlato} \cdot \Delta T_{zlato}.
$$

Pro ocel platí:
$$
    \Delta T_{ocel} = \Delta T_{H2O} \cdot \frac{\rho_{H2O} \cdot c_{H2O}}{\rho_{ocel} \cdot c_{ocel}} = 90 \cdot \frac{1 000 \cdot 4 186}{7 750 \cdot 450} \approx 108,03 \myFS \uK.
$$

Pro zlato platí:
$$
    \Delta T_{zlato} = \Delta T_{H2O} \cdot \frac{\rho_{H2O} \cdot c_{H2O}}{\rho_{zlato} \cdot c_{zlato}} = 90 \cdot \frac{1 000 \cdot 4 186}{19 320 \cdot 129} \approx 1511,63 \myFS \uK.
$$



\section*{Příklad 3 - Průtokový ohřívač}



\end{document}
